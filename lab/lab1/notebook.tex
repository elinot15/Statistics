
% Default to the notebook output style

    


% Inherit from the specified cell style.




    
\documentclass[11pt]{article}

    
    
    \usepackage[T1]{fontenc}
    % Nicer default font (+ math font) than Computer Modern for most use cases
    \usepackage{mathpazo}

    % Basic figure setup, for now with no caption control since it's done
    % automatically by Pandoc (which extracts ![](path) syntax from Markdown).
    \usepackage{graphicx}
    % We will generate all images so they have a width \maxwidth. This means
    % that they will get their normal width if they fit onto the page, but
    % are scaled down if they would overflow the margins.
    \makeatletter
    \def\maxwidth{\ifdim\Gin@nat@width>\linewidth\linewidth
    \else\Gin@nat@width\fi}
    \makeatother
    \let\Oldincludegraphics\includegraphics
    % Set max figure width to be 80% of text width, for now hardcoded.
    \renewcommand{\includegraphics}[1]{\Oldincludegraphics[width=.8\maxwidth]{#1}}
    % Ensure that by default, figures have no caption (until we provide a
    % proper Figure object with a Caption API and a way to capture that
    % in the conversion process - todo).
    \usepackage{caption}
    \DeclareCaptionLabelFormat{nolabel}{}
    \captionsetup{labelformat=nolabel}

    \usepackage{adjustbox} % Used to constrain images to a maximum size 
    \usepackage{xcolor} % Allow colors to be defined
    \usepackage{enumerate} % Needed for markdown enumerations to work
    \usepackage{geometry} % Used to adjust the document margins
    \usepackage{amsmath} % Equations
    \usepackage{amssymb} % Equations
    \usepackage{textcomp} % defines textquotesingle
    % Hack from http://tex.stackexchange.com/a/47451/13684:
    \AtBeginDocument{%
        \def\PYZsq{\textquotesingle}% Upright quotes in Pygmentized code
    }
    \usepackage{upquote} % Upright quotes for verbatim code
    \usepackage{eurosym} % defines \euro
    \usepackage[mathletters]{ucs} % Extended unicode (utf-8) support
    \usepackage[utf8x]{inputenc} % Allow utf-8 characters in the tex document
    \usepackage{fancyvrb} % verbatim replacement that allows latex
    \usepackage{grffile} % extends the file name processing of package graphics 
                         % to support a larger range 
    % The hyperref package gives us a pdf with properly built
    % internal navigation ('pdf bookmarks' for the table of contents,
    % internal cross-reference links, web links for URLs, etc.)
    \usepackage{hyperref}
    \usepackage{longtable} % longtable support required by pandoc >1.10
    \usepackage{booktabs}  % table support for pandoc > 1.12.2
    \usepackage[inline]{enumitem} % IRkernel/repr support (it uses the enumerate* environment)
    \usepackage[normalem]{ulem} % ulem is needed to support strikethroughs (\sout)
                                % normalem makes italics be italics, not underlines
    

    
    
    % Colors for the hyperref package
    \definecolor{urlcolor}{rgb}{0,.145,.698}
    \definecolor{linkcolor}{rgb}{.71,0.21,0.01}
    \definecolor{citecolor}{rgb}{.12,.54,.11}

    % ANSI colors
    \definecolor{ansi-black}{HTML}{3E424D}
    \definecolor{ansi-black-intense}{HTML}{282C36}
    \definecolor{ansi-red}{HTML}{E75C58}
    \definecolor{ansi-red-intense}{HTML}{B22B31}
    \definecolor{ansi-green}{HTML}{00A250}
    \definecolor{ansi-green-intense}{HTML}{007427}
    \definecolor{ansi-yellow}{HTML}{DDB62B}
    \definecolor{ansi-yellow-intense}{HTML}{B27D12}
    \definecolor{ansi-blue}{HTML}{208FFB}
    \definecolor{ansi-blue-intense}{HTML}{0065CA}
    \definecolor{ansi-magenta}{HTML}{D160C4}
    \definecolor{ansi-magenta-intense}{HTML}{A03196}
    \definecolor{ansi-cyan}{HTML}{60C6C8}
    \definecolor{ansi-cyan-intense}{HTML}{258F8F}
    \definecolor{ansi-white}{HTML}{C5C1B4}
    \definecolor{ansi-white-intense}{HTML}{A1A6B2}

    % commands and environments needed by pandoc snippets
    % extracted from the output of `pandoc -s`
    \providecommand{\tightlist}{%
      \setlength{\itemsep}{0pt}\setlength{\parskip}{0pt}}
    \DefineVerbatimEnvironment{Highlighting}{Verbatim}{commandchars=\\\{\}}
    % Add ',fontsize=\small' for more characters per line
    \newenvironment{Shaded}{}{}
    \newcommand{\KeywordTok}[1]{\textcolor[rgb]{0.00,0.44,0.13}{\textbf{{#1}}}}
    \newcommand{\DataTypeTok}[1]{\textcolor[rgb]{0.56,0.13,0.00}{{#1}}}
    \newcommand{\DecValTok}[1]{\textcolor[rgb]{0.25,0.63,0.44}{{#1}}}
    \newcommand{\BaseNTok}[1]{\textcolor[rgb]{0.25,0.63,0.44}{{#1}}}
    \newcommand{\FloatTok}[1]{\textcolor[rgb]{0.25,0.63,0.44}{{#1}}}
    \newcommand{\CharTok}[1]{\textcolor[rgb]{0.25,0.44,0.63}{{#1}}}
    \newcommand{\StringTok}[1]{\textcolor[rgb]{0.25,0.44,0.63}{{#1}}}
    \newcommand{\CommentTok}[1]{\textcolor[rgb]{0.38,0.63,0.69}{\textit{{#1}}}}
    \newcommand{\OtherTok}[1]{\textcolor[rgb]{0.00,0.44,0.13}{{#1}}}
    \newcommand{\AlertTok}[1]{\textcolor[rgb]{1.00,0.00,0.00}{\textbf{{#1}}}}
    \newcommand{\FunctionTok}[1]{\textcolor[rgb]{0.02,0.16,0.49}{{#1}}}
    \newcommand{\RegionMarkerTok}[1]{{#1}}
    \newcommand{\ErrorTok}[1]{\textcolor[rgb]{1.00,0.00,0.00}{\textbf{{#1}}}}
    \newcommand{\NormalTok}[1]{{#1}}
    
    % Additional commands for more recent versions of Pandoc
    \newcommand{\ConstantTok}[1]{\textcolor[rgb]{0.53,0.00,0.00}{{#1}}}
    \newcommand{\SpecialCharTok}[1]{\textcolor[rgb]{0.25,0.44,0.63}{{#1}}}
    \newcommand{\VerbatimStringTok}[1]{\textcolor[rgb]{0.25,0.44,0.63}{{#1}}}
    \newcommand{\SpecialStringTok}[1]{\textcolor[rgb]{0.73,0.40,0.53}{{#1}}}
    \newcommand{\ImportTok}[1]{{#1}}
    \newcommand{\DocumentationTok}[1]{\textcolor[rgb]{0.73,0.13,0.13}{\textit{{#1}}}}
    \newcommand{\AnnotationTok}[1]{\textcolor[rgb]{0.38,0.63,0.69}{\textbf{\textit{{#1}}}}}
    \newcommand{\CommentVarTok}[1]{\textcolor[rgb]{0.38,0.63,0.69}{\textbf{\textit{{#1}}}}}
    \newcommand{\VariableTok}[1]{\textcolor[rgb]{0.10,0.09,0.49}{{#1}}}
    \newcommand{\ControlFlowTok}[1]{\textcolor[rgb]{0.00,0.44,0.13}{\textbf{{#1}}}}
    \newcommand{\OperatorTok}[1]{\textcolor[rgb]{0.40,0.40,0.40}{{#1}}}
    \newcommand{\BuiltInTok}[1]{{#1}}
    \newcommand{\ExtensionTok}[1]{{#1}}
    \newcommand{\PreprocessorTok}[1]{\textcolor[rgb]{0.74,0.48,0.00}{{#1}}}
    \newcommand{\AttributeTok}[1]{\textcolor[rgb]{0.49,0.56,0.16}{{#1}}}
    \newcommand{\InformationTok}[1]{\textcolor[rgb]{0.38,0.63,0.69}{\textbf{\textit{{#1}}}}}
    \newcommand{\WarningTok}[1]{\textcolor[rgb]{0.38,0.63,0.69}{\textbf{\textit{{#1}}}}}
    
    
    % Define a nice break command that doesn't care if a line doesn't already
    % exist.
    \def\br{\hspace*{\fill} \\* }
    % Math Jax compatability definitions
    \def\gt{>}
    \def\lt{<}
    % Document parameters
    \title{solution\_python}
    
    
    

    % Pygments definitions
    
\makeatletter
\def\PY@reset{\let\PY@it=\relax \let\PY@bf=\relax%
    \let\PY@ul=\relax \let\PY@tc=\relax%
    \let\PY@bc=\relax \let\PY@ff=\relax}
\def\PY@tok#1{\csname PY@tok@#1\endcsname}
\def\PY@toks#1+{\ifx\relax#1\empty\else%
    \PY@tok{#1}\expandafter\PY@toks\fi}
\def\PY@do#1{\PY@bc{\PY@tc{\PY@ul{%
    \PY@it{\PY@bf{\PY@ff{#1}}}}}}}
\def\PY#1#2{\PY@reset\PY@toks#1+\relax+\PY@do{#2}}

\expandafter\def\csname PY@tok@w\endcsname{\def\PY@tc##1{\textcolor[rgb]{0.73,0.73,0.73}{##1}}}
\expandafter\def\csname PY@tok@c\endcsname{\let\PY@it=\textit\def\PY@tc##1{\textcolor[rgb]{0.25,0.50,0.50}{##1}}}
\expandafter\def\csname PY@tok@cp\endcsname{\def\PY@tc##1{\textcolor[rgb]{0.74,0.48,0.00}{##1}}}
\expandafter\def\csname PY@tok@k\endcsname{\let\PY@bf=\textbf\def\PY@tc##1{\textcolor[rgb]{0.00,0.50,0.00}{##1}}}
\expandafter\def\csname PY@tok@kp\endcsname{\def\PY@tc##1{\textcolor[rgb]{0.00,0.50,0.00}{##1}}}
\expandafter\def\csname PY@tok@kt\endcsname{\def\PY@tc##1{\textcolor[rgb]{0.69,0.00,0.25}{##1}}}
\expandafter\def\csname PY@tok@o\endcsname{\def\PY@tc##1{\textcolor[rgb]{0.40,0.40,0.40}{##1}}}
\expandafter\def\csname PY@tok@ow\endcsname{\let\PY@bf=\textbf\def\PY@tc##1{\textcolor[rgb]{0.67,0.13,1.00}{##1}}}
\expandafter\def\csname PY@tok@nb\endcsname{\def\PY@tc##1{\textcolor[rgb]{0.00,0.50,0.00}{##1}}}
\expandafter\def\csname PY@tok@nf\endcsname{\def\PY@tc##1{\textcolor[rgb]{0.00,0.00,1.00}{##1}}}
\expandafter\def\csname PY@tok@nc\endcsname{\let\PY@bf=\textbf\def\PY@tc##1{\textcolor[rgb]{0.00,0.00,1.00}{##1}}}
\expandafter\def\csname PY@tok@nn\endcsname{\let\PY@bf=\textbf\def\PY@tc##1{\textcolor[rgb]{0.00,0.00,1.00}{##1}}}
\expandafter\def\csname PY@tok@ne\endcsname{\let\PY@bf=\textbf\def\PY@tc##1{\textcolor[rgb]{0.82,0.25,0.23}{##1}}}
\expandafter\def\csname PY@tok@nv\endcsname{\def\PY@tc##1{\textcolor[rgb]{0.10,0.09,0.49}{##1}}}
\expandafter\def\csname PY@tok@no\endcsname{\def\PY@tc##1{\textcolor[rgb]{0.53,0.00,0.00}{##1}}}
\expandafter\def\csname PY@tok@nl\endcsname{\def\PY@tc##1{\textcolor[rgb]{0.63,0.63,0.00}{##1}}}
\expandafter\def\csname PY@tok@ni\endcsname{\let\PY@bf=\textbf\def\PY@tc##1{\textcolor[rgb]{0.60,0.60,0.60}{##1}}}
\expandafter\def\csname PY@tok@na\endcsname{\def\PY@tc##1{\textcolor[rgb]{0.49,0.56,0.16}{##1}}}
\expandafter\def\csname PY@tok@nt\endcsname{\let\PY@bf=\textbf\def\PY@tc##1{\textcolor[rgb]{0.00,0.50,0.00}{##1}}}
\expandafter\def\csname PY@tok@nd\endcsname{\def\PY@tc##1{\textcolor[rgb]{0.67,0.13,1.00}{##1}}}
\expandafter\def\csname PY@tok@s\endcsname{\def\PY@tc##1{\textcolor[rgb]{0.73,0.13,0.13}{##1}}}
\expandafter\def\csname PY@tok@sd\endcsname{\let\PY@it=\textit\def\PY@tc##1{\textcolor[rgb]{0.73,0.13,0.13}{##1}}}
\expandafter\def\csname PY@tok@si\endcsname{\let\PY@bf=\textbf\def\PY@tc##1{\textcolor[rgb]{0.73,0.40,0.53}{##1}}}
\expandafter\def\csname PY@tok@se\endcsname{\let\PY@bf=\textbf\def\PY@tc##1{\textcolor[rgb]{0.73,0.40,0.13}{##1}}}
\expandafter\def\csname PY@tok@sr\endcsname{\def\PY@tc##1{\textcolor[rgb]{0.73,0.40,0.53}{##1}}}
\expandafter\def\csname PY@tok@ss\endcsname{\def\PY@tc##1{\textcolor[rgb]{0.10,0.09,0.49}{##1}}}
\expandafter\def\csname PY@tok@sx\endcsname{\def\PY@tc##1{\textcolor[rgb]{0.00,0.50,0.00}{##1}}}
\expandafter\def\csname PY@tok@m\endcsname{\def\PY@tc##1{\textcolor[rgb]{0.40,0.40,0.40}{##1}}}
\expandafter\def\csname PY@tok@gh\endcsname{\let\PY@bf=\textbf\def\PY@tc##1{\textcolor[rgb]{0.00,0.00,0.50}{##1}}}
\expandafter\def\csname PY@tok@gu\endcsname{\let\PY@bf=\textbf\def\PY@tc##1{\textcolor[rgb]{0.50,0.00,0.50}{##1}}}
\expandafter\def\csname PY@tok@gd\endcsname{\def\PY@tc##1{\textcolor[rgb]{0.63,0.00,0.00}{##1}}}
\expandafter\def\csname PY@tok@gi\endcsname{\def\PY@tc##1{\textcolor[rgb]{0.00,0.63,0.00}{##1}}}
\expandafter\def\csname PY@tok@gr\endcsname{\def\PY@tc##1{\textcolor[rgb]{1.00,0.00,0.00}{##1}}}
\expandafter\def\csname PY@tok@ge\endcsname{\let\PY@it=\textit}
\expandafter\def\csname PY@tok@gs\endcsname{\let\PY@bf=\textbf}
\expandafter\def\csname PY@tok@gp\endcsname{\let\PY@bf=\textbf\def\PY@tc##1{\textcolor[rgb]{0.00,0.00,0.50}{##1}}}
\expandafter\def\csname PY@tok@go\endcsname{\def\PY@tc##1{\textcolor[rgb]{0.53,0.53,0.53}{##1}}}
\expandafter\def\csname PY@tok@gt\endcsname{\def\PY@tc##1{\textcolor[rgb]{0.00,0.27,0.87}{##1}}}
\expandafter\def\csname PY@tok@err\endcsname{\def\PY@bc##1{\setlength{\fboxsep}{0pt}\fcolorbox[rgb]{1.00,0.00,0.00}{1,1,1}{\strut ##1}}}
\expandafter\def\csname PY@tok@kc\endcsname{\let\PY@bf=\textbf\def\PY@tc##1{\textcolor[rgb]{0.00,0.50,0.00}{##1}}}
\expandafter\def\csname PY@tok@kd\endcsname{\let\PY@bf=\textbf\def\PY@tc##1{\textcolor[rgb]{0.00,0.50,0.00}{##1}}}
\expandafter\def\csname PY@tok@kn\endcsname{\let\PY@bf=\textbf\def\PY@tc##1{\textcolor[rgb]{0.00,0.50,0.00}{##1}}}
\expandafter\def\csname PY@tok@kr\endcsname{\let\PY@bf=\textbf\def\PY@tc##1{\textcolor[rgb]{0.00,0.50,0.00}{##1}}}
\expandafter\def\csname PY@tok@bp\endcsname{\def\PY@tc##1{\textcolor[rgb]{0.00,0.50,0.00}{##1}}}
\expandafter\def\csname PY@tok@fm\endcsname{\def\PY@tc##1{\textcolor[rgb]{0.00,0.00,1.00}{##1}}}
\expandafter\def\csname PY@tok@vc\endcsname{\def\PY@tc##1{\textcolor[rgb]{0.10,0.09,0.49}{##1}}}
\expandafter\def\csname PY@tok@vg\endcsname{\def\PY@tc##1{\textcolor[rgb]{0.10,0.09,0.49}{##1}}}
\expandafter\def\csname PY@tok@vi\endcsname{\def\PY@tc##1{\textcolor[rgb]{0.10,0.09,0.49}{##1}}}
\expandafter\def\csname PY@tok@vm\endcsname{\def\PY@tc##1{\textcolor[rgb]{0.10,0.09,0.49}{##1}}}
\expandafter\def\csname PY@tok@sa\endcsname{\def\PY@tc##1{\textcolor[rgb]{0.73,0.13,0.13}{##1}}}
\expandafter\def\csname PY@tok@sb\endcsname{\def\PY@tc##1{\textcolor[rgb]{0.73,0.13,0.13}{##1}}}
\expandafter\def\csname PY@tok@sc\endcsname{\def\PY@tc##1{\textcolor[rgb]{0.73,0.13,0.13}{##1}}}
\expandafter\def\csname PY@tok@dl\endcsname{\def\PY@tc##1{\textcolor[rgb]{0.73,0.13,0.13}{##1}}}
\expandafter\def\csname PY@tok@s2\endcsname{\def\PY@tc##1{\textcolor[rgb]{0.73,0.13,0.13}{##1}}}
\expandafter\def\csname PY@tok@sh\endcsname{\def\PY@tc##1{\textcolor[rgb]{0.73,0.13,0.13}{##1}}}
\expandafter\def\csname PY@tok@s1\endcsname{\def\PY@tc##1{\textcolor[rgb]{0.73,0.13,0.13}{##1}}}
\expandafter\def\csname PY@tok@mb\endcsname{\def\PY@tc##1{\textcolor[rgb]{0.40,0.40,0.40}{##1}}}
\expandafter\def\csname PY@tok@mf\endcsname{\def\PY@tc##1{\textcolor[rgb]{0.40,0.40,0.40}{##1}}}
\expandafter\def\csname PY@tok@mh\endcsname{\def\PY@tc##1{\textcolor[rgb]{0.40,0.40,0.40}{##1}}}
\expandafter\def\csname PY@tok@mi\endcsname{\def\PY@tc##1{\textcolor[rgb]{0.40,0.40,0.40}{##1}}}
\expandafter\def\csname PY@tok@il\endcsname{\def\PY@tc##1{\textcolor[rgb]{0.40,0.40,0.40}{##1}}}
\expandafter\def\csname PY@tok@mo\endcsname{\def\PY@tc##1{\textcolor[rgb]{0.40,0.40,0.40}{##1}}}
\expandafter\def\csname PY@tok@ch\endcsname{\let\PY@it=\textit\def\PY@tc##1{\textcolor[rgb]{0.25,0.50,0.50}{##1}}}
\expandafter\def\csname PY@tok@cm\endcsname{\let\PY@it=\textit\def\PY@tc##1{\textcolor[rgb]{0.25,0.50,0.50}{##1}}}
\expandafter\def\csname PY@tok@cpf\endcsname{\let\PY@it=\textit\def\PY@tc##1{\textcolor[rgb]{0.25,0.50,0.50}{##1}}}
\expandafter\def\csname PY@tok@c1\endcsname{\let\PY@it=\textit\def\PY@tc##1{\textcolor[rgb]{0.25,0.50,0.50}{##1}}}
\expandafter\def\csname PY@tok@cs\endcsname{\let\PY@it=\textit\def\PY@tc##1{\textcolor[rgb]{0.25,0.50,0.50}{##1}}}

\def\PYZbs{\char`\\}
\def\PYZus{\char`\_}
\def\PYZob{\char`\{}
\def\PYZcb{\char`\}}
\def\PYZca{\char`\^}
\def\PYZam{\char`\&}
\def\PYZlt{\char`\<}
\def\PYZgt{\char`\>}
\def\PYZsh{\char`\#}
\def\PYZpc{\char`\%}
\def\PYZdl{\char`\$}
\def\PYZhy{\char`\-}
\def\PYZsq{\char`\'}
\def\PYZdq{\char`\"}
\def\PYZti{\char`\~}
% for compatibility with earlier versions
\def\PYZat{@}
\def\PYZlb{[}
\def\PYZrb{]}
\makeatother


    % Exact colors from NB
    \definecolor{incolor}{rgb}{0.0, 0.0, 0.5}
    \definecolor{outcolor}{rgb}{0.545, 0.0, 0.0}



    
    % Prevent overflowing lines due to hard-to-break entities
    \sloppy 
    % Setup hyperref package
    \hypersetup{
      breaklinks=true,  % so long urls are correctly broken across lines
      colorlinks=true,
      urlcolor=urlcolor,
      linkcolor=linkcolor,
      citecolor=citecolor,
      }
    % Slightly bigger margins than the latex defaults
    
    \geometry{verbose,tmargin=1in,bmargin=1in,lmargin=1in,rmargin=1in}
    
    

    \begin{document}
    
    
    \maketitle
    
    

    
    \hypertarget{esercizio-3}{%
\section{ESERCIZIO 3}\label{esercizio-3}}

Prima di cominciare ricodate che il comando \texttt{help(•)} può esserve
utilizzato per accedere alla documentazione relativa all'argomento.\\
Poi, useremo alcune librerie di python, tra cui \texttt{numpy},
\texttt{pandas}, \texttt{scipy} e \texttt{matplotlib}, per importarle è
sufficiente utilizzare la keywork \texttt{import} .\\
Con \texttt{type(•)} potete controllare il tipo dell'oggetto che
inserite come argomento. - \texttt{pandas.read.csv(•)} è il metodo per
caricare file csv. - il primo argomento è il path al file da caricare. -
l'argomento \texttt{sep} indica il separatore tra i campi. - l'argomento
\texttt{decimal} indica il separatore per il numeri float.

    \begin{Verbatim}[commandchars=\\\{\}]
{\color{incolor}In [{\color{incolor}2}]:} \PY{k+kn}{import} \PY{n+nn}{pandas}
        \PY{k+kn}{import} \PY{n+nn}{matplotlib}\PY{n+nn}{.}\PY{n+nn}{pyplot} \PY{k}{as} \PY{n+nn}{plt}
        \PY{n}{data} \PY{o}{=} \PY{n}{pandas}\PY{o}{.}\PY{n}{read\PYZus{}csv}\PY{p}{(}\PY{l+s+s2}{\PYZdq{}}\PY{l+s+s2}{carsharing.csv}\PY{l+s+s2}{\PYZdq{}}\PY{p}{,} \PY{n}{sep}\PY{o}{=}\PY{l+s+s2}{\PYZdq{}}\PY{l+s+s2}{;}\PY{l+s+s2}{\PYZdq{}}\PY{p}{,} \PY{n}{decimal}\PY{o}{=}\PY{l+s+s2}{\PYZdq{}}\PY{l+s+s2}{,}\PY{l+s+s2}{\PYZdq{}}\PY{p}{)}
        \PY{n}{data}\PY{o}{.}\PY{n}{iloc}\PY{p}{[}\PY{p}{[}\PY{l+m+mi}{1}\PY{p}{,}\PY{l+m+mi}{3}\PY{p}{,}\PY{l+m+mi}{2}\PY{p}{,}\PY{l+m+mi}{4}\PY{p}{,}\PY{l+m+mi}{5}\PY{p}{]}\PY{p}{,}\PY{l+m+mi}{0}\PY{p}{:}\PY{l+m+mi}{7}\PY{p}{]}
\end{Verbatim}


\begin{Verbatim}[commandchars=\\\{\}]
{\color{outcolor}Out[{\color{outcolor}2}]:}    CarIdentifier TimeFrame  RushHour  PremiumCustomer  Distance  Time
        1            103   FRAME D         1                1       5.3  13.9
        3            110   FRAME D         1                1       2.8   5.0
        2            105   FRAME D         1               -1       0.4   4.1
        4            110   FRAME B         1               -1       2.7   5.6
        5            111   FRAME C         0               -1      11.8  13.2
\end{Verbatim}
            
    Una delle caratteristiche più importanti è l'indicizzazione, di seguito
alcune regole: - \texttt{data.iloc{[}•,•{]}} serve per indicizzare
utilizzando le interi. - \texttt{data.loc{[}•,•{]}} serve per
indicizzare utilizzando labels. - il primo argomento è per le righe, il
secondo per le colonne. - \texttt{:} indica tutta gli elementi. -
\texttt{a:b} indica tutti gli elementi partendo da \texttt{a} fino a
\texttt{b} escluso. - \texttt{a:} indica da \texttt{a} fino alla fine. -
\texttt{:b} indica dall'inizio fino alla fine. - \texttt{:-b} indica
dall'inizio fino alla fine, tolti gli ultimi \texttt{b} elementi. -
possono essere utilizzate delle liste \texttt{{[}1,2,3,4,42,69{]}} come
argomenti. - possono essere utilizzare delle liste di labes, per esempio
\texttt{{[}"Distance",\ "time"{]}}. - possono essere utilizzate
espressioni booleane, per esempio
\texttt{data.loc{[}data.Distance\ \textgreater{}\ 5,\ {[}"Distance",\ "Time"{]}{]}}
- se si indica una sola colonna o una sola riga senza utilizzare le list
(\texttt{{[}•{]}}) allora pandas ritorna, non un dataframe, ma una
series.

    \hypertarget{quanti-casi-contiene-il-file}{%
\subsection{3.1. Quanti casi contiene il
file?}\label{quanti-casi-contiene-il-file}}

    \begin{Verbatim}[commandchars=\\\{\}]
{\color{incolor}In [{\color{incolor}2}]:} \PY{n+nb}{print}\PY{p}{(}\PY{n}{data}\PY{o}{.}\PY{n}{shape}\PY{p}{[}\PY{l+m+mi}{0}\PY{p}{]}\PY{p}{)}
        \PY{n+nb}{print}\PY{p}{(}\PY{n+nb}{len}\PY{p}{(}\PY{n}{data}\PY{p}{)}\PY{p}{)}
\end{Verbatim}


    \begin{Verbatim}[commandchars=\\\{\}]
392
392

    \end{Verbatim}

    I precedenti sono modi equivalenti per conoscere il numero di righe:\\
- \texttt{data.shape{[}0{]}} ritorna il numero di elementi per tutte le
dimensioni.\\
- \texttt{len(•)} di un dataframe ritorna il numero di righe.

    \hypertarget{analizziamo-lutilizzo-del-servizio-di-car-sharing-nelle-diverse-fasce-orarie-caratteretimeframe-e-negli-orari-di-maggior-o-minor-traffico-carattere-rushhour}{%
\subsection{3.2. Analizziamo l'utilizzo del servizio di car sharing
nelle diverse fasce orarie (carattereTimeFrame) e negli orari di maggior
o minor traffico (carattere
RushHour)}\label{analizziamo-lutilizzo-del-servizio-di-car-sharing-nelle-diverse-fasce-orarie-caratteretimeframe-e-negli-orari-di-maggior-o-minor-traffico-carattere-rushhour}}

\hypertarget{il-caratteretimeframe-uxe8-nominale-ordinale-o-scalare-giustificate-la-risposta.}{%
\subsubsection{3.2.1. Il carattereTimeFrame è nominale, ordinale o
scalare? Giustificate la
risposta.}\label{il-caratteretimeframe-uxe8-nominale-ordinale-o-scalare-giustificate-la-risposta.}}

    \begin{Verbatim}[commandchars=\\\{\}]
{\color{incolor}In [{\color{incolor}3}]:} \PY{n}{data}\PY{o}{.}\PY{n}{loc}\PY{p}{[}\PY{p}{:}\PY{p}{,}\PY{l+s+s2}{\PYZdq{}}\PY{l+s+s2}{TimeFrame}\PY{l+s+s2}{\PYZdq{}}\PY{p}{]}\PY{o}{.}\PY{n}{unique}\PY{p}{(}\PY{p}{)}
\end{Verbatim}


\begin{Verbatim}[commandchars=\\\{\}]
{\color{outcolor}Out[{\color{outcolor}3}]:} array(['FRAME D', 'FRAME B', 'FRAME C', 'FRAME E', 'FRAME A'],
              dtype=object)
\end{Verbatim}
            
    TimeFrame ha carattere nominale in quanto non può essere stabilito un
ordinamento tra i suoi valori. \#\#\# 3.2.2. In quante fasce orarie è
stata suddivisa una giornata?

    \begin{Verbatim}[commandchars=\\\{\}]
{\color{incolor}In [{\color{incolor}4}]:} \PY{n}{data}\PY{o}{.}\PY{n}{loc}\PY{p}{[}\PY{p}{:}\PY{p}{,}\PY{l+s+s2}{\PYZdq{}}\PY{l+s+s2}{TimeFrame}\PY{l+s+s2}{\PYZdq{}}\PY{p}{]}\PY{o}{.}\PY{n}{nunique}\PY{p}{(}\PY{p}{)}
\end{Verbatim}


\begin{Verbatim}[commandchars=\\\{\}]
{\color{outcolor}Out[{\color{outcolor}4}]:} 5
\end{Verbatim}
            
    \hypertarget{in-quali-fasce-orarie-il-servizio-di-car-sharing-uxe8-stato-maggiormente-utilizzato}{%
\subsubsection{3.2.3 In quali fasce orarie il servizio di car sharing è
stato maggiormente
utilizzato?}\label{in-quali-fasce-orarie-il-servizio-di-car-sharing-uxe8-stato-maggiormente-utilizzato}}

    \begin{Verbatim}[commandchars=\\\{\}]
{\color{incolor}In [{\color{incolor}5}]:} \PY{n}{data}\PY{o}{.}\PY{n}{loc}\PY{p}{[}\PY{p}{:}\PY{p}{,}\PY{l+s+s2}{\PYZdq{}}\PY{l+s+s2}{TimeFrame}\PY{l+s+s2}{\PYZdq{}}\PY{p}{]}\PY{o}{.}\PY{n}{value\PYZus{}counts}\PY{p}{(}\PY{p}{)}
\end{Verbatim}


\begin{Verbatim}[commandchars=\\\{\}]
{\color{outcolor}Out[{\color{outcolor}5}]:} FRAME B    123
        FRAME C    107
        FRAME D     94
        FRAME A     47
        FRAME E     21
        Name: TimeFrame, dtype: int64
\end{Verbatim}
            
    \hypertarget{calcolate-la-tabella-delle-frequenze-congiunte-di-timeframe-e-rushhour.}{%
\subsubsection{3.2.4. Calcolate la tabella delle frequenze congiunte di
TimeFrame e
RushHour.}\label{calcolate-la-tabella-delle-frequenze-congiunte-di-timeframe-e-rushhour.}}

    \begin{Verbatim}[commandchars=\\\{\}]
{\color{incolor}In [{\color{incolor}6}]:} \PY{n}{a} \PY{o}{=} \PY{n}{data}\PY{o}{.}\PY{n}{loc}\PY{p}{[}\PY{p}{:}\PY{p}{,}\PY{p}{[}\PY{l+s+s2}{\PYZdq{}}\PY{l+s+s2}{TimeFrame}\PY{l+s+s2}{\PYZdq{}}\PY{p}{,}\PY{l+s+s2}{\PYZdq{}}\PY{l+s+s2}{RushHour}\PY{l+s+s2}{\PYZdq{}}\PY{p}{]}\PY{p}{]}
        \PY{n}{pandas}\PY{o}{.}\PY{n}{crosstab}\PY{p}{(}\PY{n}{data}\PY{o}{.}\PY{n}{loc}\PY{p}{[}\PY{p}{:}\PY{p}{,}\PY{l+s+s2}{\PYZdq{}}\PY{l+s+s2}{TimeFrame}\PY{l+s+s2}{\PYZdq{}}\PY{p}{]}\PY{p}{,}\PY{n}{data}\PY{o}{.}\PY{n}{loc}\PY{p}{[}\PY{p}{:}\PY{p}{,}\PY{l+s+s2}{\PYZdq{}}\PY{l+s+s2}{RushHour}\PY{l+s+s2}{\PYZdq{}}\PY{p}{]}\PY{p}{)}
\end{Verbatim}


\begin{Verbatim}[commandchars=\\\{\}]
{\color{outcolor}Out[{\color{outcolor}6}]:} RushHour     0    1
        TimeFrame          
        FRAME A     47    0
        FRAME B      0  123
        FRAME C    107    0
        FRAME D      0   94
        FRAME E     21    0
\end{Verbatim}
            
    \texttt{pandas.crosstab(•,•)} si occupa di calcolare la tabella di
contingenza.\\
Il primo argomento finirà per rappresentare le colonne. Il second
argomento finirà per rappresentare le righe. \#\#\# 3.2.5. Leggendo la
tabella calcolata al punto precedente determinate quali sono le
fasceorarie che corrispondono all'ora di punta. Decisamente il Frame A e
il B. \#\# 3.3. Consideriamo, solo in questo punto dell'esercizio, i
clienti che hanno aderito al programma Premium (Premium=1) \#\#\# 3.3.1.
Quanti sono?

    \begin{Verbatim}[commandchars=\\\{\}]
{\color{incolor}In [{\color{incolor}28}]:} \PY{n+nb}{len}\PY{p}{(}\PY{n}{data}\PY{o}{.}\PY{n}{loc}\PY{p}{[}\PY{n}{data}\PY{o}{.}\PY{n}{PremiumCustomer}\PY{o}{==}\PY{l+m+mi}{1}\PY{p}{,}\PY{p}{]}\PY{p}{)}
\end{Verbatim}


\begin{Verbatim}[commandchars=\\\{\}]
{\color{outcolor}Out[{\color{outcolor}28}]:} 227
\end{Verbatim}
            
    \hypertarget{calcolate-la-distanza-media-percorsa-in-un-tragitto-da-un-cliente-che-ha-aderito-al-programma-premium.}{%
\subsubsection{3.3.2. Calcolate la distanza media percorsa in un
tragitto da un cliente che ha aderito al programma
Premium.}\label{calcolate-la-distanza-media-percorsa-in-un-tragitto-da-un-cliente-che-ha-aderito-al-programma-premium.}}

    \begin{Verbatim}[commandchars=\\\{\}]
{\color{incolor}In [{\color{incolor}27}]:} \PY{n}{data}\PY{o}{.}\PY{n}{loc}\PY{p}{[}\PY{n}{data}\PY{o}{.}\PY{n}{PremiumCustomer}\PY{o}{==}\PY{l+m+mi}{1}\PY{p}{,}\PY{l+s+s2}{\PYZdq{}}\PY{l+s+s2}{Distance}\PY{l+s+s2}{\PYZdq{}}\PY{p}{]}\PY{o}{.}\PY{n}{mean}\PY{p}{(}\PY{p}{)}
\end{Verbatim}


\begin{Verbatim}[commandchars=\\\{\}]
{\color{outcolor}Out[{\color{outcolor}27}]:} 8.437444933920705
\end{Verbatim}
            
    \hypertarget{ritorniamo-a-considerare-il-dataset-completo-e-studiamo-la-distanza-percorsa-in-ciascunutilizzo-del-servizio-caratteredistance.}{%
\subsection{3.4. Ritorniamo a considerare il dataset completo e studiamo
la distanza percorsa in ciascunutilizzo del servizio
(carattereDistance).}\label{ritorniamo-a-considerare-il-dataset-completo-e-studiamo-la-distanza-percorsa-in-ciascunutilizzo-del-servizio-caratteredistance.}}

\hypertarget{il-carattere-distance-uxe8-nominale-ordinale-o-scalare-giustificate-la-risposta.}{%
\subsubsection{3.4.1. Il carattere Distance è nominale, ordinale o
scalare? Giustificate la
risposta.}\label{il-carattere-distance-uxe8-nominale-ordinale-o-scalare-giustificate-la-risposta.}}

La distanza può assumere valori in ℝ (è una variabile quantitative
continua), quindi scalare. \#\#\# 3.4.2. Tracciate il boxplot di tale
carattere.

    \begin{Verbatim}[commandchars=\\\{\}]
{\color{incolor}In [{\color{incolor}9}]:} \PY{n}{plot} \PY{o}{=} \PY{n}{pandas}\PY{o}{.}\PY{n}{DataFrame}\PY{o}{.}\PY{n}{boxplot}\PY{p}{(}\PY{n}{data}\PY{o}{.}\PY{n}{loc}\PY{p}{[}\PY{p}{:}\PY{p}{,}\PY{l+s+s2}{\PYZdq{}}\PY{l+s+s2}{Distance}\PY{l+s+s2}{\PYZdq{}}\PY{p}{]}\PY{p}{)}
\end{Verbatim}


    \begin{center}
    \adjustimage{max size={0.9\linewidth}{0.9\paperheight}}{output_19_0.png}
    \end{center}
    { \hspace*{\fill} \\}
    
    \hypertarget{in-base-allaspetto-del-grafico-ottenuto-al-punto-precedente-determinate-quali-sono-gli-indici-di-centralituxe0-e-di-dispersione-che-meglio-caratterizzano-la-distanza-percorsa-calcolandone-il-valore.}{%
\subsubsection{3.4.3. In base all'aspetto del grafico ottenuto al punto
precedente, determinate quali sono gli indici di centralità e di
dispersione che meglio caratterizzano la distanza percorsa, calcolandone
il
valore.}\label{in-base-allaspetto-del-grafico-ottenuto-al-punto-precedente-determinate-quali-sono-gli-indici-di-centralituxe0-e-di-dispersione-che-meglio-caratterizzano-la-distanza-percorsa-calcolandone-il-valore.}}

    \begin{Verbatim}[commandchars=\\\{\}]
{\color{incolor}In [{\color{incolor}3}]:} \PY{n+nb}{print}\PY{p}{(}\PY{l+s+s2}{\PYZdq{}}\PY{l+s+s2}{max:    }\PY{l+s+s2}{\PYZdq{}}\PY{p}{,} \PY{n}{data}\PY{o}{.}\PY{n}{loc}\PY{p}{[}\PY{p}{:}\PY{p}{,}\PY{l+s+s2}{\PYZdq{}}\PY{l+s+s2}{Distance}\PY{l+s+s2}{\PYZdq{}}\PY{p}{]}\PY{o}{.}\PY{n}{max}\PY{p}{(}\PY{p}{)}\PY{p}{)}
        \PY{n+nb}{print}\PY{p}{(}\PY{l+s+s2}{\PYZdq{}}\PY{l+s+s2}{q1:     }\PY{l+s+s2}{\PYZdq{}}\PY{p}{,} \PY{n}{data}\PY{o}{.}\PY{n}{loc}\PY{p}{[}\PY{p}{:}\PY{p}{,}\PY{l+s+s2}{\PYZdq{}}\PY{l+s+s2}{Distance}\PY{l+s+s2}{\PYZdq{}}\PY{p}{]}\PY{o}{.}\PY{n}{quantile}\PY{p}{(}\PY{l+m+mf}{0.25}\PY{p}{)}\PY{p}{)}
        \PY{n+nb}{print}\PY{p}{(}\PY{l+s+s2}{\PYZdq{}}\PY{l+s+s2}{median: }\PY{l+s+s2}{\PYZdq{}}\PY{p}{,} \PY{n}{data}\PY{o}{.}\PY{n}{loc}\PY{p}{[}\PY{p}{:}\PY{p}{,}\PY{l+s+s2}{\PYZdq{}}\PY{l+s+s2}{Distance}\PY{l+s+s2}{\PYZdq{}}\PY{p}{]}\PY{o}{.}\PY{n}{median}\PY{p}{(}\PY{p}{)}\PY{p}{)}
        \PY{n+nb}{print}\PY{p}{(}\PY{l+s+s2}{\PYZdq{}}\PY{l+s+s2}{q2:     }\PY{l+s+s2}{\PYZdq{}}\PY{p}{,} \PY{n}{data}\PY{o}{.}\PY{n}{loc}\PY{p}{[}\PY{p}{:}\PY{p}{,}\PY{l+s+s2}{\PYZdq{}}\PY{l+s+s2}{Distance}\PY{l+s+s2}{\PYZdq{}}\PY{p}{]}\PY{o}{.}\PY{n}{quantile}\PY{p}{(}\PY{l+m+mf}{0.50}\PY{p}{)}\PY{p}{)}
        \PY{n+nb}{print}\PY{p}{(}\PY{l+s+s2}{\PYZdq{}}\PY{l+s+s2}{q3:     }\PY{l+s+s2}{\PYZdq{}}\PY{p}{,} \PY{n}{data}\PY{o}{.}\PY{n}{loc}\PY{p}{[}\PY{p}{:}\PY{p}{,}\PY{l+s+s2}{\PYZdq{}}\PY{l+s+s2}{Distance}\PY{l+s+s2}{\PYZdq{}}\PY{p}{]}\PY{o}{.}\PY{n}{quantile}\PY{p}{(}\PY{l+m+mf}{0.75}\PY{p}{)}\PY{p}{)}
        \PY{n+nb}{print}\PY{p}{(}\PY{l+s+s2}{\PYZdq{}}\PY{l+s+s2}{min:    }\PY{l+s+s2}{\PYZdq{}}\PY{p}{,} \PY{n}{data}\PY{o}{.}\PY{n}{loc}\PY{p}{[}\PY{p}{:}\PY{p}{,}\PY{l+s+s2}{\PYZdq{}}\PY{l+s+s2}{Distance}\PY{l+s+s2}{\PYZdq{}}\PY{p}{]}\PY{o}{.}\PY{n}{min}\PY{p}{(}\PY{p}{)}\PY{p}{)}
        \PY{n+nb}{print}\PY{p}{(}\PY{l+s+s2}{\PYZdq{}}\PY{l+s+s2}{IQR:    }\PY{l+s+s2}{\PYZdq{}}\PY{p}{,} \PY{n}{data}\PY{o}{.}\PY{n}{loc}\PY{p}{[}\PY{p}{:}\PY{p}{,}\PY{l+s+s2}{\PYZdq{}}\PY{l+s+s2}{Distance}\PY{l+s+s2}{\PYZdq{}}\PY{p}{]}\PY{o}{.}\PY{n}{quantile}\PY{p}{(}\PY{l+m+mf}{0.75}\PY{p}{)}
              \PY{o}{\PYZhy{}}\PY{n}{data}\PY{o}{.}\PY{n}{loc}\PY{p}{[}\PY{p}{:}\PY{p}{,}\PY{l+s+s2}{\PYZdq{}}\PY{l+s+s2}{Distance}\PY{l+s+s2}{\PYZdq{}}\PY{p}{]}\PY{o}{.}\PY{n}{quantile}\PY{p}{(}\PY{l+m+mf}{0.25}\PY{p}{)}\PY{p}{)}
\end{Verbatim}


    \begin{Verbatim}[commandchars=\\\{\}]
max:     24.0
q1:      1.5750000000000002
median:  5.75
q2:      5.75
q3:      14.025
min:     0.1
IQR:     12.45

    \end{Verbatim}

    \hypertarget{riscontrate-una-relazione-tra-la-distanza-percorsa-e-il-tempo-in-caso-affermativo-caratterizzate-tale-relazione.-in-ogni-caso-giustificate-la-vostra-risposta-mostrando-un-grafico.}{%
\subsubsection{3.4.4. Riscontrate una relazione tra la distanza percorsa
e il Tempo? In caso affermativo, caratterizzate tale relazione. In ogni
caso giustificate la vostra risposta mostrando un
grafico.}\label{riscontrate-una-relazione-tra-la-distanza-percorsa-e-il-tempo-in-caso-affermativo-caratterizzate-tale-relazione.-in-ogni-caso-giustificate-la-vostra-risposta-mostrando-un-grafico.}}

    \begin{Verbatim}[commandchars=\\\{\}]
{\color{incolor}In [{\color{incolor}4}]:} \PY{k+kn}{import} \PY{n+nn}{matplotlib}\PY{n+nn}{.}\PY{n+nn}{pyplot} \PY{k}{as} \PY{n+nn}{plt}
        \PY{n}{fig}\PY{p}{,} \PY{n}{ax} \PY{o}{=} \PY{n}{plt}\PY{o}{.}\PY{n}{subplots}\PY{p}{(}\PY{n}{nrows}\PY{o}{=}\PY{l+m+mi}{1}\PY{p}{,} \PY{n}{ncols}\PY{o}{=}\PY{l+m+mi}{3}\PY{p}{)}
        \PY{n}{fig}\PY{o}{.}\PY{n}{set\PYZus{}size\PYZus{}inches}\PY{p}{(}\PY{l+m+mi}{15}\PY{p}{,}\PY{l+m+mi}{5}\PY{p}{)}
        \PY{n}{plot} \PY{o}{=} \PY{n}{data}\PY{o}{.}\PY{n}{plot}\PY{o}{.}\PY{n}{scatter}\PY{p}{(}\PY{n}{x}\PY{o}{=}\PY{l+s+s2}{\PYZdq{}}\PY{l+s+s2}{Distance}\PY{l+s+s2}{\PYZdq{}}\PY{p}{,}\PY{n}{y}\PY{o}{=}\PY{l+s+s2}{\PYZdq{}}\PY{l+s+s2}{Time}\PY{l+s+s2}{\PYZdq{}}\PY{p}{,}\PY{n}{ax} \PY{o}{=} \PY{n}{ax}\PY{p}{[}\PY{l+m+mi}{0}\PY{p}{]}\PY{p}{,}\PY{n}{title}\PY{o}{=}\PY{l+s+s2}{\PYZdq{}}\PY{l+s+s2}{full plot}\PY{l+s+s2}{\PYZdq{}}\PY{p}{)}
        \PY{n}{plot} \PY{o}{=} \PY{n}{data}\PY{o}{.}\PY{n}{loc}\PY{p}{[}\PY{n}{data}\PY{o}{.}\PY{n}{RushHour} \PY{o}{==} \PY{l+m+mi}{1}\PY{p}{,}\PY{p}{:}\PY{p}{]}\PY{o}{.}\PY{n}{plot}\PY{o}{.}\PY{n}{scatter}\PY{p}{(}\PY{n}{x}\PY{o}{=}\PY{l+s+s2}{\PYZdq{}}\PY{l+s+s2}{Distance}\PY{l+s+s2}{\PYZdq{}}\PY{p}{,}\PY{n}{y}\PY{o}{=}\PY{l+s+s2}{\PYZdq{}}\PY{l+s+s2}{Time}\PY{l+s+s2}{\PYZdq{}}\PY{p}{,}
                                                 \PY{n}{ax} \PY{o}{=} \PY{n}{ax}\PY{p}{[}\PY{l+m+mi}{1}\PY{p}{]}\PY{p}{,} \PY{n}{title}\PY{o}{=}\PY{l+s+s2}{\PYZdq{}}\PY{l+s+s2}{RushHour = 1}\PY{l+s+s2}{\PYZdq{}}\PY{p}{)}
        \PY{n}{plot} \PY{o}{=} \PY{n}{data}\PY{o}{.}\PY{n}{loc}\PY{p}{[}\PY{n}{data}\PY{o}{.}\PY{n}{RushHour} \PY{o}{==} \PY{l+m+mi}{0}\PY{p}{,}\PY{p}{:}\PY{p}{]}\PY{o}{.}\PY{n}{plot}\PY{o}{.}\PY{n}{scatter}\PY{p}{(}\PY{n}{x}\PY{o}{=}\PY{l+s+s2}{\PYZdq{}}\PY{l+s+s2}{Distance}\PY{l+s+s2}{\PYZdq{}}\PY{p}{,}\PY{n}{y}\PY{o}{=}\PY{l+s+s2}{\PYZdq{}}\PY{l+s+s2}{Time}\PY{l+s+s2}{\PYZdq{}}\PY{p}{,}
                                                \PY{n}{ax} \PY{o}{=} \PY{n}{ax}\PY{p}{[}\PY{l+m+mi}{2}\PY{p}{]}\PY{p}{,} \PY{n}{title}\PY{o}{=}\PY{l+s+s2}{\PYZdq{}}\PY{l+s+s2}{RushHour = 0}\PY{l+s+s2}{\PYZdq{}}\PY{p}{)}
\end{Verbatim}


    \begin{center}
    \adjustimage{max size={0.9\linewidth}{0.9\paperheight}}{output_23_0.png}
    \end{center}
    { \hspace*{\fill} \\}
    
    Per brevi distanze l'andamento è il medesimo.\\
Su distanze più lunghe si denotano due andamenti differenti. La RushHour
sembra essere responsabile delle differenze. \#\#\# 3.4.5. Calcolate
l'indice di correlazione tra la distanza e Tempo. Il valore
ottenutosupporta la risposta che avete dato al punto precedente?

    \begin{Verbatim}[commandchars=\\\{\}]
{\color{incolor}In [{\color{incolor}12}]:} \PY{n}{data}\PY{o}{.}\PY{n}{loc}\PY{p}{[}\PY{p}{:}\PY{p}{,}\PY{p}{[}\PY{l+s+s2}{\PYZdq{}}\PY{l+s+s2}{Distance}\PY{l+s+s2}{\PYZdq{}}\PY{p}{,}\PY{l+s+s2}{\PYZdq{}}\PY{l+s+s2}{Time}\PY{l+s+s2}{\PYZdq{}}\PY{p}{]}\PY{p}{]}\PY{o}{.}\PY{n}{corr}\PY{p}{(}\PY{p}{)}
\end{Verbatim}


\begin{Verbatim}[commandchars=\\\{\}]
{\color{outcolor}Out[{\color{outcolor}12}]:}           Distance      Time
         Distance  1.000000  0.627399
         Time      0.627399  1.000000
\end{Verbatim}
            
    la correlazione indica che al crescere di \texttt{Distance} la
\texttt{Time} cresce, come è ragionevole che sia.\\
\#\#\# 3.4.6. Tracciate, possibilmente nella stessa figura, il box plot
della distanza nel caso di utilizzo dell'auto in orario di punta
(RushHour=1) e in orario non di punta (RushHour=0).

    \begin{Verbatim}[commandchars=\\\{\}]
{\color{incolor}In [{\color{incolor}13}]:} \PY{n}{fig}\PY{p}{,} \PY{n}{ax} \PY{o}{=} \PY{n}{plt}\PY{o}{.}\PY{n}{subplots}\PY{p}{(}\PY{n}{nrows}\PY{o}{=}\PY{l+m+mi}{1}\PY{p}{,} \PY{n}{ncols}\PY{o}{=}\PY{l+m+mi}{2}\PY{p}{)}
         \PY{n}{fig}\PY{o}{.}\PY{n}{set\PYZus{}size\PYZus{}inches}\PY{p}{(}\PY{l+m+mi}{15}\PY{p}{,}\PY{l+m+mi}{5}\PY{p}{)}
         \PY{n}{plt0} \PY{o}{=} \PY{n}{pandas}\PY{o}{.}\PY{n}{DataFrame}\PY{o}{.}\PY{n}{boxplot}\PY{p}{(}\PY{n}{data}\PY{o}{.}\PY{n}{loc}\PY{p}{[}\PY{n}{data}\PY{o}{.}\PY{n}{RushHour}\PY{o}{==}\PY{l+m+mi}{1}\PY{p}{,}\PY{l+s+s2}{\PYZdq{}}\PY{l+s+s2}{Distance}\PY{l+s+s2}{\PYZdq{}}\PY{p}{]}\PY{p}{,}\PY{n}{ax}\PY{o}{=}\PY{n}{ax}\PY{p}{[}\PY{l+m+mi}{0}\PY{p}{]}\PY{p}{)}
         \PY{n}{plt1} \PY{o}{=} \PY{n}{pandas}\PY{o}{.}\PY{n}{DataFrame}\PY{o}{.}\PY{n}{boxplot}\PY{p}{(}\PY{n}{data}\PY{o}{.}\PY{n}{loc}\PY{p}{[}\PY{n}{data}\PY{o}{.}\PY{n}{RushHour}\PY{o}{==}\PY{l+m+mi}{0}\PY{p}{,}\PY{l+s+s2}{\PYZdq{}}\PY{l+s+s2}{Distance}\PY{l+s+s2}{\PYZdq{}}\PY{p}{]}\PY{p}{,}\PY{n}{ax}\PY{o}{=}\PY{n}{ax}\PY{p}{[}\PY{l+m+mi}{1}\PY{p}{]}\PY{p}{)}
\end{Verbatim}


    \begin{center}
    \adjustimage{max size={0.9\linewidth}{0.9\paperheight}}{output_27_0.png}
    \end{center}
    { \hspace*{\fill} \\}
    
    \hypertarget{ispezionando-i-due-grafici-ottenuti-al-punto-precedente-dite-se-negli-orari-di-puntasono-privilegiati-spostamenti-piuxf9-brevi-oppure-piuxf9-lunghi-rispetto-agli-orari-non-di-punta-giustificando-la-risposta.}{%
\subsubsection{3.4.7. Ispezionando i due grafici ottenuti al punto
precedente, dite se negli orari di puntasono privilegiati spostamenti
``più brevi'' oppure ``più lunghi'' rispetto agli orari non di punta,
giustificando la
risposta.}\label{ispezionando-i-due-grafici-ottenuti-al-punto-precedente-dite-se-negli-orari-di-puntasono-privilegiati-spostamenti-piuxf9-brevi-oppure-piuxf9-lunghi-rispetto-agli-orari-non-di-punta-giustificando-la-risposta.}}

Dal precedente grafico si può notare che:\\
- quando \texttt{RushHour=1} il 50\% degli spostamenti è ``più breve'',
tra 0 e 5.\\
- quando \texttt{RushHour=0} il 50\% degli spostamenti è ``più lungo'',
tra 10 e 20. - dall'istogramma si nota anche che la distanza presenta
una distribuzione bimodale, dove, le due mode corrispondono con i picchi
delle distribuzioni quando la \texttt{RushHour=0} e quando
\texttt{RushHour=1}.

Di conseguenza, durante gli orari di punta sono privilegiati spostamenti
brevi. \#\#\# 3.4.8. Tracciate, possibilmente nella stessa figura, il
box plot della distanza nel caso di utilizzo dell'auto da parte dei
clienti che hanno aderito al programma Premium (Premium=1) e di quelli
che non vi hanno aderito (Premium=-1).

    \begin{Verbatim}[commandchars=\\\{\}]
{\color{incolor}In [{\color{incolor}5}]:} \PY{n}{fig}\PY{p}{,} \PY{n}{ax} \PY{o}{=} \PY{n}{plt}\PY{o}{.}\PY{n}{subplots}\PY{p}{(}\PY{n}{nrows}\PY{o}{=}\PY{l+m+mi}{1}\PY{p}{,} \PY{n}{ncols}\PY{o}{=}\PY{l+m+mi}{2}\PY{p}{)}
        \PY{n}{fig}\PY{o}{.}\PY{n}{set\PYZus{}size\PYZus{}inches}\PY{p}{(}\PY{l+m+mi}{15}\PY{p}{,}\PY{l+m+mi}{5}\PY{p}{)}
        \PY{n}{plt0} \PY{o}{=} \PY{n}{pandas}\PY{o}{.}\PY{n}{DataFrame}\PY{o}{.}\PY{n}{boxplot}\PY{p}{(}\PY{n}{data}\PY{o}{.}\PY{n}{loc}\PY{p}{[}\PY{n}{data}\PY{o}{.}\PY{n}{PremiumCustomer}\PY{o}{==}\PY{o}{\PYZhy{}}\PY{l+m+mi}{1}\PY{p}{,}\PY{l+s+s2}{\PYZdq{}}\PY{l+s+s2}{Distance}\PY{l+s+s2}{\PYZdq{}}\PY{p}{]}\PY{p}{,}
                                        \PY{n}{ax}\PY{o}{=}\PY{n}{ax}\PY{p}{[}\PY{l+m+mi}{0}\PY{p}{]}\PY{p}{)}
        \PY{n}{plt1} \PY{o}{=} \PY{n}{pandas}\PY{o}{.}\PY{n}{DataFrame}\PY{o}{.}\PY{n}{boxplot}\PY{p}{(}\PY{n}{data}\PY{o}{.}\PY{n}{loc}\PY{p}{[}\PY{n}{data}\PY{o}{.}\PY{n}{PremiumCustomer}\PY{o}{==}\PY{l+m+mi}{1}\PY{p}{,}\PY{l+s+s2}{\PYZdq{}}\PY{l+s+s2}{Distance}\PY{l+s+s2}{\PYZdq{}}\PY{p}{]}\PY{p}{,}
                                        \PY{n}{ax}\PY{o}{=}\PY{n}{ax}\PY{p}{[}\PY{l+m+mi}{1}\PY{p}{]}\PY{p}{)}
\end{Verbatim}


    \begin{center}
    \adjustimage{max size={0.9\linewidth}{0.9\paperheight}}{output_29_0.png}
    \end{center}
    { \hspace*{\fill} \\}
    
    \hypertarget{ispezionando-i-due-grafici-ottenuti-al-punto-precedente-notate-una-grossa-differenzanelle-distanze-percorse-dai-clienti-dei-due-gruppi}{%
\subsection{3.4.9. Ispezionando i due grafici ottenuti al punto
precedente, notate una grossa differenzanelle distanze percorse dai
clienti dei due
gruppi?}\label{ispezionando-i-due-grafici-ottenuti-al-punto-precedente-notate-una-grossa-differenzanelle-distanze-percorse-dai-clienti-dei-due-gruppi}}

Gli utenti premium tendono a percorerre distanze più lunghe rispetto
agli utenti normali. \#\#\# 3.4.10. In Figura 1 è mostrato l'istogramma
della distanza percorsa. In tale grafico si può individuare la presenza
di due gruppi abbastanza distinti. I due gruppi sono relativi al tipo di
cliente (Premium=1 oppure Premium=-1) oppure all'orario di utilizzo del
veicolo (RushHour=1 oppure RushHour=0)? In altri termini, la distanza
percorsa dipende dal fatto che l'utente sia un cliente
Premium/non-Premium oppure dal fatto che l'utilizzo è avvenuto in orario
Rush/non-Rush? Suggerimento: per rispondere a questa domanda basta
ispezionarei boxplot prodotti nei punti precedenti di questo esercizio.

c'è un differenza nella distanza percorsa se l'utente è premium e
non-premium.\\
c'è un differenza nella distanza percorsa se l'utente sia nella RushHour
e non-RushHour.\\
nel secondo caso la differenza è più marcata e responsabile delle 2 mode
nell'istogramma della distanza.

    \begin{Verbatim}[commandchars=\\\{\}]
{\color{incolor}In [{\color{incolor}6}]:} \PY{n}{fig}\PY{p}{,} \PY{n}{ax} \PY{o}{=} \PY{n}{plt}\PY{o}{.}\PY{n}{subplots}\PY{p}{(}\PY{n}{nrows}\PY{o}{=}\PY{l+m+mi}{2}\PY{p}{,} \PY{n}{ncols}\PY{o}{=}\PY{l+m+mi}{2}\PY{p}{)}
        \PY{n}{fig}\PY{o}{.}\PY{n}{set\PYZus{}size\PYZus{}inches}\PY{p}{(}\PY{l+m+mi}{10}\PY{p}{,}\PY{l+m+mi}{10}\PY{p}{)}
        \PY{n}{plt0} \PY{o}{=} \PY{n}{pandas}\PY{o}{.}\PY{n}{DataFrame}\PY{o}{.}\PY{n}{hist}\PY{p}{(}\PY{n}{data}\PY{o}{.}\PY{n}{loc}\PY{p}{[}\PY{n}{data}\PY{o}{.}\PY{n}{RushHour} \PY{o}{==} \PY{l+m+mi}{1}\PY{p}{,}\PY{p}{:}\PY{p}{]}\PY{p}{,} \PY{n}{column}\PY{o}{=}\PY{l+s+s2}{\PYZdq{}}\PY{l+s+s2}{Distance}\PY{l+s+s2}{\PYZdq{}}\PY{p}{,}
                                     \PY{n}{ax}\PY{o}{=}\PY{n}{ax}\PY{p}{[}\PY{l+m+mi}{0}\PY{p}{,}\PY{l+m+mi}{0}\PY{p}{]}\PY{p}{)}
        \PY{n}{plt1} \PY{o}{=} \PY{n}{pandas}\PY{o}{.}\PY{n}{DataFrame}\PY{o}{.}\PY{n}{hist}\PY{p}{(}\PY{n}{data}\PY{o}{.}\PY{n}{loc}\PY{p}{[}\PY{n}{data}\PY{o}{.}\PY{n}{RushHour} \PY{o}{==} \PY{l+m+mi}{0}\PY{p}{,}\PY{p}{:}\PY{p}{]}\PY{p}{,} \PY{n}{column}\PY{o}{=}\PY{l+s+s2}{\PYZdq{}}\PY{l+s+s2}{Distance}\PY{l+s+s2}{\PYZdq{}}\PY{p}{,}
                                     \PY{n}{ax}\PY{o}{=}\PY{n}{ax}\PY{p}{[}\PY{l+m+mi}{0}\PY{p}{,}\PY{l+m+mi}{1}\PY{p}{]}\PY{p}{)}
        \PY{n}{plt2} \PY{o}{=} \PY{n}{pandas}\PY{o}{.}\PY{n}{DataFrame}\PY{o}{.}\PY{n}{hist}\PY{p}{(}\PY{n}{data}\PY{o}{.}\PY{n}{loc}\PY{p}{[}\PY{n}{data}\PY{o}{.}\PY{n}{PremiumCustomer} \PY{o}{==} \PY{l+m+mi}{1}\PY{p}{,}\PY{p}{:}\PY{p}{]}\PY{p}{,} \PY{n}{column}\PY{o}{=}\PY{l+s+s2}{\PYZdq{}}\PY{l+s+s2}{Distance}\PY{l+s+s2}{\PYZdq{}}\PY{p}{,}
                                     \PY{n}{ax}\PY{o}{=}\PY{n}{ax}\PY{p}{[}\PY{l+m+mi}{1}\PY{p}{,}\PY{l+m+mi}{0}\PY{p}{]}\PY{p}{)}
        \PY{n}{plt3} \PY{o}{=} \PY{n}{pandas}\PY{o}{.}\PY{n}{DataFrame}\PY{o}{.}\PY{n}{hist}\PY{p}{(}\PY{n}{data}\PY{o}{.}\PY{n}{loc}\PY{p}{[}\PY{n}{data}\PY{o}{.}\PY{n}{PremiumCustomer} \PY{o}{==} \PY{o}{\PYZhy{}}\PY{l+m+mi}{1}\PY{p}{,}\PY{p}{:}\PY{p}{]}\PY{p}{,} \PY{n}{column}\PY{o}{=}\PY{l+s+s2}{\PYZdq{}}\PY{l+s+s2}{Distance}\PY{l+s+s2}{\PYZdq{}}\PY{p}{,}
                                     \PY{n}{ax}\PY{o}{=}\PY{n}{ax}\PY{p}{[}\PY{l+m+mi}{1}\PY{p}{,}\PY{l+m+mi}{1}\PY{p}{]}\PY{p}{)}
\end{Verbatim}


    \begin{center}
    \adjustimage{max size={0.9\linewidth}{0.9\paperheight}}{output_31_0.png}
    \end{center}
    { \hspace*{\fill} \\}
    
    Questi 4 grafici evidenziano quello che è già stato notato. \#\#\#
3.4.11. Calcolate la distanza media nei due gruppi di orario (di
punta/non di punta) e commentate l'istogramma di Figura 1 utilizzando
queste due informazioni.

    \begin{Verbatim}[commandchars=\\\{\}]
{\color{incolor}In [{\color{incolor}7}]:} \PY{n+nb}{print}\PY{p}{(}\PY{l+s+s2}{\PYZdq{}}\PY{l+s+s2}{media in orario di punta    :}\PY{l+s+s2}{\PYZdq{}}\PY{p}{,} \PY{n}{data}\PY{o}{.}\PY{n}{loc}\PY{p}{[}\PY{n}{data}\PY{o}{.}\PY{n}{RushHour} \PY{o}{==} \PY{l+m+mi}{1}\PY{p}{,} \PY{l+s+s2}{\PYZdq{}}\PY{l+s+s2}{Distance}\PY{l+s+s2}{\PYZdq{}}\PY{p}{]}
              \PY{o}{.}\PY{n}{mean}\PY{p}{(}\PY{p}{)}\PY{p}{)}
        \PY{n+nb}{print}\PY{p}{(}\PY{l+s+s2}{\PYZdq{}}\PY{l+s+s2}{media in orario di non punta:}\PY{l+s+s2}{\PYZdq{}}\PY{p}{,} \PY{n}{data}\PY{o}{.}\PY{n}{loc}\PY{p}{[}\PY{n}{data}\PY{o}{.}\PY{n}{RushHour} \PY{o}{==} \PY{l+m+mi}{0}\PY{p}{,} \PY{l+s+s2}{\PYZdq{}}\PY{l+s+s2}{Distance}\PY{l+s+s2}{\PYZdq{}}\PY{p}{]}
              \PY{o}{.}\PY{n}{mean}\PY{p}{(}\PY{p}{)}\PY{p}{)}
\end{Verbatim}


    \begin{Verbatim}[commandchars=\\\{\}]
media in orario di punta    : 3.3193548387096796
media in orario di non punta: 13.487428571428563

    \end{Verbatim}

    In orario di punta si tende a percerrere distanze brevi, come avevamo
già notato. \#\#\# 3.4.12. Sempre in riferimento ai due gruppi di orario
(di punta/non di punta), calcolate la varianza within groups e la
varianza between groups

    \begin{Verbatim}[commandchars=\\\{\}]
{\color{incolor}In [{\color{incolor}11}]:} \PY{n}{group1} \PY{o}{=} \PY{n}{data}\PY{o}{.}\PY{n}{loc}\PY{p}{[}\PY{n}{data}\PY{o}{.}\PY{n}{RushHour} \PY{o}{==} \PY{l+m+mi}{0}\PY{p}{,} \PY{l+s+s2}{\PYZdq{}}\PY{l+s+s2}{Distance}\PY{l+s+s2}{\PYZdq{}}\PY{p}{]}
         \PY{n}{group2} \PY{o}{=} \PY{n}{data}\PY{o}{.}\PY{n}{loc}\PY{p}{[}\PY{n}{data}\PY{o}{.}\PY{n}{RushHour} \PY{o}{==} \PY{l+m+mi}{1}\PY{p}{,} \PY{l+s+s2}{\PYZdq{}}\PY{l+s+s2}{Distance}\PY{l+s+s2}{\PYZdq{}}\PY{p}{]}
         
         \PY{n}{size}        \PY{o}{=} \PY{n+nb}{len}\PY{p}{(}\PY{n}{data}\PY{o}{.}\PY{n}{Distance}\PY{p}{)}
         \PY{n}{size\PYZus{}group1} \PY{o}{=} \PY{n+nb}{len}\PY{p}{(}\PY{n}{group1}\PY{p}{)}
         \PY{n}{size\PYZus{}group2} \PY{o}{=} \PY{n+nb}{len}\PY{p}{(}\PY{n}{group2}\PY{p}{)}
         
         \PY{n}{mean}        \PY{o}{=} \PY{n}{data}\PY{o}{.}\PY{n}{Distance}\PY{o}{.}\PY{n}{mean}\PY{p}{(}\PY{p}{)}
         \PY{n}{mean\PYZus{}group1} \PY{o}{=} \PY{n}{group1}\PY{o}{.}\PY{n}{mean}\PY{p}{(}\PY{p}{)}
         \PY{n}{mean\PYZus{}group2} \PY{o}{=} \PY{n}{group2}\PY{o}{.}\PY{n}{mean}\PY{p}{(}\PY{p}{)}
         
         \PY{n}{var}        \PY{o}{=} \PY{n}{data}\PY{o}{.}\PY{n}{Distance}\PY{o}{.}\PY{n}{var}\PY{p}{(}\PY{p}{)}
         \PY{n}{var\PYZus{}group1} \PY{o}{=} \PY{n}{group1}\PY{o}{.}\PY{n}{var}\PY{p}{(}\PY{p}{)}
         \PY{n}{var\PYZus{}group2} \PY{o}{=} \PY{n}{group2}\PY{o}{.}\PY{n}{var}\PY{p}{(}\PY{p}{)}
         
         \PY{n}{variance\PYZus{}within\PYZus{}groups}  \PY{o}{=} \PY{p}{(}\PY{n}{size\PYZus{}group1}\PY{o}{/}\PY{n}{size}\PY{p}{)}\PY{o}{*}\PY{n}{var\PYZus{}group1} 
         \PY{o}{+} \PY{p}{(}\PY{n}{size\PYZus{}group2}\PY{o}{/}\PY{n}{size}\PY{p}{)}\PY{o}{*}\PY{n}{var\PYZus{}group2}
         \PY{n}{variance\PYZus{}between\PYZus{}groups} \PY{o}{=} \PY{p}{(}\PY{n}{size\PYZus{}group1}\PY{o}{/}\PY{n}{size}\PY{p}{)}\PY{o}{*}\PY{p}{(}\PY{n}{mean\PYZus{}group1}\PY{o}{\PYZhy{}}\PY{n}{mean}\PY{p}{)}\PY{o}{*}\PY{o}{*}\PY{l+m+mi}{2} 
         \PY{o}{+} \PY{p}{(}\PY{n}{size\PYZus{}group2}\PY{o}{/}\PY{n}{size}\PY{p}{)}\PY{o}{*}\PY{p}{(}\PY{n}{mean\PYZus{}group2}\PY{o}{\PYZhy{}}\PY{n}{mean}\PY{p}{)}\PY{o}{*}\PY{o}{*}\PY{l+m+mi}{2}
         
         \PY{n+nb}{print}\PY{p}{(}\PY{l+s+s2}{\PYZdq{}}\PY{l+s+s2}{variance within  groups:}\PY{l+s+s2}{\PYZdq{}}\PY{p}{,} \PY{n}{variance\PYZus{}within\PYZus{}groups}\PY{p}{)}
         \PY{n+nb}{print}\PY{p}{(}\PY{l+s+s2}{\PYZdq{}}\PY{l+s+s2}{variance between groups:}\PY{l+s+s2}{\PYZdq{}}\PY{p}{,} \PY{n}{variance\PYZus{}between\PYZus{}groups}\PY{p}{)}
\end{Verbatim}


    \begin{Verbatim}[commandchars=\\\{\}]
variance within  groups: 13.127032899366636
variance between groups: 14.144144642299338

    \end{Verbatim}

    \hypertarget{esercizio-4}{%
\section{Esercizio 4}\label{esercizio-4}}

\hypertarget{tracciate-un-grafico-rappresentativo-della-distribuzione-della-distanza-percorsa-negli-oraridi-punta}{%
\subsection{4.1. Tracciate un grafico rappresentativo della
distribuzione della distanza percorsa negli oraridi
punta}\label{tracciate-un-grafico-rappresentativo-della-distribuzione-della-distanza-percorsa-negli-oraridi-punta}}

    \begin{Verbatim}[commandchars=\\\{\}]
{\color{incolor}In [{\color{incolor}12}]:} \PY{n}{fig}\PY{p}{,} \PY{n}{ax} \PY{o}{=} \PY{n}{plt}\PY{o}{.}\PY{n}{subplots}\PY{p}{(}\PY{n}{nrows}\PY{o}{=}\PY{l+m+mi}{1}\PY{p}{,} \PY{n}{ncols}\PY{o}{=}\PY{l+m+mi}{2}\PY{p}{)}
         \PY{n}{fig}\PY{o}{.}\PY{n}{set\PYZus{}size\PYZus{}inches}\PY{p}{(}\PY{l+m+mi}{10}\PY{p}{,}\PY{l+m+mi}{5}\PY{p}{)}
         \PY{n}{plot0} \PY{o}{=} \PY{n}{data}\PY{o}{.}\PY{n}{loc}\PY{p}{[}\PY{n}{data}\PY{o}{.}\PY{n}{RushHour} \PY{o}{==} \PY{l+m+mi}{1}\PY{p}{,}\PY{p}{:}\PY{p}{]}\PY{o}{.}\PY{n}{plot}\PY{o}{.}\PY{n}{scatter}\PY{p}{(}\PY{n}{x}\PY{o}{=}\PY{l+s+s2}{\PYZdq{}}\PY{l+s+s2}{Distance}\PY{l+s+s2}{\PYZdq{}}\PY{p}{,}\PY{n}{y}\PY{o}{=}\PY{l+s+s2}{\PYZdq{}}\PY{l+s+s2}{Time}\PY{l+s+s2}{\PYZdq{}}\PY{p}{,}
                                                 \PY{n}{ax} \PY{o}{=} \PY{n}{ax}\PY{p}{[}\PY{l+m+mi}{0}\PY{p}{]}\PY{p}{,} \PY{n}{title}\PY{o}{=}\PY{l+s+s2}{\PYZdq{}}\PY{l+s+s2}{RushHour = 1}\PY{l+s+s2}{\PYZdq{}}\PY{p}{)}
         \PY{n}{plot1} \PY{o}{=} \PY{n}{data}\PY{o}{.}\PY{n}{loc}\PY{p}{[}\PY{n}{data}\PY{o}{.}\PY{n}{RushHour} \PY{o}{==} \PY{l+m+mi}{0}\PY{p}{,}\PY{p}{:}\PY{p}{]}\PY{o}{.}\PY{n}{plot}\PY{o}{.}\PY{n}{scatter}\PY{p}{(}\PY{n}{x}\PY{o}{=}\PY{l+s+s2}{\PYZdq{}}\PY{l+s+s2}{Distance}\PY{l+s+s2}{\PYZdq{}}\PY{p}{,}\PY{n}{y}\PY{o}{=}\PY{l+s+s2}{\PYZdq{}}\PY{l+s+s2}{Time}\PY{l+s+s2}{\PYZdq{}}\PY{p}{,}
                                                 \PY{n}{ax} \PY{o}{=} \PY{n}{ax}\PY{p}{[}\PY{l+m+mi}{1}\PY{p}{]}\PY{p}{,} \PY{n}{title}\PY{o}{=}\PY{l+s+s2}{\PYZdq{}}\PY{l+s+s2}{RushHour = 0}\PY{l+s+s2}{\PYZdq{}}\PY{p}{)}
\end{Verbatim}


    \begin{center}
    \adjustimage{max size={0.9\linewidth}{0.9\paperheight}}{output_37_0.png}
    \end{center}
    { \hspace*{\fill} \\}
    
    In ore normali la crescità sembra lineare.\\
Durante la RushHour la crescità è maggiore. \#\# 4.2. È plausibile
affermare che negli orari di punta la distanza segue una legge normale?
Giustificate la risposta.

    \begin{Verbatim}[commandchars=\\\{\}]
{\color{incolor}In [{\color{incolor}13}]:} \PY{k+kn}{import} \PY{n+nn}{scipy}\PY{n+nn}{.}\PY{n+nn}{stats} \PY{k}{as} \PY{n+nn}{stats}
         \PY{n}{plot} \PY{o}{=} \PY{n}{stats}\PY{o}{.}\PY{n}{probplot}\PY{p}{(}\PY{n}{data}\PY{o}{.}\PY{n}{loc}\PY{p}{[}\PY{n}{data}\PY{o}{.}\PY{n}{RushHour} \PY{o}{==} \PY{l+m+mi}{1}\PY{p}{,}\PY{l+s+s2}{\PYZdq{}}\PY{l+s+s2}{Distance}\PY{l+s+s2}{\PYZdq{}}\PY{p}{]}\PY{p}{,} \PY{n}{dist}\PY{o}{=}\PY{l+s+s2}{\PYZdq{}}\PY{l+s+s2}{norm}\PY{l+s+s2}{\PYZdq{}}\PY{p}{,}
                               \PY{n}{plot}\PY{o}{=}\PY{n}{plt}\PY{p}{)}
\end{Verbatim}


    \begin{center}
    \adjustimage{max size={0.9\linewidth}{0.9\paperheight}}{output_39_0.png}
    \end{center}
    { \hspace*{\fill} \\}
    
    La distribuzione delle distanze sembra più essere esponenziale. \#\#
4.3. Calcolate la media e la deviazione standard della distanza negli
orari di punta.

    \begin{Verbatim}[commandchars=\\\{\}]
{\color{incolor}In [{\color{incolor}20}]:} \PY{n+nb}{print}\PY{p}{(}\PY{l+s+s2}{\PYZdq{}}\PY{l+s+s2}{mean   :}\PY{l+s+s2}{\PYZdq{}}\PY{p}{,}\PY{n}{data}\PY{o}{.}\PY{n}{loc}\PY{p}{[}\PY{n}{data}\PY{o}{.}\PY{n}{RushHour} \PY{o}{==} \PY{l+m+mi}{1}\PY{p}{,}\PY{l+s+s2}{\PYZdq{}}\PY{l+s+s2}{Distance}\PY{l+s+s2}{\PYZdq{}}\PY{p}{]}\PY{o}{.}\PY{n}{mean}\PY{p}{(}\PY{p}{)}\PY{p}{)}
         \PY{n+nb}{print}\PY{p}{(}\PY{l+s+s2}{\PYZdq{}}\PY{l+s+s2}{std dev:}\PY{l+s+s2}{\PYZdq{}}\PY{p}{,}\PY{n}{data}\PY{o}{.}\PY{n}{loc}\PY{p}{[}\PY{n}{data}\PY{o}{.}\PY{n}{RushHour} \PY{o}{==} \PY{l+m+mi}{1}\PY{p}{,}\PY{l+s+s2}{\PYZdq{}}\PY{l+s+s2}{Distance}\PY{l+s+s2}{\PYZdq{}}\PY{p}{]}\PY{o}{.}\PY{n}{var}\PY{p}{(}\PY{p}{)}\PY{o}{*}\PY{o}{*}\PY{l+m+mf}{0.5}\PY{p}{)}
\end{Verbatim}


    \begin{Verbatim}[commandchars=\\\{\}]
mean   : 3.319354838709678
std dev: 3.711106147915897

    \end{Verbatim}

    \hypertarget{esercizio-5}{%
\section{ESERCIZIO 5}\label{esercizio-5}}

Concentriamoci ora sulla distanza percorsa dai veicoli negli orari non
di punta. \#\# 5.1. Tracciate un grafico opportuno che descriva la
distanza percorsa negli orari non di punta.

    \begin{Verbatim}[commandchars=\\\{\}]
{\color{incolor}In [{\color{incolor}14}]:} \PY{n}{plt0} \PY{o}{=} \PY{n}{pandas}\PY{o}{.}\PY{n}{DataFrame}\PY{o}{.}\PY{n}{hist}\PY{p}{(}\PY{n}{data}\PY{o}{.}\PY{n}{loc}\PY{p}{[}\PY{n}{data}\PY{o}{.}\PY{n}{RushHour} \PY{o}{==} \PY{l+m+mi}{0}\PY{p}{,}\PY{p}{:}\PY{p}{]}\PY{p}{,} 
                                      \PY{n}{column}\PY{o}{=}\PY{l+s+s2}{\PYZdq{}}\PY{l+s+s2}{Distance}\PY{l+s+s2}{\PYZdq{}}\PY{p}{,}\PY{n}{bins}\PY{o}{=}\PY{l+m+mi}{25}\PY{p}{)}
\end{Verbatim}


    \begin{center}
    \adjustimage{max size={0.9\linewidth}{0.9\paperheight}}{output_43_0.png}
    \end{center}
    { \hspace*{\fill} \\}
    
    \hypertarget{uxe8-plausibile-affermare-che-negli-orari-non-di-punta-la-distanza-segue-una-legge-normale-giustificate-la-risposta.}{%
\subsection{5.2. È plausibile affermare che negli orari non di punta la
distanza segue una legge normale? Giustificate la
risposta.}\label{uxe8-plausibile-affermare-che-negli-orari-non-di-punta-la-distanza-segue-una-legge-normale-giustificate-la-risposta.}}

    \begin{Verbatim}[commandchars=\\\{\}]
{\color{incolor}In [{\color{incolor}15}]:} \PY{n}{plot} \PY{o}{=} \PY{n}{stats}\PY{o}{.}\PY{n}{probplot}\PY{p}{(}\PY{n}{data}\PY{o}{.}\PY{n}{loc}\PY{p}{[}\PY{n}{data}\PY{o}{.}\PY{n}{RushHour} \PY{o}{==} \PY{l+m+mi}{0}\PY{p}{,}\PY{l+s+s2}{\PYZdq{}}\PY{l+s+s2}{Distance}\PY{l+s+s2}{\PYZdq{}}\PY{p}{]}\PY{p}{,}
                               \PY{n}{dist}\PY{o}{=}\PY{l+s+s2}{\PYZdq{}}\PY{l+s+s2}{norm}\PY{l+s+s2}{\PYZdq{}}\PY{p}{,} \PY{n}{plot}\PY{o}{=}\PY{n}{plt}\PY{p}{)}
\end{Verbatim}


    \begin{center}
    \adjustimage{max size={0.9\linewidth}{0.9\paperheight}}{output_45_0.png}
    \end{center}
    { \hspace*{\fill} \\}
    
    ci si avvicina, ma non si direbbe. \# Esercizio 6 Selezionate in una
variabile chiamata \texttt{tragitti\_brevi} tutti i casi in cui il
veicolo è stato utilizzato per percorrere un tragitto breve, cioè di
lunghezza inferiore a 1.5 km. \#\# 6.1. Tracciate il grafico di
dispersione della distanza e del tempo per i tragitti brevi.

    \begin{Verbatim}[commandchars=\\\{\}]
{\color{incolor}In [{\color{incolor}17}]:} \PY{n}{tragitti\PYZus{}brevi} \PY{o}{=} \PY{n}{data}\PY{o}{.}\PY{n}{loc}\PY{p}{[}\PY{n}{data}\PY{o}{.}\PY{n}{Distance} \PY{o}{\PYZlt{}}\PY{o}{=} \PY{l+m+mf}{1.5}\PY{p}{,}\PY{p}{:}\PY{p}{]}
         \PY{n}{plot} \PY{o}{=} \PY{n}{tragitti\PYZus{}brevi}\PY{o}{.}\PY{n}{plot}\PY{o}{.}\PY{n}{scatter}\PY{p}{(}\PY{n}{x}\PY{o}{=}\PY{l+s+s2}{\PYZdq{}}\PY{l+s+s2}{Distance}\PY{l+s+s2}{\PYZdq{}}\PY{p}{,}\PY{n}{y}\PY{o}{=}\PY{l+s+s2}{\PYZdq{}}\PY{l+s+s2}{Time}\PY{l+s+s2}{\PYZdq{}}\PY{p}{)}
\end{Verbatim}


    \begin{center}
    \adjustimage{max size={0.9\linewidth}{0.9\paperheight}}{output_47_0.png}
    \end{center}
    { \hspace*{\fill} \\}
    
    \hypertarget{commentate-il-grafico-che-avete-tracciato-al-punto-precedente-possibilmente-collegandolo-al-valore-assunto-dallindice-di-variazione-per-il-carattere-time.}{%
\subsection{6.2. Commentate il grafico che avete tracciato al punto
precedente, possibilmente collegandolo al valore assunto dall'indice di
variazione per il carattere
Time.}\label{commentate-il-grafico-che-avete-tracciato-al-punto-precedente-possibilmente-collegandolo-al-valore-assunto-dallindice-di-variazione-per-il-carattere-time.}}

    \begin{Verbatim}[commandchars=\\\{\}]
{\color{incolor}In [{\color{incolor}18}]:} \PY{n}{var}       \PY{o}{=} \PY{n}{tragitti\PYZus{}brevi}\PY{o}{.}\PY{n}{Time}\PY{o}{.}\PY{n}{var}\PY{p}{(}\PY{p}{)}
         \PY{n}{mean\PYZus{}Time} \PY{o}{=} \PY{n}{tragitti\PYZus{}brevi}\PY{o}{.}\PY{n}{Time}\PY{o}{.}\PY{n}{mean}\PY{p}{(}\PY{p}{)}
         \PY{n}{sd\PYZus{}Time}   \PY{o}{=} \PY{n}{tragitti\PYZus{}brevi}\PY{o}{.}\PY{n}{Time}\PY{o}{.}\PY{n}{var}\PY{p}{(}\PY{p}{)}\PY{o}{*}\PY{o}{*}\PY{l+m+mf}{0.5}
         \PY{n}{coeff\PYZus{}var} \PY{o}{=} \PY{n}{sd\PYZus{}Time}\PY{o}{/}\PY{n+nb}{abs}\PY{p}{(}\PY{n}{mean\PYZus{}Time}\PY{p}{)}
         \PY{n+nb}{print}\PY{p}{(}\PY{l+s+s2}{\PYZdq{}}\PY{l+s+s2}{varianza                 :}\PY{l+s+s2}{\PYZdq{}}\PY{p}{,}\PY{n}{var}\PY{p}{)}
         \PY{n+nb}{print}\PY{p}{(}\PY{l+s+s2}{\PYZdq{}}\PY{l+s+s2}{standard deviation       :}\PY{l+s+s2}{\PYZdq{}}\PY{p}{,}\PY{n}{sd\PYZus{}Time}\PY{p}{)}
         \PY{n+nb}{print}\PY{p}{(}\PY{l+s+s2}{\PYZdq{}}\PY{l+s+s2}{coefficiente variazione  :}\PY{l+s+s2}{\PYZdq{}}\PY{p}{,}\PY{n}{coeff\PYZus{}var}\PY{p}{)}
         \PY{n+nb}{print}\PY{p}{(}\PY{l+s+s2}{\PYZdq{}}\PY{l+s+s2}{coefficiente correlazione:}\PY{l+s+se}{\PYZbs{}n}\PY{l+s+s2}{\PYZdq{}}\PY{p}{,}
               \PY{n}{tragitti\PYZus{}brevi}\PY{o}{.}\PY{n}{loc}\PY{p}{[}\PY{p}{:}\PY{p}{,}\PY{p}{[}\PY{l+s+s2}{\PYZdq{}}\PY{l+s+s2}{Time}\PY{l+s+s2}{\PYZdq{}}\PY{p}{,}\PY{l+s+s2}{\PYZdq{}}\PY{l+s+s2}{Distance}\PY{l+s+s2}{\PYZdq{}}\PY{p}{]}\PY{p}{]}\PY{o}{.}\PY{n}{corr}\PY{p}{(}\PY{p}{)}\PY{p}{)}
\end{Verbatim}


    \begin{Verbatim}[commandchars=\\\{\}]
varianza                 : 0.23958342099726518
standard deviation       : 0.4894725947356656
coefficiente variazione  : 0.1223057477921858
coefficiente correlazione:
              Time  Distance
Time      1.00000   0.14576
Distance  0.14576   1.00000

    \end{Verbatim}

    Il tempo per tragitti brevi è praticamente costante.\\
Questo lo si riscontra anche nel coefficiente di variazione che assume
un valore particolarmente basso.\\
Lo stesso è confermato dall'indice di correlazione tra \texttt{Distance}
e \texttt{Time}


    % Add a bibliography block to the postdoc
    
    
    
    \end{document}
