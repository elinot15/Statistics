
% Default to the notebook output style

    


% Inherit from the specified cell style.




    
\documentclass[11pt]{article}

    
    
    \usepackage[T1]{fontenc}
    % Nicer default font (+ math font) than Computer Modern for most use cases
    \usepackage{mathpazo}

    % Basic figure setup, for now with no caption control since it's done
    % automatically by Pandoc (which extracts ![](path) syntax from Markdown).
    \usepackage{graphicx}
    % We will generate all images so they have a width \maxwidth. This means
    % that they will get their normal width if they fit onto the page, but
    % are scaled down if they would overflow the margins.
    \makeatletter
    \def\maxwidth{\ifdim\Gin@nat@width>\linewidth\linewidth
    \else\Gin@nat@width\fi}
    \makeatother
    \let\Oldincludegraphics\includegraphics
    % Set max figure width to be 80% of text width, for now hardcoded.
    \renewcommand{\includegraphics}[1]{\Oldincludegraphics[width=.8\maxwidth]{#1}}
    % Ensure that by default, figures have no caption (until we provide a
    % proper Figure object with a Caption API and a way to capture that
    % in the conversion process - todo).
    \usepackage{caption}
    \DeclareCaptionLabelFormat{nolabel}{}
    \captionsetup{labelformat=nolabel}

    \usepackage{adjustbox} % Used to constrain images to a maximum size 
    \usepackage{xcolor} % Allow colors to be defined
    \usepackage{enumerate} % Needed for markdown enumerations to work
    \usepackage{geometry} % Used to adjust the document margins
    \usepackage{amsmath} % Equations
    \usepackage{amssymb} % Equations
    \usepackage{textcomp} % defines textquotesingle
    % Hack from http://tex.stackexchange.com/a/47451/13684:
    \AtBeginDocument{%
        \def\PYZsq{\textquotesingle}% Upright quotes in Pygmentized code
    }
    \usepackage{upquote} % Upright quotes for verbatim code
    \usepackage{eurosym} % defines \euro
    \usepackage[mathletters]{ucs} % Extended unicode (utf-8) support
    \usepackage[utf8x]{inputenc} % Allow utf-8 characters in the tex document
    \usepackage{fancyvrb} % verbatim replacement that allows latex
    \usepackage{grffile} % extends the file name processing of package graphics 
                         % to support a larger range 
    % The hyperref package gives us a pdf with properly built
    % internal navigation ('pdf bookmarks' for the table of contents,
    % internal cross-reference links, web links for URLs, etc.)
    \usepackage{hyperref}
    \usepackage{longtable} % longtable support required by pandoc >1.10
    \usepackage{booktabs}  % table support for pandoc > 1.12.2
    \usepackage[inline]{enumitem} % IRkernel/repr support (it uses the enumerate* environment)
    \usepackage[normalem]{ulem} % ulem is needed to support strikethroughs (\sout)
                                % normalem makes italics be italics, not underlines
    

    
    
    % Colors for the hyperref package
    \definecolor{urlcolor}{rgb}{0,.145,.698}
    \definecolor{linkcolor}{rgb}{.71,0.21,0.01}
    \definecolor{citecolor}{rgb}{.12,.54,.11}

    % ANSI colors
    \definecolor{ansi-black}{HTML}{3E424D}
    \definecolor{ansi-black-intense}{HTML}{282C36}
    \definecolor{ansi-red}{HTML}{E75C58}
    \definecolor{ansi-red-intense}{HTML}{B22B31}
    \definecolor{ansi-green}{HTML}{00A250}
    \definecolor{ansi-green-intense}{HTML}{007427}
    \definecolor{ansi-yellow}{HTML}{DDB62B}
    \definecolor{ansi-yellow-intense}{HTML}{B27D12}
    \definecolor{ansi-blue}{HTML}{208FFB}
    \definecolor{ansi-blue-intense}{HTML}{0065CA}
    \definecolor{ansi-magenta}{HTML}{D160C4}
    \definecolor{ansi-magenta-intense}{HTML}{A03196}
    \definecolor{ansi-cyan}{HTML}{60C6C8}
    \definecolor{ansi-cyan-intense}{HTML}{258F8F}
    \definecolor{ansi-white}{HTML}{C5C1B4}
    \definecolor{ansi-white-intense}{HTML}{A1A6B2}

    % commands and environments needed by pandoc snippets
    % extracted from the output of `pandoc -s`
    \providecommand{\tightlist}{%
      \setlength{\itemsep}{0pt}\setlength{\parskip}{0pt}}
    \DefineVerbatimEnvironment{Highlighting}{Verbatim}{commandchars=\\\{\}}
    % Add ',fontsize=\small' for more characters per line
    \newenvironment{Shaded}{}{}
    \newcommand{\KeywordTok}[1]{\textcolor[rgb]{0.00,0.44,0.13}{\textbf{{#1}}}}
    \newcommand{\DataTypeTok}[1]{\textcolor[rgb]{0.56,0.13,0.00}{{#1}}}
    \newcommand{\DecValTok}[1]{\textcolor[rgb]{0.25,0.63,0.44}{{#1}}}
    \newcommand{\BaseNTok}[1]{\textcolor[rgb]{0.25,0.63,0.44}{{#1}}}
    \newcommand{\FloatTok}[1]{\textcolor[rgb]{0.25,0.63,0.44}{{#1}}}
    \newcommand{\CharTok}[1]{\textcolor[rgb]{0.25,0.44,0.63}{{#1}}}
    \newcommand{\StringTok}[1]{\textcolor[rgb]{0.25,0.44,0.63}{{#1}}}
    \newcommand{\CommentTok}[1]{\textcolor[rgb]{0.38,0.63,0.69}{\textit{{#1}}}}
    \newcommand{\OtherTok}[1]{\textcolor[rgb]{0.00,0.44,0.13}{{#1}}}
    \newcommand{\AlertTok}[1]{\textcolor[rgb]{1.00,0.00,0.00}{\textbf{{#1}}}}
    \newcommand{\FunctionTok}[1]{\textcolor[rgb]{0.02,0.16,0.49}{{#1}}}
    \newcommand{\RegionMarkerTok}[1]{{#1}}
    \newcommand{\ErrorTok}[1]{\textcolor[rgb]{1.00,0.00,0.00}{\textbf{{#1}}}}
    \newcommand{\NormalTok}[1]{{#1}}
    
    % Additional commands for more recent versions of Pandoc
    \newcommand{\ConstantTok}[1]{\textcolor[rgb]{0.53,0.00,0.00}{{#1}}}
    \newcommand{\SpecialCharTok}[1]{\textcolor[rgb]{0.25,0.44,0.63}{{#1}}}
    \newcommand{\VerbatimStringTok}[1]{\textcolor[rgb]{0.25,0.44,0.63}{{#1}}}
    \newcommand{\SpecialStringTok}[1]{\textcolor[rgb]{0.73,0.40,0.53}{{#1}}}
    \newcommand{\ImportTok}[1]{{#1}}
    \newcommand{\DocumentationTok}[1]{\textcolor[rgb]{0.73,0.13,0.13}{\textit{{#1}}}}
    \newcommand{\AnnotationTok}[1]{\textcolor[rgb]{0.38,0.63,0.69}{\textbf{\textit{{#1}}}}}
    \newcommand{\CommentVarTok}[1]{\textcolor[rgb]{0.38,0.63,0.69}{\textbf{\textit{{#1}}}}}
    \newcommand{\VariableTok}[1]{\textcolor[rgb]{0.10,0.09,0.49}{{#1}}}
    \newcommand{\ControlFlowTok}[1]{\textcolor[rgb]{0.00,0.44,0.13}{\textbf{{#1}}}}
    \newcommand{\OperatorTok}[1]{\textcolor[rgb]{0.40,0.40,0.40}{{#1}}}
    \newcommand{\BuiltInTok}[1]{{#1}}
    \newcommand{\ExtensionTok}[1]{{#1}}
    \newcommand{\PreprocessorTok}[1]{\textcolor[rgb]{0.74,0.48,0.00}{{#1}}}
    \newcommand{\AttributeTok}[1]{\textcolor[rgb]{0.49,0.56,0.16}{{#1}}}
    \newcommand{\InformationTok}[1]{\textcolor[rgb]{0.38,0.63,0.69}{\textbf{\textit{{#1}}}}}
    \newcommand{\WarningTok}[1]{\textcolor[rgb]{0.38,0.63,0.69}{\textbf{\textit{{#1}}}}}
    
    
    % Define a nice break command that doesn't care if a line doesn't already
    % exist.
    \def\br{\hspace*{\fill} \\* }
    % Math Jax compatability definitions
    \def\gt{>}
    \def\lt{<}
    % Document parameters
    \title{solution\_python}
    
    
    

    % Pygments definitions
    
\makeatletter
\def\PY@reset{\let\PY@it=\relax \let\PY@bf=\relax%
    \let\PY@ul=\relax \let\PY@tc=\relax%
    \let\PY@bc=\relax \let\PY@ff=\relax}
\def\PY@tok#1{\csname PY@tok@#1\endcsname}
\def\PY@toks#1+{\ifx\relax#1\empty\else%
    \PY@tok{#1}\expandafter\PY@toks\fi}
\def\PY@do#1{\PY@bc{\PY@tc{\PY@ul{%
    \PY@it{\PY@bf{\PY@ff{#1}}}}}}}
\def\PY#1#2{\PY@reset\PY@toks#1+\relax+\PY@do{#2}}

\expandafter\def\csname PY@tok@w\endcsname{\def\PY@tc##1{\textcolor[rgb]{0.73,0.73,0.73}{##1}}}
\expandafter\def\csname PY@tok@c\endcsname{\let\PY@it=\textit\def\PY@tc##1{\textcolor[rgb]{0.25,0.50,0.50}{##1}}}
\expandafter\def\csname PY@tok@cp\endcsname{\def\PY@tc##1{\textcolor[rgb]{0.74,0.48,0.00}{##1}}}
\expandafter\def\csname PY@tok@k\endcsname{\let\PY@bf=\textbf\def\PY@tc##1{\textcolor[rgb]{0.00,0.50,0.00}{##1}}}
\expandafter\def\csname PY@tok@kp\endcsname{\def\PY@tc##1{\textcolor[rgb]{0.00,0.50,0.00}{##1}}}
\expandafter\def\csname PY@tok@kt\endcsname{\def\PY@tc##1{\textcolor[rgb]{0.69,0.00,0.25}{##1}}}
\expandafter\def\csname PY@tok@o\endcsname{\def\PY@tc##1{\textcolor[rgb]{0.40,0.40,0.40}{##1}}}
\expandafter\def\csname PY@tok@ow\endcsname{\let\PY@bf=\textbf\def\PY@tc##1{\textcolor[rgb]{0.67,0.13,1.00}{##1}}}
\expandafter\def\csname PY@tok@nb\endcsname{\def\PY@tc##1{\textcolor[rgb]{0.00,0.50,0.00}{##1}}}
\expandafter\def\csname PY@tok@nf\endcsname{\def\PY@tc##1{\textcolor[rgb]{0.00,0.00,1.00}{##1}}}
\expandafter\def\csname PY@tok@nc\endcsname{\let\PY@bf=\textbf\def\PY@tc##1{\textcolor[rgb]{0.00,0.00,1.00}{##1}}}
\expandafter\def\csname PY@tok@nn\endcsname{\let\PY@bf=\textbf\def\PY@tc##1{\textcolor[rgb]{0.00,0.00,1.00}{##1}}}
\expandafter\def\csname PY@tok@ne\endcsname{\let\PY@bf=\textbf\def\PY@tc##1{\textcolor[rgb]{0.82,0.25,0.23}{##1}}}
\expandafter\def\csname PY@tok@nv\endcsname{\def\PY@tc##1{\textcolor[rgb]{0.10,0.09,0.49}{##1}}}
\expandafter\def\csname PY@tok@no\endcsname{\def\PY@tc##1{\textcolor[rgb]{0.53,0.00,0.00}{##1}}}
\expandafter\def\csname PY@tok@nl\endcsname{\def\PY@tc##1{\textcolor[rgb]{0.63,0.63,0.00}{##1}}}
\expandafter\def\csname PY@tok@ni\endcsname{\let\PY@bf=\textbf\def\PY@tc##1{\textcolor[rgb]{0.60,0.60,0.60}{##1}}}
\expandafter\def\csname PY@tok@na\endcsname{\def\PY@tc##1{\textcolor[rgb]{0.49,0.56,0.16}{##1}}}
\expandafter\def\csname PY@tok@nt\endcsname{\let\PY@bf=\textbf\def\PY@tc##1{\textcolor[rgb]{0.00,0.50,0.00}{##1}}}
\expandafter\def\csname PY@tok@nd\endcsname{\def\PY@tc##1{\textcolor[rgb]{0.67,0.13,1.00}{##1}}}
\expandafter\def\csname PY@tok@s\endcsname{\def\PY@tc##1{\textcolor[rgb]{0.73,0.13,0.13}{##1}}}
\expandafter\def\csname PY@tok@sd\endcsname{\let\PY@it=\textit\def\PY@tc##1{\textcolor[rgb]{0.73,0.13,0.13}{##1}}}
\expandafter\def\csname PY@tok@si\endcsname{\let\PY@bf=\textbf\def\PY@tc##1{\textcolor[rgb]{0.73,0.40,0.53}{##1}}}
\expandafter\def\csname PY@tok@se\endcsname{\let\PY@bf=\textbf\def\PY@tc##1{\textcolor[rgb]{0.73,0.40,0.13}{##1}}}
\expandafter\def\csname PY@tok@sr\endcsname{\def\PY@tc##1{\textcolor[rgb]{0.73,0.40,0.53}{##1}}}
\expandafter\def\csname PY@tok@ss\endcsname{\def\PY@tc##1{\textcolor[rgb]{0.10,0.09,0.49}{##1}}}
\expandafter\def\csname PY@tok@sx\endcsname{\def\PY@tc##1{\textcolor[rgb]{0.00,0.50,0.00}{##1}}}
\expandafter\def\csname PY@tok@m\endcsname{\def\PY@tc##1{\textcolor[rgb]{0.40,0.40,0.40}{##1}}}
\expandafter\def\csname PY@tok@gh\endcsname{\let\PY@bf=\textbf\def\PY@tc##1{\textcolor[rgb]{0.00,0.00,0.50}{##1}}}
\expandafter\def\csname PY@tok@gu\endcsname{\let\PY@bf=\textbf\def\PY@tc##1{\textcolor[rgb]{0.50,0.00,0.50}{##1}}}
\expandafter\def\csname PY@tok@gd\endcsname{\def\PY@tc##1{\textcolor[rgb]{0.63,0.00,0.00}{##1}}}
\expandafter\def\csname PY@tok@gi\endcsname{\def\PY@tc##1{\textcolor[rgb]{0.00,0.63,0.00}{##1}}}
\expandafter\def\csname PY@tok@gr\endcsname{\def\PY@tc##1{\textcolor[rgb]{1.00,0.00,0.00}{##1}}}
\expandafter\def\csname PY@tok@ge\endcsname{\let\PY@it=\textit}
\expandafter\def\csname PY@tok@gs\endcsname{\let\PY@bf=\textbf}
\expandafter\def\csname PY@tok@gp\endcsname{\let\PY@bf=\textbf\def\PY@tc##1{\textcolor[rgb]{0.00,0.00,0.50}{##1}}}
\expandafter\def\csname PY@tok@go\endcsname{\def\PY@tc##1{\textcolor[rgb]{0.53,0.53,0.53}{##1}}}
\expandafter\def\csname PY@tok@gt\endcsname{\def\PY@tc##1{\textcolor[rgb]{0.00,0.27,0.87}{##1}}}
\expandafter\def\csname PY@tok@err\endcsname{\def\PY@bc##1{\setlength{\fboxsep}{0pt}\fcolorbox[rgb]{1.00,0.00,0.00}{1,1,1}{\strut ##1}}}
\expandafter\def\csname PY@tok@kc\endcsname{\let\PY@bf=\textbf\def\PY@tc##1{\textcolor[rgb]{0.00,0.50,0.00}{##1}}}
\expandafter\def\csname PY@tok@kd\endcsname{\let\PY@bf=\textbf\def\PY@tc##1{\textcolor[rgb]{0.00,0.50,0.00}{##1}}}
\expandafter\def\csname PY@tok@kn\endcsname{\let\PY@bf=\textbf\def\PY@tc##1{\textcolor[rgb]{0.00,0.50,0.00}{##1}}}
\expandafter\def\csname PY@tok@kr\endcsname{\let\PY@bf=\textbf\def\PY@tc##1{\textcolor[rgb]{0.00,0.50,0.00}{##1}}}
\expandafter\def\csname PY@tok@bp\endcsname{\def\PY@tc##1{\textcolor[rgb]{0.00,0.50,0.00}{##1}}}
\expandafter\def\csname PY@tok@fm\endcsname{\def\PY@tc##1{\textcolor[rgb]{0.00,0.00,1.00}{##1}}}
\expandafter\def\csname PY@tok@vc\endcsname{\def\PY@tc##1{\textcolor[rgb]{0.10,0.09,0.49}{##1}}}
\expandafter\def\csname PY@tok@vg\endcsname{\def\PY@tc##1{\textcolor[rgb]{0.10,0.09,0.49}{##1}}}
\expandafter\def\csname PY@tok@vi\endcsname{\def\PY@tc##1{\textcolor[rgb]{0.10,0.09,0.49}{##1}}}
\expandafter\def\csname PY@tok@vm\endcsname{\def\PY@tc##1{\textcolor[rgb]{0.10,0.09,0.49}{##1}}}
\expandafter\def\csname PY@tok@sa\endcsname{\def\PY@tc##1{\textcolor[rgb]{0.73,0.13,0.13}{##1}}}
\expandafter\def\csname PY@tok@sb\endcsname{\def\PY@tc##1{\textcolor[rgb]{0.73,0.13,0.13}{##1}}}
\expandafter\def\csname PY@tok@sc\endcsname{\def\PY@tc##1{\textcolor[rgb]{0.73,0.13,0.13}{##1}}}
\expandafter\def\csname PY@tok@dl\endcsname{\def\PY@tc##1{\textcolor[rgb]{0.73,0.13,0.13}{##1}}}
\expandafter\def\csname PY@tok@s2\endcsname{\def\PY@tc##1{\textcolor[rgb]{0.73,0.13,0.13}{##1}}}
\expandafter\def\csname PY@tok@sh\endcsname{\def\PY@tc##1{\textcolor[rgb]{0.73,0.13,0.13}{##1}}}
\expandafter\def\csname PY@tok@s1\endcsname{\def\PY@tc##1{\textcolor[rgb]{0.73,0.13,0.13}{##1}}}
\expandafter\def\csname PY@tok@mb\endcsname{\def\PY@tc##1{\textcolor[rgb]{0.40,0.40,0.40}{##1}}}
\expandafter\def\csname PY@tok@mf\endcsname{\def\PY@tc##1{\textcolor[rgb]{0.40,0.40,0.40}{##1}}}
\expandafter\def\csname PY@tok@mh\endcsname{\def\PY@tc##1{\textcolor[rgb]{0.40,0.40,0.40}{##1}}}
\expandafter\def\csname PY@tok@mi\endcsname{\def\PY@tc##1{\textcolor[rgb]{0.40,0.40,0.40}{##1}}}
\expandafter\def\csname PY@tok@il\endcsname{\def\PY@tc##1{\textcolor[rgb]{0.40,0.40,0.40}{##1}}}
\expandafter\def\csname PY@tok@mo\endcsname{\def\PY@tc##1{\textcolor[rgb]{0.40,0.40,0.40}{##1}}}
\expandafter\def\csname PY@tok@ch\endcsname{\let\PY@it=\textit\def\PY@tc##1{\textcolor[rgb]{0.25,0.50,0.50}{##1}}}
\expandafter\def\csname PY@tok@cm\endcsname{\let\PY@it=\textit\def\PY@tc##1{\textcolor[rgb]{0.25,0.50,0.50}{##1}}}
\expandafter\def\csname PY@tok@cpf\endcsname{\let\PY@it=\textit\def\PY@tc##1{\textcolor[rgb]{0.25,0.50,0.50}{##1}}}
\expandafter\def\csname PY@tok@c1\endcsname{\let\PY@it=\textit\def\PY@tc##1{\textcolor[rgb]{0.25,0.50,0.50}{##1}}}
\expandafter\def\csname PY@tok@cs\endcsname{\let\PY@it=\textit\def\PY@tc##1{\textcolor[rgb]{0.25,0.50,0.50}{##1}}}

\def\PYZbs{\char`\\}
\def\PYZus{\char`\_}
\def\PYZob{\char`\{}
\def\PYZcb{\char`\}}
\def\PYZca{\char`\^}
\def\PYZam{\char`\&}
\def\PYZlt{\char`\<}
\def\PYZgt{\char`\>}
\def\PYZsh{\char`\#}
\def\PYZpc{\char`\%}
\def\PYZdl{\char`\$}
\def\PYZhy{\char`\-}
\def\PYZsq{\char`\'}
\def\PYZdq{\char`\"}
\def\PYZti{\char`\~}
% for compatibility with earlier versions
\def\PYZat{@}
\def\PYZlb{[}
\def\PYZrb{]}
\makeatother


    % Exact colors from NB
    \definecolor{incolor}{rgb}{0.0, 0.0, 0.5}
    \definecolor{outcolor}{rgb}{0.545, 0.0, 0.0}



    
    % Prevent overflowing lines due to hard-to-break entities
    \sloppy 
    % Setup hyperref package
    \hypersetup{
      breaklinks=true,  % so long urls are correctly broken across lines
      colorlinks=true,
      urlcolor=urlcolor,
      linkcolor=linkcolor,
      citecolor=citecolor,
      }
    % Slightly bigger margins than the latex defaults
    
    \geometry{verbose,tmargin=1in,bmargin=1in,lmargin=1in,rmargin=1in}
    
    

    \begin{document}
    
    
    \maketitle
    
    

    
    \hypertarget{esercizio-a}{%
\subsection{Esercizio A}\label{esercizio-a}}

\begin{itemize}
\tightlist
\item
  A.1. Quanti sono i cani seguiti dall'ambulatorio?\\
\item
  A.2. Quanti cani soffrono di ipertensione?
\end{itemize}

    \begin{Verbatim}[commandchars=\\\{\}]
{\color{incolor}In [{\color{incolor}21}]:} \PY{k+kn}{import} \PY{n+nn}{pandas}
         \PY{n}{data} \PY{o}{=} \PY{n}{pandas}\PY{o}{.}\PY{n}{read\PYZus{}csv}\PY{p}{(}\PY{l+s+s2}{\PYZdq{}}\PY{l+s+s2}{cani.csv}\PY{l+s+s2}{\PYZdq{}}\PY{p}{,}\PY{n}{sep}\PY{o}{=}\PY{l+s+s2}{\PYZdq{}}\PY{l+s+s2}{;}\PY{l+s+s2}{\PYZdq{}}\PY{p}{,}\PY{n}{decimal}\PY{o}{=}\PY{l+s+s2}{\PYZdq{}}\PY{l+s+s2}{,}\PY{l+s+s2}{\PYZdq{}}\PY{p}{)}
         \PY{n+nb}{print}\PY{p}{(}\PY{l+s+s2}{\PYZdq{}}\PY{l+s+s2}{A.1.: }\PY{l+s+s2}{\PYZdq{}}\PY{p}{,} \PY{n+nb}{len}\PY{p}{(}\PY{n}{data}\PY{p}{)}\PY{p}{)}
         \PY{n+nb}{print}\PY{p}{(}\PY{l+s+s2}{\PYZdq{}}\PY{l+s+s2}{A.2.: }\PY{l+s+s2}{\PYZdq{}}\PY{p}{,} \PY{n+nb}{len}\PY{p}{(}\PY{n}{data}\PY{o}{.}\PY{n}{loc}\PY{p}{[}\PY{n}{data}\PY{o}{.}\PY{n}{IP}\PY{o}{==}\PY{l+s+s2}{\PYZdq{}}\PY{l+s+s2}{SI}\PY{l+s+s2}{\PYZdq{}}\PY{p}{,}\PY{p}{:}\PY{p}{]}\PY{p}{)}\PY{p}{)}
         \PY{n}{data}\PY{o}{.}\PY{n}{iloc}\PY{p}{[}\PY{l+m+mi}{1}\PY{p}{:}\PY{l+m+mi}{5}\PY{p}{,}\PY{p}{]}
\end{Verbatim}


    \begin{Verbatim}[commandchars=\\\{\}]
A.1.:  161
A.2.:  58

    \end{Verbatim}

\begin{Verbatim}[commandchars=\\\{\}]
{\color{outcolor}Out[{\color{outcolor}21}]:}   Cartella  IP  GravitaIP  EtaAnni  MORTE   MC  SURVIVALTIME  Terapia  \textbackslash{}
         1    C0621  NO          0    15.21      1  1.0           341        3   
         2    B0918  NO          0    15.77      1  1.0           117        3   
         3    R1009  NO          0    13.54      1  1.0            93        4   
         4    R1513  NO          0    10.72      0  NaN           666        3   
         
           Antiaritmico  PesoKg  VTricuspide  AsxAo  OndaE  OndaEA  FrazEspuls  \textbackslash{}
         1           NO     6.0          0.0   1.61   1.50    1.47        83.0   
         2           NO     3.2          0.0   2.50   1.87    2.08        91.0   
         3           SI    26.5          0.0   3.07   2.28    2.85        75.0   
         4           NO    11.0          0.0   2.54   0.75    0.86        69.0   
         
            FrazAccorc    EDVI   ESVI  Allodiast  Allosist  
         1        51.0  130.58  23.88       1.94      0.99  
         2        62.0  180.35  16.22       2.21      0.84  
         3        44.0  234.17  40.15       2.09      1.16  
         4        38.0  139.84  43.89       1.98      1.23  
\end{Verbatim}
            
    A.3. Consideriamo ora l'età dei pazienti.\\
- A.3.1. Tracciare un istogramma dell'età dei cani con i seguenti
accorgimenti:\\
- fissando a un anno l'ampiezza delle classi e - considerando gli
intervalli chiusi a sinistra e aperti a destra.

    \begin{Verbatim}[commandchars=\\\{\}]
{\color{incolor}In [{\color{incolor}3}]:} \PY{k+kn}{import} \PY{n+nn}{math}
        \PY{k+kn}{import} \PY{n+nn}{matplotlib}\PY{n+nn}{.}\PY{n+nn}{pyplot} \PY{k}{as} \PY{n+nn}{plt}
        \PY{n}{age\PYZus{}min} \PY{o}{=} \PY{n+nb}{min}\PY{p}{(}\PY{n}{data}\PY{o}{.}\PY{n}{EtaAnni}\PY{o}{.}\PY{n}{apply}\PY{p}{(}\PY{n}{math}\PY{o}{.}\PY{n}{floor}\PY{p}{)}\PY{p}{)}
        \PY{n}{age\PYZus{}max} \PY{o}{=} \PY{n+nb}{max}\PY{p}{(}\PY{n}{data}\PY{o}{.}\PY{n}{EtaAnni}\PY{o}{.}\PY{n}{apply}\PY{p}{(}\PY{n}{math}\PY{o}{.}\PY{n}{floor}\PY{p}{)}\PY{p}{)}\PY{o}{+}\PY{l+m+mi}{1}
        \PY{n}{plt0} \PY{o}{=} \PY{n}{data}\PY{o}{.}\PY{n}{EtaAnni}\PY{o}{.}\PY{n}{hist}\PY{p}{(}\PY{n}{bins}\PY{o}{=}\PY{n+nb}{range}\PY{p}{(}\PY{n}{age\PYZus{}min}\PY{p}{,}\PY{n}{age\PYZus{}max}\PY{p}{)}\PY{p}{,} 
                                 \PY{n}{align}\PY{o}{=}\PY{l+s+s2}{\PYZdq{}}\PY{l+s+s2}{left}\PY{l+s+s2}{\PYZdq{}}\PY{p}{,} \PY{n}{grid}\PY{o}{=}\PY{k+kc}{True}\PY{p}{,}\PY{p}{)}
        \PY{n}{plt0} \PY{o}{=} \PY{n}{plt0}\PY{o}{.}\PY{n}{set\PYZus{}xticks}\PY{p}{(}\PY{n+nb}{list}\PY{p}{(}\PY{n+nb}{range}\PY{p}{(}\PY{l+m+mi}{0}\PY{p}{,}\PY{l+m+mi}{16}\PY{p}{)}\PY{p}{)}\PY{p}{)}
\end{Verbatim}


    \begin{center}
    \adjustimage{max size={0.9\linewidth}{0.9\paperheight}}{output_3_0.png}
    \end{center}
    { \hspace*{\fill} \\}
    
    l'istogramma del'età presenta una forte assimetria a destra.\\
Questo significa che l'ambulatorio registra più frequentemente cani di
età elevata,\\
il che è estremamente ragionevole dato che questi sono sicuramente più
soggetti a malattie.

A.3.2. Descrivere l'età dei pazienti compilando la Tabella 1, in cui
scegliere un opportuno indice di centralità e un opportuno indice di
dispersione

    \begin{Verbatim}[commandchars=\\\{\}]
{\color{incolor}In [{\color{incolor}4}]:} \PY{n+nb}{print}\PY{p}{(}\PY{l+s+s2}{\PYZdq{}}\PY{l+s+s2}{min    :}\PY{l+s+s2}{\PYZdq{}}\PY{p}{,} \PY{n}{data}\PY{o}{.}\PY{n}{EtaAnni}\PY{o}{.}\PY{n}{min}\PY{p}{(}\PY{p}{)}\PY{p}{)}
        \PY{n+nb}{print}\PY{p}{(}\PY{l+s+s2}{\PYZdq{}}\PY{l+s+s2}{median :}\PY{l+s+s2}{\PYZdq{}}\PY{p}{,} \PY{n}{data}\PY{o}{.}\PY{n}{EtaAnni}\PY{o}{.}\PY{n}{median}\PY{p}{(}\PY{p}{)}\PY{p}{)}
        \PY{n+nb}{print}\PY{p}{(}\PY{l+s+s2}{\PYZdq{}}\PY{l+s+s2}{IQR    :}\PY{l+s+s2}{\PYZdq{}}\PY{p}{,} \PY{n}{data}\PY{o}{.}\PY{n}{EtaAnni}\PY{o}{.}\PY{n}{quantile}\PY{p}{(}\PY{l+m+mf}{0.75}\PY{p}{)}
              \PY{o}{\PYZhy{}}\PY{n}{data}\PY{o}{.}\PY{n}{EtaAnni}\PY{o}{.}\PY{n}{quantile}\PY{p}{(}\PY{l+m+mf}{0.25}\PY{p}{)}\PY{p}{)}
        \PY{n+nb}{print}\PY{p}{(}\PY{l+s+s2}{\PYZdq{}}\PY{l+s+s2}{max    :}\PY{l+s+s2}{\PYZdq{}}\PY{p}{,} \PY{n}{data}\PY{o}{.}\PY{n}{EtaAnni}\PY{o}{.}\PY{n}{max}\PY{p}{(}\PY{p}{)}\PY{p}{)}
\end{Verbatim}


    \begin{Verbatim}[commandchars=\\\{\}]
min    : 1.22
median : 12.55
IQR    : 3.17
max    : 16.84

    \end{Verbatim}

    A.3.3. Quanti sono i pazienti di età compresa nell'intervallo tra i 12 e
i 13 anni, estremo inferiore incluso ed estremo superiore escluso ?\\
A.3.4. Quanti anni ha il cane più anziano?\\
A.3.5. Qual è la fascia di età maggiormente rappresentata? Si risponda
con un intervallo chiuso a sinistra e aperto a destra.

    \begin{Verbatim}[commandchars=\\\{\}]
{\color{incolor}In [{\color{incolor}5}]:} \PY{n+nb}{print}\PY{p}{(}\PY{l+s+s2}{\PYZdq{}}\PY{l+s+s2}{A.3.3:}\PY{l+s+s2}{\PYZdq{}}\PY{p}{,}\PY{n}{data}\PY{o}{.}\PY{n}{loc}\PY{p}{[}\PY{p}{(}\PY{n}{data}\PY{o}{.}\PY{n}{EtaAnni} \PY{o}{\PYZlt{}} \PY{l+m+mi}{13}\PY{p}{)} \PY{o}{\PYZam{}} \PY{p}{(}\PY{n}{data}\PY{o}{.}\PY{n}{EtaAnni} \PY{o}{\PYZgt{}}\PY{o}{=} \PY{l+m+mi}{12}\PY{p}{)}\PY{p}{,}
                      \PY{p}{:}\PY{p}{]}\PY{o}{.}\PY{n+nf+fm}{\PYZus{}\PYZus{}len\PYZus{}\PYZus{}}\PY{p}{(}\PY{p}{)}\PY{p}{)}
        \PY{n+nb}{print}\PY{p}{(}\PY{l+s+s2}{\PYZdq{}}\PY{l+s+s2}{A.3.4:}\PY{l+s+s2}{\PYZdq{}}\PY{p}{,}\PY{n}{data}\PY{o}{.}\PY{n}{EtaAnni}\PY{o}{.}\PY{n}{max}\PY{p}{(}\PY{p}{)}\PY{p}{)}
        \PY{n+nb}{print}\PY{p}{(}\PY{l+s+s2}{\PYZdq{}}\PY{l+s+s2}{A.3.5:}\PY{l+s+s2}{\PYZdq{}}\PY{p}{,}\PY{l+s+s2}{\PYZdq{}}\PY{l+s+s2}{[12,13)}\PY{l+s+s2}{\PYZdq{}}\PY{p}{)}
\end{Verbatim}


    \begin{Verbatim}[commandchars=\\\{\}]
A.3.3: 32
A.3.4: 16.84
A.3.5: [12,13)

    \end{Verbatim}

    A.4. Consideriamo le variabili MORTE e MC.\\
A.4.1. Quanti cani sono deceduti?

    \begin{Verbatim}[commandchars=\\\{\}]
{\color{incolor}In [{\color{incolor}73}]:} \PY{n+nb}{print}\PY{p}{(}\PY{n+nb}{len}\PY{p}{(}\PY{n}{data}\PY{o}{.}\PY{n}{loc}\PY{p}{[}\PY{n}{data}\PY{o}{.}\PY{n}{MORTE} \PY{o}{==} \PY{l+m+mi}{1}\PY{p}{,}\PY{p}{:}\PY{p}{]}\PY{p}{)}\PY{p}{)}
         \PY{n+nb}{print}\PY{p}{(}\PY{n}{data}\PY{o}{.}\PY{n}{loc}\PY{p}{[}\PY{n}{data}\PY{o}{.}\PY{n}{MORTE} \PY{o}{==} \PY{l+m+mi}{1}\PY{p}{,}\PY{p}{:}\PY{p}{]}\PY{o}{.}\PY{n+nf+fm}{\PYZus{}\PYZus{}len\PYZus{}\PYZus{}}\PY{p}{(}\PY{p}{)}\PY{p}{)}
\end{Verbatim}


    \begin{Verbatim}[commandchars=\\\{\}]
118
118

    \end{Verbatim}

    4.2. Nell'inserire le informazioni riguardo a un cane deceduto,
l'operatore ha sempre specificato se la morte è avvenuta per cause
cardiache o per altre cause? Se larisposta è ``no'', in quanti casi
(sempre relativamente ai cani deceduti) l'operatore ha omesso tale
informazione?\\
l'operatore non ha sempre specificato l'argomento di MC, infatti, per
alcune righe MC è lasciato come \texttt{NA}.

    \begin{Verbatim}[commandchars=\\\{\}]
{\color{incolor}In [{\color{incolor}6}]:} \PY{n}{data}\PY{o}{.}\PY{n}{loc}\PY{p}{[}\PY{n}{pandas}\PY{o}{.}\PY{n}{isna}\PY{p}{(}\PY{n}{data}\PY{o}{.}\PY{n}{MC}\PY{p}{)}\PY{p}{,} \PY{p}{]}\PY{o}{.}\PY{n+nf+fm}{\PYZus{}\PYZus{}len\PYZus{}\PYZus{}}\PY{p}{(}\PY{p}{)}
\end{Verbatim}


\begin{Verbatim}[commandchars=\\\{\}]
{\color{outcolor}Out[{\color{outcolor}6}]:} 46
\end{Verbatim}
            
    A.4.3. Controllare che non ci siano nei dati incongruenze riguardo alla
morte, ovvero che non ci siano casi per i quali il cane risulta vivo ma
morto di morte cardiaca.

    \begin{Verbatim}[commandchars=\\\{\}]
{\color{incolor}In [{\color{incolor}6}]:} \PY{n}{data}\PY{o}{.}\PY{n}{loc}\PY{p}{[}\PY{p}{(}\PY{o}{\PYZti{}} \PY{n}{pandas}\PY{o}{.}\PY{n}{isna}\PY{p}{(}\PY{n}{data}\PY{o}{.}\PY{n}{MC}\PY{p}{)}\PY{p}{)} \PY{o}{\PYZam{}} \PY{p}{(}\PY{n}{data}\PY{o}{.}\PY{n}{MC} \PY{o}{==} \PY{l+m+mi}{1}\PY{p}{)} \PY{o}{\PYZam{}} 
                 \PY{p}{(}\PY{n}{data}\PY{o}{.}\PY{n}{MORTE} \PY{o}{==} \PY{l+m+mi}{0}\PY{p}{)}\PY{p}{,}\PY{p}{:}\PY{p}{]}\PY{o}{.}\PY{n+nf+fm}{\PYZus{}\PYZus{}len\PYZus{}\PYZus{}}\PY{p}{(}\PY{p}{)}
\end{Verbatim}


\begin{Verbatim}[commandchars=\\\{\}]
{\color{outcolor}Out[{\color{outcolor}6}]:} 0
\end{Verbatim}
            
    non ci sono incongruenze.\\
A.4.5. Tra le morti avvenute, quale percentuale è stata per cause
cardiache?

    \begin{Verbatim}[commandchars=\\\{\}]
{\color{incolor}In [{\color{incolor}7}]:} \PY{n}{casi\PYZus{}favorevoli} \PY{o}{=} \PY{n}{data}\PY{o}{.}\PY{n}{loc}\PY{p}{[}\PY{p}{(}\PY{o}{\PYZti{}} \PY{n}{pandas}\PY{o}{.}\PY{n}{isna}\PY{p}{(}\PY{n}{data}\PY{o}{.}\PY{n}{MC}\PY{p}{)}\PY{p}{)} \PY{o}{\PYZam{}}
                                   \PY{p}{(}\PY{n}{data}\PY{o}{.}\PY{n}{MC} \PY{o}{==} \PY{l+m+mi}{1}\PY{p}{)}   \PY{p}{,}\PY{p}{:}\PY{p}{]}\PY{o}{.}\PY{n+nf+fm}{\PYZus{}\PYZus{}len\PYZus{}\PYZus{}}\PY{p}{(}\PY{p}{)}
        \PY{n}{casi\PYZus{}totali}     \PY{o}{=} \PY{n}{data}\PY{o}{.}\PY{n}{loc}\PY{p}{[}\PY{p}{(}\PY{o}{\PYZti{}} \PY{n}{pandas}\PY{o}{.}\PY{n}{isna}\PY{p}{(}\PY{n}{data}\PY{o}{.}\PY{n}{MC}\PY{p}{)}\PY{p}{)} \PY{o}{\PYZam{}}
                                   \PY{p}{(}\PY{n}{data}\PY{o}{.}\PY{n}{MORTE} \PY{o}{==} \PY{l+m+mi}{1}\PY{p}{)}\PY{p}{,}\PY{p}{:}\PY{p}{]}\PY{o}{.}\PY{n+nf+fm}{\PYZus{}\PYZus{}len\PYZus{}\PYZus{}}\PY{p}{(}\PY{p}{)}
        \PY{n+nb}{print}\PY{p}{(}\PY{n}{casi\PYZus{}favorevoli}\PY{o}{*}\PY{l+m+mi}{100}\PY{o}{/}\PY{n}{casi\PYZus{}totali}\PY{p}{,}\PY{l+s+s2}{\PYZdq{}}\PY{l+s+s2}{\PYZpc{}}\PY{l+s+s2}{\PYZdq{}}\PY{p}{,}\PY{n}{sep}\PY{o}{=}\PY{l+s+s2}{\PYZdq{}}\PY{l+s+s2}{\PYZdq{}}\PY{p}{)}
\end{Verbatim}


    \begin{Verbatim}[commandchars=\\\{\}]
75.65217391304348\%

    \end{Verbatim}

    A.5. La variabile GravitaIP è un indice di gravità dell'ipertensione.\\
A.5.1. Si tratta di un carattere scalare, ordinale oppure nominale?\\
A.5.2. Quali valori può assumere?

    \begin{Verbatim}[commandchars=\\\{\}]
{\color{incolor}In [{\color{incolor}9}]:} \PY{n}{data}\PY{o}{.}\PY{n}{GravitaIP}\PY{o}{.}\PY{n}{unique}\PY{p}{(}\PY{p}{)}
\end{Verbatim}


\begin{Verbatim}[commandchars=\\\{\}]
{\color{outcolor}Out[{\color{outcolor}9}]:} array([0, 1, 2, 3])
\end{Verbatim}
            
    A.5.1: è ordinale in quanto rappresenta un ordinamento sulla gravità.\\
A.5.2: può assumere valori 0,1,2,3

A.5.3. Produrre la tabella delle frequenze relative di GravitaIP.

    \begin{Verbatim}[commandchars=\\\{\}]
{\color{incolor}In [{\color{incolor}9}]:} \PY{n}{relative\PYZus{}freq} \PY{o}{=} \PY{n}{data}\PY{o}{.}\PY{n}{GravitaIP}\PY{o}{.}\PY{n}{value\PYZus{}counts}\PY{p}{(}\PY{p}{)}
        \PY{o}{/}\PY{n+nb}{sum}\PY{p}{(}\PY{n}{data}\PY{o}{.}\PY{n}{GravitaIP}\PY{o}{.}\PY{n}{value\PYZus{}counts}\PY{p}{(}\PY{p}{)}\PY{p}{)}
        \PY{n}{relative\PYZus{}freq}
\end{Verbatim}


\begin{Verbatim}[commandchars=\\\{\}]
{\color{outcolor}Out[{\color{outcolor}9}]:} 0    103
        1     29
        2     18
        3     11
        Name: GravitaIP, dtype: int64
\end{Verbatim}
            
    A.5.4. Tracciare un grafico opportuno per descrivere la gravità
dell'ipertensione.

    \begin{Verbatim}[commandchars=\\\{\}]
{\color{incolor}In [{\color{incolor}10}]:} \PY{n}{fig}\PY{p}{,} \PY{n}{ax} \PY{o}{=} \PY{n}{plt}\PY{o}{.}\PY{n}{subplots}\PY{p}{(}\PY{n}{nrows}\PY{o}{=}\PY{l+m+mi}{1}\PY{p}{,} \PY{n}{ncols}\PY{o}{=}\PY{l+m+mi}{2}\PY{p}{)}
         \PY{n}{fig}\PY{o}{.}\PY{n}{set\PYZus{}size\PYZus{}inches}\PY{p}{(}\PY{l+m+mi}{15}\PY{p}{,}\PY{l+m+mi}{5}\PY{p}{)}
         \PY{n}{relative\PYZus{}freq}\PY{o}{.}\PY{n}{index} \PY{o}{=} \PY{n+nb}{list}\PY{p}{(}\PY{n}{relative\PYZus{}freq}\PY{o}{.}\PY{n}{apply}\PY{p}{(}\PY{k}{lambda} \PY{n}{x}\PY{p}{:} 
                                                 \PY{n+nb}{str}\PY{p}{(}\PY{n}{x}\PY{p}{)}\PY{p}{[}\PY{p}{:}\PY{l+m+mi}{6}\PY{p}{]}\PY{p}{)}\PY{p}{)}
         \PY{n}{plt0} \PY{o}{=} \PY{n}{relative\PYZus{}freq}\PY{o}{.}\PY{n}{plot}\PY{o}{.}\PY{n}{pie}\PY{p}{(}\PY{n}{ax} \PY{o}{=} \PY{n}{ax}\PY{p}{[}\PY{l+m+mi}{0}\PY{p}{]}\PY{p}{)}
         \PY{n}{plt0} \PY{o}{=} \PY{n}{plt0}\PY{o}{.}\PY{n}{legend}\PY{p}{(}\PY{p}{[}\PY{l+s+s2}{\PYZdq{}}\PY{l+s+s2}{0}\PY{l+s+s2}{\PYZdq{}}\PY{p}{,}\PY{l+s+s2}{\PYZdq{}}\PY{l+s+s2}{1}\PY{l+s+s2}{\PYZdq{}}\PY{p}{,}\PY{l+s+s2}{\PYZdq{}}\PY{l+s+s2}{2}\PY{l+s+s2}{\PYZdq{}}\PY{p}{,}\PY{l+s+s2}{\PYZdq{}}\PY{l+s+s2}{3}\PY{l+s+s2}{\PYZdq{}}\PY{p}{]}\PY{p}{)}
         
         \PY{n}{relative\PYZus{}freq}\PY{o}{.}\PY{n}{index} \PY{o}{=} \PY{p}{[}\PY{l+s+s2}{\PYZdq{}}\PY{l+s+s2}{0}\PY{l+s+s2}{\PYZdq{}}\PY{p}{,}\PY{l+s+s2}{\PYZdq{}}\PY{l+s+s2}{1}\PY{l+s+s2}{\PYZdq{}}\PY{p}{,}\PY{l+s+s2}{\PYZdq{}}\PY{l+s+s2}{2}\PY{l+s+s2}{\PYZdq{}}\PY{p}{,}\PY{l+s+s2}{\PYZdq{}}\PY{l+s+s2}{3}\PY{l+s+s2}{\PYZdq{}}\PY{p}{]}
         \PY{n}{plt1} \PY{o}{=} \PY{n}{relative\PYZus{}freq}\PY{o}{.}\PY{n}{plot}\PY{o}{.}\PY{n}{bar}\PY{p}{(}\PY{n}{ax} \PY{o}{=} \PY{n}{ax}\PY{p}{[}\PY{l+m+mi}{1}\PY{p}{]}\PY{p}{)}
         \PY{n}{plt1} \PY{o}{=} \PY{n}{plt1}\PY{o}{.}\PY{n}{legend}\PY{p}{(}\PY{p}{)}
\end{Verbatim}


    \begin{center}
    \adjustimage{max size={0.9\linewidth}{0.9\paperheight}}{output_21_0.png}
    \end{center}
    { \hspace*{\fill} \\}
    
    Essendo solamente 4 i possibili valori di \texttt{GravitàIP}, I grafici
più adeguati sono quello a torta e quello a barre.

    A.6. Consideriamo l'assunzione di farmaci antiaritmici e la morte per
cause cardiache.\\
A.6.1. Produrre la tabella delle frequenze assolute del carattere
Antiaritmico.\\
A.6.2. Quanti sono i cani che assumono un farmaco antiaritmico?

    \begin{Verbatim}[commandchars=\\\{\}]
{\color{incolor}In [{\color{incolor}91}]:} \PY{n+nb}{print}\PY{p}{(}\PY{l+s+s2}{\PYZdq{}}\PY{l+s+s2}{A.6.1:}\PY{l+s+s2}{\PYZdq{}}\PY{p}{)}
         \PY{n+nb}{print}\PY{p}{(}\PY{n}{data}\PY{o}{.}\PY{n}{Antiaritmico}\PY{o}{.}\PY{n}{value\PYZus{}counts}\PY{p}{(}\PY{p}{)}\PY{p}{)}
         \PY{n+nb}{print}\PY{p}{(}\PY{l+s+s2}{\PYZdq{}}\PY{l+s+s2}{A.6.2:}\PY{l+s+s2}{\PYZdq{}}\PY{p}{)}
         \PY{n+nb}{print}\PY{p}{(}\PY{n}{data}\PY{o}{.}\PY{n}{loc}\PY{p}{[}\PY{n}{data}\PY{o}{.}\PY{n}{Antiaritmico} \PY{o}{==} \PY{l+s+s2}{\PYZdq{}}\PY{l+s+s2}{SI}\PY{l+s+s2}{\PYZdq{}}\PY{p}{,}\PY{p}{:}\PY{p}{]}\PY{o}{.}\PY{n+nf+fm}{\PYZus{}\PYZus{}len\PYZus{}\PYZus{}}\PY{p}{(}\PY{p}{)}\PY{p}{)}
\end{Verbatim}


    \begin{Verbatim}[commandchars=\\\{\}]
A.6.1:
NO    150
SI     11
Name: Antiaritmico, dtype: int64
A.6.2:
11

    \end{Verbatim}

    A.6.3. Il carattere Antiaritmico è categorico. Volendolo convertire in
un carattere numerico, con quale valore numerico mettereste in
corrispondenza valore ``SI''? Con quale il``NO''?\\
rispettivamente 1 e 0 sembra una buona scelta.

A.6.4. Produrre la tabella delle frequenze assolute congiunte dei
caratteri Antiaritmico e MC.

    \begin{Verbatim}[commandchars=\\\{\}]
{\color{incolor}In [{\color{incolor}13}]:} \PY{n}{frequenze\PYZus{}congiunte} \PY{o}{=} \PY{n}{pandas}\PY{o}{.}\PY{n}{crosstab}\PY{p}{(}\PY{n}{data}\PY{o}{.}\PY{n}{Antiaritmico}\PY{p}{,}\PY{n}{data}\PY{o}{.}\PY{n}{MC}\PY{p}{)} 
         \PY{n}{frequenze\PYZus{}congiunte}
\end{Verbatim}


\begin{Verbatim}[commandchars=\\\{\}]
{\color{outcolor}Out[{\color{outcolor}13}]:} MC            0.0  1.0
         Antiaritmico          
         NO             28   78
         SI              0    9
\end{Verbatim}
            
    A.6.5. Quale percentuale dei cani morti per cause cardiache assumeva un
farmaco antiaritmico?

    \begin{Verbatim}[commandchars=\\\{\}]
{\color{incolor}In [{\color{incolor}14}]:} \PY{n}{casi\PYZus{}totali}     \PY{o}{=} \PY{n+nb}{sum}\PY{p}{(}\PY{n}{frequenze\PYZus{}congiunte}\PY{o}{.}\PY{n}{loc}\PY{p}{[}\PY{p}{:}\PY{p}{,}\PY{l+m+mi}{1}\PY{p}{]}\PY{p}{)}
         \PY{n}{casi\PYZus{}favorevoli} \PY{o}{=} \PY{n}{frequenze\PYZus{}congiunte}\PY{o}{.}\PY{n}{loc}\PY{p}{[}\PY{l+s+s2}{\PYZdq{}}\PY{l+s+s2}{SI}\PY{l+s+s2}{\PYZdq{}}\PY{p}{,}\PY{l+m+mi}{1}\PY{p}{]}
         \PY{n+nb}{print}\PY{p}{(}\PY{n}{casi\PYZus{}favorevoli}\PY{o}{*}\PY{l+m+mi}{100}\PY{o}{/}\PY{n}{casi\PYZus{}totali}\PY{p}{,}\PY{l+s+s2}{\PYZdq{}}\PY{l+s+s2}{\PYZpc{}}\PY{l+s+s2}{\PYZdq{}}\PY{p}{)}
\end{Verbatim}


    \begin{Verbatim}[commandchars=\\\{\}]
10.344827586206897 \%

    \end{Verbatim}

    A.7. Il carattere SURVIVALTIME (tempo di sopravvivenza) ci dice per
quanti giorni il paziente è rimasto in vita a partire dalla prima visita
presso l'ambulatorio. Come mostratonei grafici di Figura 1, la
distribuzione delle frequenze del tempo di sopravvivenza ha un aspetto
molto diverso se si considera rispetto ai cani ancora in vita oppure a
quelli morti. Potete rispondere alle seguenti due domande semplicemente
ispezionando i graficidi Figura 1, considerando un anno costituito da
365 giorni.

A.7.1. Quale percentuale di cani tuttora vivi è in cura presso
l'ambulatorio da meno di un anno?

    \begin{Verbatim}[commandchars=\\\{\}]
{\color{incolor}In [{\color{incolor}11}]:} \PY{n}{casi\PYZus{}favorevoli} \PY{o}{=} \PY{n+nb}{len}\PY{p}{(}\PY{n}{data}\PY{o}{.}\PY{n}{loc}\PY{p}{[}\PY{p}{(}\PY{n}{data}\PY{o}{.}\PY{n}{MORTE}\PY{o}{==}\PY{l+m+mi}{0}\PY{p}{)} \PY{o}{\PYZam{}} 
                         \PY{p}{(}\PY{n}{data}\PY{o}{.}\PY{n}{SURVIVALTIME} \PY{o}{\PYZlt{}}\PY{o}{=} \PY{l+m+mi}{365}\PY{p}{)}\PY{p}{,}\PY{l+s+s2}{\PYZdq{}}\PY{l+s+s2}{SURVIVALTIME}\PY{l+s+s2}{\PYZdq{}}\PY{p}{]}\PY{p}{)}
         \PY{n}{casi\PYZus{}totali}     \PY{o}{=} \PY{n+nb}{len}\PY{p}{(}\PY{n}{data}\PY{o}{.}\PY{n}{loc}\PY{p}{[}\PY{n}{data}\PY{o}{.}\PY{n}{MORTE}\PY{o}{==}\PY{l+m+mi}{0}\PY{p}{,}\PY{l+s+s2}{\PYZdq{}}\PY{l+s+s2}{SURVIVALTIME}\PY{l+s+s2}{\PYZdq{}}\PY{p}{]}\PY{p}{)}
         \PY{n+nb}{print}\PY{p}{(}\PY{n}{casi\PYZus{}favorevoli}\PY{o}{*}\PY{l+m+mi}{100}\PY{o}{/}\PY{n}{casi\PYZus{}totali}\PY{p}{,}\PY{l+s+s2}{\PYZdq{}}\PY{l+s+s2}{\PYZpc{}}\PY{l+s+s2}{\PYZdq{}}\PY{p}{)}
\end{Verbatim}


    \begin{Verbatim}[commandchars=\\\{\}]
65.11627906976744 \%

    \end{Verbatim}

    A.7.2. Quale percentuale di cani deceduti è sopravvissuta più di 3 anni?

    \begin{Verbatim}[commandchars=\\\{\}]
{\color{incolor}In [{\color{incolor}12}]:} \PY{n}{casi\PYZus{}favorevoli} \PY{o}{=} \PY{n}{data}\PY{o}{.}\PY{n}{loc}\PY{p}{[}\PY{p}{(}\PY{n}{data}\PY{o}{.}\PY{n}{MORTE} \PY{o}{==} \PY{l+m+mi}{1}\PY{p}{)} \PY{o}{\PYZam{}} 
                      \PY{p}{(}\PY{n}{data}\PY{o}{.}\PY{n}{SURVIVALTIME} \PY{o}{\PYZgt{}} \PY{l+m+mi}{365}\PY{o}{*}\PY{l+m+mi}{3}\PY{p}{)}\PY{p}{,} \PY{p}{:}\PY{p}{]}\PY{o}{.}\PY{n+nf+fm}{\PYZus{}\PYZus{}len\PYZus{}\PYZus{}}\PY{p}{(}\PY{p}{)}
         \PY{n}{casi\PYZus{}totali}     \PY{o}{=} \PY{n}{data}\PY{o}{.}\PY{n}{loc}\PY{p}{[}\PY{n}{data}\PY{o}{.}\PY{n}{MORTE} \PY{o}{==} \PY{l+m+mi}{1}\PY{p}{,} \PY{p}{:}\PY{p}{]}\PY{o}{.}\PY{n+nf+fm}{\PYZus{}\PYZus{}len\PYZus{}\PYZus{}}\PY{p}{(}\PY{p}{)}
         \PY{n+nb}{print}\PY{p}{(}\PY{n}{casi\PYZus{}favorevoli}\PY{o}{*}\PY{l+m+mi}{100}\PY{o}{/}\PY{n}{casi\PYZus{}totali}\PY{p}{,}\PY{l+s+s2}{\PYZdq{}}\PY{l+s+s2}{\PYZpc{}}\PY{l+s+s2}{\PYZdq{}}\PY{p}{)}
\end{Verbatim}


    \begin{Verbatim}[commandchars=\\\{\}]
10.169491525423728 \%

    \end{Verbatim}

    A.7.3. Tracciare un grafico opportuno per descrivere il tempo di
sopravvivenza.

    \begin{Verbatim}[commandchars=\\\{\}]
{\color{incolor}In [{\color{incolor}13}]:} \PY{n}{fig}\PY{p}{,} \PY{n}{ax} \PY{o}{=} \PY{n}{plt}\PY{o}{.}\PY{n}{subplots}\PY{p}{(}\PY{n}{nrows}\PY{o}{=}\PY{l+m+mi}{1}\PY{p}{,} \PY{n}{ncols}\PY{o}{=}\PY{l+m+mi}{2}\PY{p}{)}
         \PY{n}{fig}\PY{o}{.}\PY{n}{set\PYZus{}size\PYZus{}inches}\PY{p}{(}\PY{l+m+mi}{15}\PY{p}{,}\PY{l+m+mi}{5}\PY{p}{)}
         \PY{n}{plt0} \PY{o}{=} \PY{n}{data}\PY{o}{.}\PY{n}{SURVIVALTIME}\PY{o}{.}\PY{n}{hist}\PY{p}{(}\PY{n}{bins}\PY{o}{=}\PY{n+nb}{range}\PY{p}{(}\PY{l+m+mi}{0}\PY{p}{,}\PY{l+m+mi}{3000}\PY{p}{)}\PY{p}{[}\PY{l+m+mi}{0}\PY{p}{:}\PY{l+m+mi}{3000}\PY{p}{:}\PY{l+m+mi}{365}\PY{p}{]}\PY{p}{,}
                                       \PY{n}{ax}\PY{o}{=}\PY{n}{ax}\PY{p}{[}\PY{l+m+mi}{0}\PY{p}{]}\PY{p}{)}
         \PY{n}{plt1} \PY{o}{=} \PY{n}{data}\PY{o}{.}\PY{n}{SURVIVALTIME}\PY{o}{.}\PY{n}{plot}\PY{o}{.}\PY{n}{box}\PY{p}{(}\PY{n}{ax}\PY{o}{=}\PY{n}{ax}\PY{p}{[}\PY{l+m+mi}{1}\PY{p}{]}\PY{p}{)}
\end{Verbatim}


    \begin{center}
    \adjustimage{max size={0.9\linewidth}{0.9\paperheight}}{output_34_0.png}
    \end{center}
    { \hspace*{\fill} \\}
    
    7.4. La Figura 2(a) mostra il boxplot del tempo di sopravvivenza dei
cani del dataset. Completare il grafico con il valore numerico degli
estremi dellascatola.

    \begin{Verbatim}[commandchars=\\\{\}]
{\color{incolor}In [{\color{incolor}18}]:} \PY{n+nb}{print}\PY{p}{(}\PY{n}{data}\PY{o}{.}\PY{n}{SURVIVALTIME}\PY{o}{.}\PY{n}{quantile}\PY{p}{(}\PY{l+m+mf}{0.25}\PY{p}{)}\PY{p}{)}
         \PY{n+nb}{print}\PY{p}{(}\PY{n}{data}\PY{o}{.}\PY{n}{SURVIVALTIME}\PY{o}{.}\PY{n}{quantile}\PY{p}{(}\PY{l+m+mf}{0.75}\PY{p}{)}\PY{p}{)}
\end{Verbatim}


    \begin{Verbatim}[commandchars=\\\{\}]
113.0
711.0

    \end{Verbatim}

    A.7.5. Quanti animali sono compresi all'interno della scatola(estremi
inclusi)?

    \begin{Verbatim}[commandchars=\\\{\}]
{\color{incolor}In [{\color{incolor}14}]:} \PY{n}{data}\PY{o}{.}\PY{n}{loc}\PY{p}{[}\PY{p}{(}\PY{o}{\PYZti{}}\PY{n}{data}\PY{o}{.}\PY{n}{SURVIVALTIME}\PY{o}{.}\PY{n}{isna}\PY{p}{(}\PY{p}{)}\PY{p}{)} \PY{o}{\PYZam{}} \PY{p}{(}\PY{n}{data}\PY{o}{.}\PY{n}{SURVIVALTIME} \PY{o}{\PYZlt{}}\PY{o}{=} \PY{l+m+mi}{711}\PY{p}{)} \PY{o}{\PYZam{}} 
                  \PY{p}{(}\PY{n}{data}\PY{o}{.}\PY{n}{SURVIVALTIME} \PY{o}{\PYZgt{}}\PY{o}{=} \PY{l+m+mi}{113}\PY{p}{)}\PY{p}{,}\PY{p}{]}\PY{o}{.}\PY{n+nf+fm}{\PYZus{}\PYZus{}len\PYZus{}\PYZus{}}\PY{p}{(}\PY{p}{)}
\end{Verbatim}


\begin{Verbatim}[commandchars=\\\{\}]
{\color{outcolor}Out[{\color{outcolor}14}]:} 81
\end{Verbatim}
            
    A.7.6. Tracciare un grafico opportuno, diverso dal boxplot, che descriva
bene il tempo disopravvivenza dei cani considerati.

    \begin{Verbatim}[commandchars=\\\{\}]
{\color{incolor}In [{\color{incolor}15}]:} \PY{n}{fig}\PY{p}{,} \PY{n}{ax} \PY{o}{=} \PY{n}{plt}\PY{o}{.}\PY{n}{subplots}\PY{p}{(}\PY{n}{nrows}\PY{o}{=}\PY{l+m+mi}{1}\PY{p}{,} \PY{n}{ncols}\PY{o}{=}\PY{l+m+mi}{2}\PY{p}{)}
         \PY{n}{fig}\PY{o}{.}\PY{n}{set\PYZus{}size\PYZus{}inches}\PY{p}{(}\PY{l+m+mi}{15}\PY{p}{,}\PY{l+m+mi}{5}\PY{p}{)}
         
         \PY{n}{plt0} \PY{o}{=} \PY{n}{data}\PY{o}{.}\PY{n}{SURVIVALTIME}\PY{o}{.}\PY{n}{hist}\PY{p}{(}\PY{n}{bins}\PY{o}{=}\PY{n+nb}{range}\PY{p}{(}\PY{l+m+mi}{0}\PY{p}{,}\PY{l+m+mi}{3000}\PY{p}{)}\PY{p}{[}\PY{l+m+mi}{0}\PY{p}{:}\PY{l+m+mi}{3000}\PY{p}{:}\PY{l+m+mi}{365}\PY{p}{]}\PY{p}{,} 
                                       \PY{n}{ax}\PY{o}{=}\PY{n}{ax}\PY{p}{[}\PY{l+m+mi}{0}\PY{p}{]}\PY{p}{,} \PY{n}{grid}\PY{o}{=}\PY{k+kc}{False}\PY{p}{)}
         
         \PY{c+c1}{\PYZsh{} più leggibile}
         \PY{n}{plt1} \PY{o}{=} \PY{n}{data}\PY{o}{.}\PY{n}{SURVIVALTIME}\PY{o}{.}\PY{n}{hist}\PY{p}{(}\PY{n}{bins}\PY{o}{=}\PY{n+nb}{range}\PY{p}{(}\PY{l+m+mi}{0}\PY{p}{,}\PY{l+m+mi}{3000}\PY{p}{)}\PY{p}{[}\PY{l+m+mi}{0}\PY{p}{:}\PY{l+m+mi}{3000}\PY{p}{:}\PY{l+m+mi}{365}\PY{p}{]}\PY{p}{,} 
                                       \PY{n}{ax}\PY{o}{=}\PY{n}{ax}\PY{p}{[}\PY{l+m+mi}{1}\PY{p}{]}\PY{p}{,} \PY{n}{grid}\PY{o}{=}\PY{k+kc}{False}\PY{p}{)}
         \PY{n}{plt1} \PY{o}{=} \PY{n}{plt1}\PY{o}{.}\PY{n}{set\PYZus{}xticklabels}\PY{p}{(}\PY{n+nb}{range}\PY{p}{(}\PY{l+m+mi}{0}\PY{p}{,}\PY{l+m+mi}{3000}\PY{p}{)}\PY{p}{[}\PY{l+m+mi}{0}\PY{p}{:}\PY{l+m+mi}{3000}\PY{p}{:}\PY{l+m+mi}{365}\PY{p}{]}\PY{p}{)}
\end{Verbatim}


    \begin{center}
    \adjustimage{max size={0.9\linewidth}{0.9\paperheight}}{output_40_0.png}
    \end{center}
    { \hspace*{\fill} \\}
    
    A.7.7. Suggerite un modello teorico a voi noto che possa spiegare
l'andamento aleatorio della variabile casuale X=``Tempo di sopravvivenza
dei cani che frequentano (ofrequenteranno) l'ambulatorio''. Giustificate
la risposta.\\
è decrescente, potrebbe essere una geometrica o una esponenziale

A.7.8. Calcolate il tempo di sopravvivenza medio.

    \begin{Verbatim}[commandchars=\\\{\}]
{\color{incolor}In [{\color{incolor}21}]:} \PY{n}{data}\PY{o}{.}\PY{n}{SURVIVALTIME}\PY{o}{.}\PY{n}{mean}\PY{p}{(}\PY{p}{)}
\end{Verbatim}


\begin{Verbatim}[commandchars=\\\{\}]
{\color{outcolor}Out[{\color{outcolor}21}]:} 459.888198757764
\end{Verbatim}
            
    A.7.9. Calcolate la deviazione standard del tempo di sopravvivenza.

    \begin{Verbatim}[commandchars=\\\{\}]
{\color{incolor}In [{\color{incolor}22}]:} \PY{n}{data}\PY{o}{.}\PY{n}{SURVIVALTIME}\PY{o}{.}\PY{n}{std}\PY{p}{(}\PY{p}{)}
\end{Verbatim}


\begin{Verbatim}[commandchars=\\\{\}]
{\color{outcolor}Out[{\color{outcolor}22}]:} 467.19670634793675
\end{Verbatim}
            
    A.8. Consideriamo il carattere Allodiast.\\
A.8.1. Controllare se esso può essere considerato normale.

    \begin{Verbatim}[commandchars=\\\{\}]
{\color{incolor}In [{\color{incolor}23}]:} \PY{k+kn}{import} \PY{n+nn}{scipy}\PY{n+nn}{.}\PY{n+nn}{stats} \PY{k}{as} \PY{n+nn}{stats}
         \PY{k+kn}{import} \PY{n+nn}{numpy} \PY{k}{as} \PY{n+nn}{np}
         \PY{n}{fig}\PY{p}{,} \PY{n}{ax} \PY{o}{=} \PY{n}{plt}\PY{o}{.}\PY{n}{subplots}\PY{p}{(}\PY{n}{nrows}\PY{o}{=}\PY{l+m+mi}{1}\PY{p}{,} \PY{n}{ncols}\PY{o}{=}\PY{l+m+mi}{2}\PY{p}{)}
         \PY{n}{fig}\PY{o}{.}\PY{n}{set\PYZus{}size\PYZus{}inches}\PY{p}{(}\PY{l+m+mi}{15}\PY{p}{,}\PY{l+m+mi}{5}\PY{p}{)}
         
         \PY{n}{data}\PY{o}{.}\PY{n}{Allodiast}\PY{o}{.}\PY{n}{hist}\PY{p}{(}\PY{n}{bins}\PY{o}{=}\PY{n}{np}\PY{o}{.}\PY{n}{arange}\PY{p}{(}\PY{l+m+mi}{1}\PY{p}{,}\PY{l+m+mi}{3}\PY{p}{,}\PY{l+m+mf}{0.1}\PY{p}{)}\PY{p}{,} \PY{n}{ax}\PY{o}{=}\PY{n}{ax}\PY{p}{[}\PY{l+m+mi}{0}\PY{p}{]}\PY{p}{)} 
         
         \PY{n+nb}{print}\PY{p}{(}\PY{n}{data}\PY{o}{.}\PY{n}{Allodiast}\PY{o}{.}\PY{n}{mean}\PY{p}{(}\PY{p}{)}\PY{p}{)}
         \PY{n+nb}{print}\PY{p}{(}\PY{n}{data}\PY{o}{.}\PY{n}{Allodiast}\PY{o}{.}\PY{n}{median}\PY{p}{(}\PY{p}{)}\PY{p}{)}
         
         \PY{n}{plot} \PY{o}{=} \PY{n}{stats}\PY{o}{.}\PY{n}{probplot}\PY{p}{(}\PY{n}{data}\PY{o}{.}\PY{n}{Allodiast}\PY{p}{,} \PY{n}{dist}\PY{o}{=}\PY{l+s+s2}{\PYZdq{}}\PY{l+s+s2}{norm}\PY{l+s+s2}{\PYZdq{}}\PY{p}{,} \PY{n}{plot}\PY{o}{=}\PY{n}{ax}\PY{p}{[}\PY{l+m+mi}{1}\PY{p}{]}\PY{p}{)}
\end{Verbatim}


    \begin{Verbatim}[commandchars=\\\{\}]
2.013354037267081
2.0

    \end{Verbatim}

    \begin{center}
    \adjustimage{max size={0.9\linewidth}{0.9\paperheight}}{output_46_1.png}
    \end{center}
    { \hspace*{\fill} \\}
    
    media e mediana sono simili -\textgreater{} favore alla normalità.\\
dall'istogramma si vede una certa simmetria.\\
dal qqplot c'è una buona aderenza alla normalità meno nelle code.

8.2. Controllare che nell'intervallo di semi ampiezza 2 deviazioni
standard e centrato sulla media risiede circa il 96\% delle osservazioni
per tale carattere.

    \begin{Verbatim}[commandchars=\\\{\}]
{\color{incolor}In [{\color{incolor}24}]:} \PY{k+kn}{from} \PY{n+nn}{scipy} \PY{k}{import} \PY{n}{stats}
         
         \PY{n}{media} \PY{o}{=} \PY{n}{data}\PY{o}{.}\PY{n}{Allodiast}\PY{o}{.}\PY{n}{mean}\PY{p}{(}\PY{n}{skipna}\PY{o}{=}\PY{k+kc}{True}\PY{p}{)}
         \PY{n}{devst} \PY{o}{=} \PY{n}{data}\PY{o}{.}\PY{n}{Allodiast}\PY{o}{.}\PY{n}{std}\PY{p}{(}\PY{n}{skipna}\PY{o}{=}\PY{k+kc}{True}\PY{p}{)}
         \PY{n}{mul}   \PY{o}{=}  \PY{l+m+mi}{2}
         
         \PY{n}{p1} \PY{o}{=} \PY{n}{stats}\PY{o}{.}\PY{n}{percentileofscore}\PY{p}{(}\PY{n}{data}\PY{o}{.}\PY{n}{Allodiast}\PY{p}{,} \PY{n}{media} \PY{o}{+} \PY{n}{mul}\PY{o}{*}\PY{n}{devst}\PY{p}{)}
         \PY{n}{p2} \PY{o}{=} \PY{n}{stats}\PY{o}{.}\PY{n}{percentileofscore}\PY{p}{(}\PY{n}{data}\PY{o}{.}\PY{n}{Allodiast}\PY{p}{,} \PY{n}{media} \PY{o}{\PYZhy{}} \PY{n}{mul}\PY{o}{*}\PY{n}{devst}\PY{p}{)}
         
         \PY{n+nb}{print}\PY{p}{(}\PY{n}{p1}\PY{o}{\PYZhy{}}\PY{n}{p2}\PY{p}{)}
\end{Verbatim}


    \begin{Verbatim}[commandchars=\\\{\}]
96.27329192546584

    \end{Verbatim}

    Per conclude \texttt{Allodiast} non è normale, ma si comporta come tale
vicino alla media (cioè fino a due deviazioni standard dalla media).

A.9. I caratteri EDVI e Allodiast sono indipendenti? Motivare la
risposta, anche con l'ausilio di un grafico.

    \begin{Verbatim}[commandchars=\\\{\}]
{\color{incolor}In [{\color{incolor}25}]:} \PY{n}{plt0} \PY{o}{=} \PY{n}{data}\PY{o}{.}\PY{n}{plot}\PY{o}{.}\PY{n}{scatter}\PY{p}{(}\PY{l+s+s2}{\PYZdq{}}\PY{l+s+s2}{EDVI}\PY{l+s+s2}{\PYZdq{}}\PY{p}{,}\PY{l+s+s2}{\PYZdq{}}\PY{l+s+s2}{Allodiast}\PY{l+s+s2}{\PYZdq{}}\PY{p}{)}
         \PY{n}{data}\PY{o}{.}\PY{n}{loc}\PY{p}{[}\PY{p}{:}\PY{p}{,} \PY{p}{[}\PY{l+s+s2}{\PYZdq{}}\PY{l+s+s2}{EDVI}\PY{l+s+s2}{\PYZdq{}}\PY{p}{,}\PY{l+s+s2}{\PYZdq{}}\PY{l+s+s2}{Allodiast}\PY{l+s+s2}{\PYZdq{}}\PY{p}{]}\PY{p}{]}\PY{o}{.}\PY{n}{corr}\PY{p}{(}\PY{p}{)}
\end{Verbatim}


\begin{Verbatim}[commandchars=\\\{\}]
{\color{outcolor}Out[{\color{outcolor}25}]:}                EDVI  Allodiast
         EDVI       1.000000   0.907304
         Allodiast  0.907304   1.000000
\end{Verbatim}
            
    \begin{center}
    \adjustimage{max size={0.9\linewidth}{0.9\paperheight}}{output_50_1.png}
    \end{center}
    { \hspace*{\fill} \\}
    
    c'è infetti una relazione lineare tra \texttt{EDVI} e \texttt{Allodiast}
.\\
Infatti, all'aumentare di uno aumenta anche l'altro.

\hypertarget{esercizio-b}{%
\subsection{Esercizio B}\label{esercizio-b}}

Create una variabile che contenga la parte di dataset relativa ai cani
morti e considerando soltanto i casi in cui sia il carattere MC, sia il
carattere OndaEA non siano mancanti. Nel presente esercizio le domande
si riferiranno esclusivamente a questo sottoinsieme di casi.

    \begin{Verbatim}[commandchars=\\\{\}]
{\color{incolor}In [{\color{incolor}18}]:} \PY{n}{dataB} \PY{o}{=} \PY{n}{data}\PY{o}{.}\PY{n}{loc}\PY{p}{[}\PY{p}{(}\PY{n}{data}\PY{o}{.}\PY{n}{MORTE}\PY{o}{==}\PY{l+m+mi}{1}\PY{p}{)} \PY{o}{\PYZam{}} \PY{p}{(}\PY{o}{\PYZti{}}\PY{n}{data}\PY{o}{.}\PY{n}{MC}\PY{o}{.}\PY{n}{isna}\PY{p}{(}\PY{p}{)} \PY{p}{)}\PY{o}{\PYZam{}} 
                          \PY{p}{(}\PY{o}{\PYZti{}}\PY{n}{data}\PY{o}{.}\PY{n}{OndaEA}\PY{o}{.}\PY{n}{isna}\PY{p}{(}\PY{p}{)}\PY{p}{)}\PY{p}{,}\PY{p}{:}\PY{p}{]}
\end{Verbatim}


    B.1. L'OndaEA è un carattere scalare oppure ordinale?\\
E' un carattere scalare

    \begin{Verbatim}[commandchars=\\\{\}]
{\color{incolor}In [{\color{incolor}27}]:} \PY{n}{plt0} \PY{o}{=} \PY{n}{dataB}\PY{o}{.}\PY{n}{OndaEA}\PY{o}{.}\PY{n}{plot}\PY{o}{.}\PY{n}{box}\PY{p}{(}\PY{n}{title} \PY{o}{=} \PY{l+s+s2}{\PYZdq{}}\PY{l+s+s2}{Onda EA}\PY{l+s+s2}{\PYZdq{}}\PY{p}{)}
\end{Verbatim}


    \begin{center}
    \adjustimage{max size={0.9\linewidth}{0.9\paperheight}}{output_54_0.png}
    \end{center}
    { \hspace*{\fill} \\}
    
    B.3. Il grafico ottenuto dovrebbe mostrare la presenza di un outlier.
Determinare il valore diOndaEAper tale individuo.

    \begin{Verbatim}[commandchars=\\\{\}]
{\color{incolor}In [{\color{incolor}28}]:} \PY{n}{dataB}\PY{o}{.}\PY{n}{loc}\PY{p}{[}\PY{n}{dataB}\PY{o}{.}\PY{n}{OndaEA} \PY{o}{==} \PY{n}{dataB}\PY{o}{.}\PY{n}{OndaEA}\PY{o}{.}\PY{n}{max}\PY{p}{(}\PY{p}{)}\PY{p}{,}\PY{p}{:}\PY{p}{]}\PY{o}{.}\PY{n}{OndaEA}\PY{o}{.}\PY{n}{to\PYZus{}list}\PY{p}{(}\PY{p}{)}\PY{p}{[}\PY{l+m+mi}{0}\PY{p}{]}
\end{Verbatim}


\begin{Verbatim}[commandchars=\\\{\}]
{\color{outcolor}Out[{\color{outcolor}28}]:} 4.19
\end{Verbatim}
            
    B.4. L'outlier individuato è un cane morto per cause cardiache oppure
no?

    \begin{Verbatim}[commandchars=\\\{\}]
{\color{incolor}In [{\color{incolor}29}]:} \PY{n}{dataB}\PY{o}{.}\PY{n}{loc}\PY{p}{[}\PY{n}{dataB}\PY{o}{.}\PY{n}{OndaEA} \PY{o}{==} \PY{n+nb}{max}\PY{p}{(}\PY{n}{dataB}\PY{o}{.}\PY{n}{OndaEA}\PY{p}{)}\PY{p}{,}\PY{p}{:}\PY{p}{]}\PY{o}{.}\PY{n}{MC}\PY{o}{.}\PY{n}{to\PYZus{}list}\PY{p}{(}\PY{p}{)}\PY{p}{[}\PY{l+m+mi}{0}\PY{p}{]}
\end{Verbatim}


\begin{Verbatim}[commandchars=\\\{\}]
{\color{outcolor}Out[{\color{outcolor}29}]:} 1.0
\end{Verbatim}
            
    B.5. Si controlli che il terzo quartile, chiamiamolo s, dell'OndaEA
relativamente ai cani deceduti per cause non cardiache è 1.41.

    \begin{Verbatim}[commandchars=\\\{\}]
{\color{incolor}In [{\color{incolor}19}]:} \PY{n}{s} \PY{o}{=} \PY{n}{dataB}\PY{o}{.}\PY{n}{loc}\PY{p}{[}\PY{n}{dataB}\PY{o}{.}\PY{n}{MC}\PY{o}{==}\PY{l+m+mi}{0}\PY{p}{,}\PY{l+s+s2}{\PYZdq{}}\PY{l+s+s2}{OndaEA}\PY{l+s+s2}{\PYZdq{}}\PY{p}{]}\PY{o}{.}\PY{n}{dropna}\PY{p}{(}\PY{p}{)}\PY{o}{.}\PY{n}{quantile}\PY{p}{(}\PY{l+m+mf}{0.75}\PY{p}{)}
\end{Verbatim}


    B.6. Quanti sono i cani deceduti per cause cardiache? Quanti per altre
cause?

    \begin{Verbatim}[commandchars=\\\{\}]
{\color{incolor}In [{\color{incolor}31}]:} \PY{n+nb}{print}\PY{p}{(}\PY{n}{dataB}\PY{o}{.}\PY{n}{loc}\PY{p}{[}\PY{n}{dataB}\PY{o}{.}\PY{n}{MC}\PY{o}{==}\PY{l+m+mi}{1}\PY{p}{,}\PY{p}{:}\PY{p}{]}\PY{o}{.}\PY{n+nf+fm}{\PYZus{}\PYZus{}len\PYZus{}\PYZus{}}\PY{p}{(}\PY{p}{)}\PY{p}{)}
         \PY{n+nb}{print}\PY{p}{(}\PY{n}{dataB}\PY{o}{.}\PY{n}{loc}\PY{p}{[}\PY{n}{dataB}\PY{o}{.}\PY{n}{MC}\PY{o}{==}\PY{l+m+mi}{0}\PY{p}{,}\PY{p}{:}\PY{p}{]}\PY{o}{.}\PY{n+nf+fm}{\PYZus{}\PYZus{}len\PYZus{}\PYZus{}}\PY{p}{(}\PY{p}{)}\PY{p}{)}
\end{Verbatim}


    \begin{Verbatim}[commandchars=\\\{\}]
66
17

    \end{Verbatim}

    B.7. All'interno del dataset che stiamo considerando, quanti cani
deceduti per cause cardiache avevano il valore di OndaEA ≥ s? E quanti
cani deceduti per cause non cardiache avevano il valore di OndaEA
\textless{} s?

    \begin{Verbatim}[commandchars=\\\{\}]
{\color{incolor}In [{\color{incolor}32}]:} \PY{n+nb}{print}\PY{p}{(}\PY{n}{dataB}\PY{o}{.}\PY{n}{loc}\PY{p}{[}\PY{p}{(}\PY{n}{dataB}\PY{o}{.}\PY{n}{MC}\PY{o}{==}\PY{l+m+mi}{1}\PY{p}{)} \PY{o}{\PYZam{}} \PY{p}{(}\PY{n}{dataB}\PY{o}{.}\PY{n}{OndaEA} \PY{o}{\PYZgt{}}\PY{o}{=} \PY{n}{s}\PY{p}{)}\PY{p}{,}\PY{p}{:} \PY{p}{]}\PY{o}{.}\PY{n+nf+fm}{\PYZus{}\PYZus{}len\PYZus{}\PYZus{}}\PY{p}{(}\PY{p}{)}\PY{p}{)}
         \PY{n+nb}{print}\PY{p}{(}\PY{n}{dataB}\PY{o}{.}\PY{n}{loc}\PY{p}{[}\PY{p}{(}\PY{n}{dataB}\PY{o}{.}\PY{n}{MC}\PY{o}{==}\PY{l+m+mi}{0}\PY{p}{)} \PY{o}{\PYZam{}} \PY{p}{(}\PY{n}{dataB}\PY{o}{.}\PY{n}{OndaEA} \PY{o}{\PYZlt{}}  \PY{n}{s}\PY{p}{)}\PY{p}{,}\PY{p}{:} \PY{p}{]}\PY{o}{.}\PY{n+nf+fm}{\PYZus{}\PYZus{}len\PYZus{}\PYZus{}}\PY{p}{(}\PY{p}{)}\PY{p}{)}
\end{Verbatim}


    \begin{Verbatim}[commandchars=\\\{\}]
41
12

    \end{Verbatim}

    B.8. Utilizziamo il valorestrovato al punto 5. come soglia per un
classificatore binario chediscrimina tra morte cardiaca e morte non
cardiaca: il classificatore classificherà come morte cardiaca i casi per
i quali OndaEA ≥ s e come morte non cardiaca i casi per i quali OndaEA
\textless{} s. Calcolare la sensibilità e la specificità di questo
classificatore.

    \begin{Verbatim}[commandchars=\\\{\}]
{\color{incolor}In [{\color{incolor}33}]:} \PY{n}{TP} \PY{o}{=} \PY{n}{dataB}\PY{o}{.}\PY{n}{loc}\PY{p}{[}\PY{p}{(}\PY{n}{dataB}\PY{o}{.}\PY{n}{MC}\PY{o}{==}\PY{l+m+mi}{1}\PY{p}{)} \PY{o}{\PYZam{}} \PY{p}{(}\PY{n}{dataB}\PY{o}{.}\PY{n}{OndaEA} \PY{o}{\PYZgt{}}\PY{o}{=} \PY{n}{s}\PY{p}{)}\PY{p}{,}\PY{p}{:}\PY{p}{]}\PY{o}{.}\PY{n+nf+fm}{\PYZus{}\PYZus{}len\PYZus{}\PYZus{}}\PY{p}{(}\PY{p}{)}
         \PY{n}{FP} \PY{o}{=} \PY{n}{dataB}\PY{o}{.}\PY{n}{loc}\PY{p}{[}\PY{p}{(}\PY{n}{dataB}\PY{o}{.}\PY{n}{MC}\PY{o}{==}\PY{l+m+mi}{0}\PY{p}{)} \PY{o}{\PYZam{}} \PY{p}{(}\PY{n}{dataB}\PY{o}{.}\PY{n}{OndaEA} \PY{o}{\PYZgt{}}\PY{o}{=} \PY{n}{s}\PY{p}{)}\PY{p}{,}\PY{p}{:}\PY{p}{]}\PY{o}{.}\PY{n+nf+fm}{\PYZus{}\PYZus{}len\PYZus{}\PYZus{}}\PY{p}{(}\PY{p}{)}
         \PY{n}{TN} \PY{o}{=} \PY{n}{dataB}\PY{o}{.}\PY{n}{loc}\PY{p}{[}\PY{p}{(}\PY{n}{dataB}\PY{o}{.}\PY{n}{MC}\PY{o}{==}\PY{l+m+mi}{0}\PY{p}{)} \PY{o}{\PYZam{}} \PY{p}{(}\PY{n}{dataB}\PY{o}{.}\PY{n}{OndaEA} \PY{o}{\PYZlt{}}  \PY{n}{s}\PY{p}{)}\PY{p}{,}\PY{p}{:}\PY{p}{]}\PY{o}{.}\PY{n+nf+fm}{\PYZus{}\PYZus{}len\PYZus{}\PYZus{}}\PY{p}{(}\PY{p}{)}
         \PY{n}{FN} \PY{o}{=} \PY{n}{dataB}\PY{o}{.}\PY{n}{loc}\PY{p}{[}\PY{p}{(}\PY{n}{dataB}\PY{o}{.}\PY{n}{MC}\PY{o}{==}\PY{l+m+mi}{1}\PY{p}{)} \PY{o}{\PYZam{}} \PY{p}{(}\PY{n}{dataB}\PY{o}{.}\PY{n}{OndaEA} \PY{o}{\PYZlt{}}  \PY{n}{s}\PY{p}{)}\PY{p}{,}\PY{p}{:}\PY{p}{]}\PY{o}{.}\PY{n+nf+fm}{\PYZus{}\PYZus{}len\PYZus{}\PYZus{}}\PY{p}{(}\PY{p}{)}
         \PY{n}{sensibilita} \PY{o}{=} \PY{n}{TP} \PY{o}{/} \PY{p}{(} \PY{n}{TP} \PY{o}{+} \PY{n}{FN} \PY{p}{)}
         \PY{n}{specificita} \PY{o}{=} \PY{n}{TN} \PY{o}{/} \PY{p}{(} \PY{n}{FP} \PY{o}{+} \PY{n}{TN} \PY{p}{)}
         \PY{n+nb}{print}\PY{p}{(}\PY{n}{sensibilita}\PY{p}{)}   \PY{c+c1}{\PYZsh{} True positive rate TPR}
         \PY{n+nb}{print}\PY{p}{(}\PY{n}{specificita}\PY{p}{)}   \PY{c+c1}{\PYZsh{} True negative rate TNR}
         \PY{n+nb}{print}\PY{p}{(}\PY{l+m+mi}{1}\PY{o}{\PYZhy{}}\PY{n}{specificita}\PY{p}{)} \PY{c+c1}{\PYZsh{} False positive rate FPR}
\end{Verbatim}


    \begin{Verbatim}[commandchars=\\\{\}]
0.6212121212121212
0.7058823529411765
0.2941176470588235

    \end{Verbatim}

    B.9. Tracciare il grafico della curva ROC per il classificatore
individuato nei punti precedenti, basato sul carattere OndaEA.

    \begin{Verbatim}[commandchars=\\\{\}]
{\color{incolor}In [{\color{incolor}23}]:} \PY{k+kn}{import} \PY{n+nn}{numpy} \PY{k}{as} \PY{n+nn}{np}
         \PY{n}{soglie} \PY{o}{=} \PY{n+nb}{sorted}\PY{p}{(}\PY{n}{data}\PY{o}{.}\PY{n}{OndaEA}\PY{o}{.}\PY{n}{dropna}\PY{p}{(}\PY{p}{)}\PY{o}{.}\PY{n}{unique}\PY{p}{(}\PY{p}{)}\PY{p}{)}
         \PY{n}{sensibilita} \PY{o}{=} \PY{p}{[}\PY{p}{]}
         \PY{n}{specificita} \PY{o}{=} \PY{p}{[}\PY{p}{]}
         \PY{k}{for} \PY{n}{s} \PY{o+ow}{in} \PY{n}{soglie}\PY{p}{:}
             \PY{n}{TP} \PY{o}{=} \PY{n}{dataB}\PY{o}{.}\PY{n}{loc}\PY{p}{[}\PY{p}{(}\PY{o}{\PYZti{}}\PY{n}{pandas}\PY{o}{.}\PY{n}{isna}\PY{p}{(}\PY{n}{dataB}\PY{o}{.}\PY{n}{MC}\PY{p}{)}\PY{p}{)} \PY{o}{\PYZam{}} 
               \PY{p}{(}\PY{o}{\PYZti{}}\PY{n}{pandas}\PY{o}{.}\PY{n}{isna}\PY{p}{(}\PY{n}{dataB}\PY{o}{.}\PY{n}{OndaEA}\PY{p}{)}\PY{p}{)} \PY{o}{\PYZam{}} \PY{p}{(}\PY{n}{dataB}\PY{o}{.}\PY{n}{MC}\PY{o}{==}\PY{l+m+mi}{1}\PY{p}{)} 
               \PY{o}{\PYZam{}} \PY{p}{(}\PY{n}{dataB}\PY{o}{.}\PY{n}{OndaEA} \PY{o}{\PYZgt{}}\PY{o}{=} \PY{n}{s}\PY{p}{)}\PY{p}{,}\PY{p}{:}\PY{p}{]}\PY{o}{.}\PY{n+nf+fm}{\PYZus{}\PYZus{}len\PYZus{}\PYZus{}}\PY{p}{(}\PY{p}{)}
             \PY{n}{FP} \PY{o}{=} \PY{n}{dataB}\PY{o}{.}\PY{n}{loc}\PY{p}{[}\PY{p}{(}\PY{o}{\PYZti{}}\PY{n}{pandas}\PY{o}{.}\PY{n}{isna}\PY{p}{(}\PY{n}{dataB}\PY{o}{.}\PY{n}{MC}\PY{p}{)}\PY{p}{)} \PY{o}{\PYZam{}} 
                 \PY{p}{(}\PY{o}{\PYZti{}}\PY{n}{pandas}\PY{o}{.}\PY{n}{isna}\PY{p}{(}\PY{n}{dataB}\PY{o}{.}\PY{n}{OndaEA}\PY{p}{)}\PY{p}{)} \PY{o}{\PYZam{}} \PY{p}{(}\PY{n}{dataB}\PY{o}{.}\PY{n}{MC}\PY{o}{==}\PY{l+m+mi}{0}\PY{p}{)} \PY{o}{\PYZam{}} 
                            \PY{p}{(}\PY{n}{dataB}\PY{o}{.}\PY{n}{OndaEA} \PY{o}{\PYZgt{}}\PY{o}{=} \PY{n}{s}\PY{p}{)}\PY{p}{,}\PY{p}{:}\PY{p}{]}\PY{o}{.}\PY{n+nf+fm}{\PYZus{}\PYZus{}len\PYZus{}\PYZus{}}\PY{p}{(}\PY{p}{)}
             \PY{n}{TN} \PY{o}{=} \PY{n}{dataB}\PY{o}{.}\PY{n}{loc}\PY{p}{[}\PY{p}{(}\PY{o}{\PYZti{}}\PY{n}{pandas}\PY{o}{.}\PY{n}{isna}\PY{p}{(}\PY{n}{dataB}\PY{o}{.}\PY{n}{MC}\PY{p}{)}\PY{p}{)} \PY{o}{\PYZam{}} 
                            \PY{p}{(}\PY{o}{\PYZti{}}\PY{n}{pandas}\PY{o}{.}\PY{n}{isna}\PY{p}{(}\PY{n}{dataB}\PY{o}{.}\PY{n}{OndaEA}\PY{p}{)}\PY{p}{)} \PY{o}{\PYZam{}} 
                     \PY{p}{(}\PY{n}{dataB}\PY{o}{.}\PY{n}{MC}\PY{o}{==}\PY{l+m+mi}{0}\PY{p}{)} \PY{o}{\PYZam{}} \PY{p}{(}\PY{n}{dataB}\PY{o}{.}\PY{n}{OndaEA} \PY{o}{\PYZlt{}}  \PY{n}{s}\PY{p}{)}\PY{p}{,}\PY{p}{:}\PY{p}{]}\PY{o}{.}\PY{n+nf+fm}{\PYZus{}\PYZus{}len\PYZus{}\PYZus{}}\PY{p}{(}\PY{p}{)}
             \PY{n}{FN} \PY{o}{=} \PY{n}{dataB}\PY{o}{.}\PY{n}{loc}\PY{p}{[}\PY{p}{(}\PY{o}{\PYZti{}}\PY{n}{pandas}\PY{o}{.}\PY{n}{isna}\PY{p}{(}\PY{n}{dataB}\PY{o}{.}\PY{n}{MC}\PY{p}{)}\PY{p}{)} \PY{o}{\PYZam{}}
                            \PY{p}{(}\PY{o}{\PYZti{}}\PY{n}{pandas}\PY{o}{.}\PY{n}{isna}\PY{p}{(}\PY{n}{dataB}\PY{o}{.}\PY{n}{OndaEA}\PY{p}{)}\PY{p}{)} \PY{o}{\PYZam{}} 
                     \PY{p}{(}\PY{n}{dataB}\PY{o}{.}\PY{n}{MC}\PY{o}{==}\PY{l+m+mi}{1}\PY{p}{)} \PY{o}{\PYZam{}} \PY{p}{(}\PY{n}{dataB}\PY{o}{.}\PY{n}{OndaEA} \PY{o}{\PYZlt{}}  \PY{n}{s}\PY{p}{)}\PY{p}{,}\PY{p}{:}\PY{p}{]}\PY{o}{.}\PY{n+nf+fm}{\PYZus{}\PYZus{}len\PYZus{}\PYZus{}}\PY{p}{(}\PY{p}{)}
             \PY{n}{sensibilita}\PY{o}{.}\PY{n}{append}\PY{p}{(}\PY{n}{TP} \PY{o}{/} \PY{p}{(} \PY{n}{TP} \PY{o}{+} \PY{n}{FN} \PY{p}{)}\PY{p}{)}
             \PY{n}{specificita}\PY{o}{.}\PY{n}{append}\PY{p}{(}\PY{n}{TN} \PY{o}{/} \PY{p}{(} \PY{n}{FP} \PY{o}{+} \PY{n}{TN} \PY{p}{)}\PY{p}{)}
         
             
             
         \PY{n}{plt0} \PY{o}{=} \PY{n}{plt}\PY{o}{.}\PY{n}{plot}\PY{p}{(}\PY{l+m+mi}{1}\PY{o}{\PYZhy{}}\PY{n}{np}\PY{o}{.}\PY{n}{array}\PY{p}{(}\PY{n}{specificita}\PY{p}{)}\PY{p}{,} \PY{n}{np}\PY{o}{.}\PY{n}{array}\PY{p}{(}\PY{n}{sensibilita}\PY{p}{)}\PY{p}{)}
         \PY{n}{plt0} \PY{o}{=} \PY{n}{plt}\PY{o}{.}\PY{n}{plot}\PY{p}{(}\PY{p}{[}\PY{l+m+mi}{0}\PY{p}{,}\PY{l+m+mi}{1}\PY{p}{]}\PY{p}{,}\PY{p}{[}\PY{l+m+mi}{0}\PY{p}{,}\PY{l+m+mi}{1}\PY{p}{]}\PY{p}{)}
         \PY{c+c1}{\PYZsh{}plot(1\PYZhy{}specificita,sensibilita,type = \PYZdq{}l\PYZdq{},xlab = \PYZdq{}FPR\PYZdq{},ylab = }
         \PY{c+c1}{\PYZsh{}\PYZdq{}TPR\PYZdq{}, col=\PYZdq{}blue\PYZdq{})}
         \PY{c+c1}{\PYZsh{}lines(c(0,1),c(0,1),type=\PYZdq{}l\PYZdq{},col=\PYZdq{}grey\PYZdq{})}
\end{Verbatim}


    \begin{center}
    \adjustimage{max size={0.9\linewidth}{0.9\paperheight}}{output_68_0.png}
    \end{center}
    { \hspace*{\fill} \\}
    

    % Add a bibliography block to the postdoc
    
    
    
    \end{document}
