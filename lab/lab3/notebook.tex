
% Default to the notebook output style

    


% Inherit from the specified cell style.




    
\documentclass[11pt]{article}

    
    
    \usepackage[T1]{fontenc}
    % Nicer default font (+ math font) than Computer Modern for most use cases
    \usepackage{mathpazo}

    % Basic figure setup, for now with no caption control since it's done
    % automatically by Pandoc (which extracts ![](path) syntax from Markdown).
    \usepackage{graphicx}
    % We will generate all images so they have a width \maxwidth. This means
    % that they will get their normal width if they fit onto the page, but
    % are scaled down if they would overflow the margins.
    \makeatletter
    \def\maxwidth{\ifdim\Gin@nat@width>\linewidth\linewidth
    \else\Gin@nat@width\fi}
    \makeatother
    \let\Oldincludegraphics\includegraphics
    % Set max figure width to be 80% of text width, for now hardcoded.
    \renewcommand{\includegraphics}[1]{\Oldincludegraphics[width=.8\maxwidth]{#1}}
    % Ensure that by default, figures have no caption (until we provide a
    % proper Figure object with a Caption API and a way to capture that
    % in the conversion process - todo).
    \usepackage{caption}
    \DeclareCaptionLabelFormat{nolabel}{}
    \captionsetup{labelformat=nolabel}

    \usepackage{adjustbox} % Used to constrain images to a maximum size 
    \usepackage{xcolor} % Allow colors to be defined
    \usepackage{enumerate} % Needed for markdown enumerations to work
    \usepackage{geometry} % Used to adjust the document margins
    \usepackage{amsmath} % Equations
    \usepackage{amssymb} % Equations
    \usepackage{textcomp} % defines textquotesingle
    % Hack from http://tex.stackexchange.com/a/47451/13684:
    \AtBeginDocument{%
        \def\PYZsq{\textquotesingle}% Upright quotes in Pygmentized code
    }
    \usepackage{upquote} % Upright quotes for verbatim code
    \usepackage{eurosym} % defines \euro
    \usepackage[mathletters]{ucs} % Extended unicode (utf-8) support
    \usepackage[utf8x]{inputenc} % Allow utf-8 characters in the tex document
    \usepackage{fancyvrb} % verbatim replacement that allows latex
    \usepackage{grffile} % extends the file name processing of package graphics 
                         % to support a larger range 
    % The hyperref package gives us a pdf with properly built
    % internal navigation ('pdf bookmarks' for the table of contents,
    % internal cross-reference links, web links for URLs, etc.)
    \usepackage{hyperref}
    \usepackage{longtable} % longtable support required by pandoc >1.10
    \usepackage{booktabs}  % table support for pandoc > 1.12.2
    \usepackage[inline]{enumitem} % IRkernel/repr support (it uses the enumerate* environment)
    \usepackage[normalem]{ulem} % ulem is needed to support strikethroughs (\sout)
                                % normalem makes italics be italics, not underlines
    

    
    
    % Colors for the hyperref package
    \definecolor{urlcolor}{rgb}{0,.145,.698}
    \definecolor{linkcolor}{rgb}{.71,0.21,0.01}
    \definecolor{citecolor}{rgb}{.12,.54,.11}

    % ANSI colors
    \definecolor{ansi-black}{HTML}{3E424D}
    \definecolor{ansi-black-intense}{HTML}{282C36}
    \definecolor{ansi-red}{HTML}{E75C58}
    \definecolor{ansi-red-intense}{HTML}{B22B31}
    \definecolor{ansi-green}{HTML}{00A250}
    \definecolor{ansi-green-intense}{HTML}{007427}
    \definecolor{ansi-yellow}{HTML}{DDB62B}
    \definecolor{ansi-yellow-intense}{HTML}{B27D12}
    \definecolor{ansi-blue}{HTML}{208FFB}
    \definecolor{ansi-blue-intense}{HTML}{0065CA}
    \definecolor{ansi-magenta}{HTML}{D160C4}
    \definecolor{ansi-magenta-intense}{HTML}{A03196}
    \definecolor{ansi-cyan}{HTML}{60C6C8}
    \definecolor{ansi-cyan-intense}{HTML}{258F8F}
    \definecolor{ansi-white}{HTML}{C5C1B4}
    \definecolor{ansi-white-intense}{HTML}{A1A6B2}

    % commands and environments needed by pandoc snippets
    % extracted from the output of `pandoc -s`
    \providecommand{\tightlist}{%
      \setlength{\itemsep}{0pt}\setlength{\parskip}{0pt}}
    \DefineVerbatimEnvironment{Highlighting}{Verbatim}{commandchars=\\\{\}}
    % Add ',fontsize=\small' for more characters per line
    \newenvironment{Shaded}{}{}
    \newcommand{\KeywordTok}[1]{\textcolor[rgb]{0.00,0.44,0.13}{\textbf{{#1}}}}
    \newcommand{\DataTypeTok}[1]{\textcolor[rgb]{0.56,0.13,0.00}{{#1}}}
    \newcommand{\DecValTok}[1]{\textcolor[rgb]{0.25,0.63,0.44}{{#1}}}
    \newcommand{\BaseNTok}[1]{\textcolor[rgb]{0.25,0.63,0.44}{{#1}}}
    \newcommand{\FloatTok}[1]{\textcolor[rgb]{0.25,0.63,0.44}{{#1}}}
    \newcommand{\CharTok}[1]{\textcolor[rgb]{0.25,0.44,0.63}{{#1}}}
    \newcommand{\StringTok}[1]{\textcolor[rgb]{0.25,0.44,0.63}{{#1}}}
    \newcommand{\CommentTok}[1]{\textcolor[rgb]{0.38,0.63,0.69}{\textit{{#1}}}}
    \newcommand{\OtherTok}[1]{\textcolor[rgb]{0.00,0.44,0.13}{{#1}}}
    \newcommand{\AlertTok}[1]{\textcolor[rgb]{1.00,0.00,0.00}{\textbf{{#1}}}}
    \newcommand{\FunctionTok}[1]{\textcolor[rgb]{0.02,0.16,0.49}{{#1}}}
    \newcommand{\RegionMarkerTok}[1]{{#1}}
    \newcommand{\ErrorTok}[1]{\textcolor[rgb]{1.00,0.00,0.00}{\textbf{{#1}}}}
    \newcommand{\NormalTok}[1]{{#1}}
    
    % Additional commands for more recent versions of Pandoc
    \newcommand{\ConstantTok}[1]{\textcolor[rgb]{0.53,0.00,0.00}{{#1}}}
    \newcommand{\SpecialCharTok}[1]{\textcolor[rgb]{0.25,0.44,0.63}{{#1}}}
    \newcommand{\VerbatimStringTok}[1]{\textcolor[rgb]{0.25,0.44,0.63}{{#1}}}
    \newcommand{\SpecialStringTok}[1]{\textcolor[rgb]{0.73,0.40,0.53}{{#1}}}
    \newcommand{\ImportTok}[1]{{#1}}
    \newcommand{\DocumentationTok}[1]{\textcolor[rgb]{0.73,0.13,0.13}{\textit{{#1}}}}
    \newcommand{\AnnotationTok}[1]{\textcolor[rgb]{0.38,0.63,0.69}{\textbf{\textit{{#1}}}}}
    \newcommand{\CommentVarTok}[1]{\textcolor[rgb]{0.38,0.63,0.69}{\textbf{\textit{{#1}}}}}
    \newcommand{\VariableTok}[1]{\textcolor[rgb]{0.10,0.09,0.49}{{#1}}}
    \newcommand{\ControlFlowTok}[1]{\textcolor[rgb]{0.00,0.44,0.13}{\textbf{{#1}}}}
    \newcommand{\OperatorTok}[1]{\textcolor[rgb]{0.40,0.40,0.40}{{#1}}}
    \newcommand{\BuiltInTok}[1]{{#1}}
    \newcommand{\ExtensionTok}[1]{{#1}}
    \newcommand{\PreprocessorTok}[1]{\textcolor[rgb]{0.74,0.48,0.00}{{#1}}}
    \newcommand{\AttributeTok}[1]{\textcolor[rgb]{0.49,0.56,0.16}{{#1}}}
    \newcommand{\InformationTok}[1]{\textcolor[rgb]{0.38,0.63,0.69}{\textbf{\textit{{#1}}}}}
    \newcommand{\WarningTok}[1]{\textcolor[rgb]{0.38,0.63,0.69}{\textbf{\textit{{#1}}}}}
    
    
    % Define a nice break command that doesn't care if a line doesn't already
    % exist.
    \def\br{\hspace*{\fill} \\* }
    % Math Jax compatability definitions
    \def\gt{>}
    \def\lt{<}
    % Document parameters
    \title{solution\_python}
    
    
    

    % Pygments definitions
    
\makeatletter
\def\PY@reset{\let\PY@it=\relax \let\PY@bf=\relax%
    \let\PY@ul=\relax \let\PY@tc=\relax%
    \let\PY@bc=\relax \let\PY@ff=\relax}
\def\PY@tok#1{\csname PY@tok@#1\endcsname}
\def\PY@toks#1+{\ifx\relax#1\empty\else%
    \PY@tok{#1}\expandafter\PY@toks\fi}
\def\PY@do#1{\PY@bc{\PY@tc{\PY@ul{%
    \PY@it{\PY@bf{\PY@ff{#1}}}}}}}
\def\PY#1#2{\PY@reset\PY@toks#1+\relax+\PY@do{#2}}

\expandafter\def\csname PY@tok@w\endcsname{\def\PY@tc##1{\textcolor[rgb]{0.73,0.73,0.73}{##1}}}
\expandafter\def\csname PY@tok@c\endcsname{\let\PY@it=\textit\def\PY@tc##1{\textcolor[rgb]{0.25,0.50,0.50}{##1}}}
\expandafter\def\csname PY@tok@cp\endcsname{\def\PY@tc##1{\textcolor[rgb]{0.74,0.48,0.00}{##1}}}
\expandafter\def\csname PY@tok@k\endcsname{\let\PY@bf=\textbf\def\PY@tc##1{\textcolor[rgb]{0.00,0.50,0.00}{##1}}}
\expandafter\def\csname PY@tok@kp\endcsname{\def\PY@tc##1{\textcolor[rgb]{0.00,0.50,0.00}{##1}}}
\expandafter\def\csname PY@tok@kt\endcsname{\def\PY@tc##1{\textcolor[rgb]{0.69,0.00,0.25}{##1}}}
\expandafter\def\csname PY@tok@o\endcsname{\def\PY@tc##1{\textcolor[rgb]{0.40,0.40,0.40}{##1}}}
\expandafter\def\csname PY@tok@ow\endcsname{\let\PY@bf=\textbf\def\PY@tc##1{\textcolor[rgb]{0.67,0.13,1.00}{##1}}}
\expandafter\def\csname PY@tok@nb\endcsname{\def\PY@tc##1{\textcolor[rgb]{0.00,0.50,0.00}{##1}}}
\expandafter\def\csname PY@tok@nf\endcsname{\def\PY@tc##1{\textcolor[rgb]{0.00,0.00,1.00}{##1}}}
\expandafter\def\csname PY@tok@nc\endcsname{\let\PY@bf=\textbf\def\PY@tc##1{\textcolor[rgb]{0.00,0.00,1.00}{##1}}}
\expandafter\def\csname PY@tok@nn\endcsname{\let\PY@bf=\textbf\def\PY@tc##1{\textcolor[rgb]{0.00,0.00,1.00}{##1}}}
\expandafter\def\csname PY@tok@ne\endcsname{\let\PY@bf=\textbf\def\PY@tc##1{\textcolor[rgb]{0.82,0.25,0.23}{##1}}}
\expandafter\def\csname PY@tok@nv\endcsname{\def\PY@tc##1{\textcolor[rgb]{0.10,0.09,0.49}{##1}}}
\expandafter\def\csname PY@tok@no\endcsname{\def\PY@tc##1{\textcolor[rgb]{0.53,0.00,0.00}{##1}}}
\expandafter\def\csname PY@tok@nl\endcsname{\def\PY@tc##1{\textcolor[rgb]{0.63,0.63,0.00}{##1}}}
\expandafter\def\csname PY@tok@ni\endcsname{\let\PY@bf=\textbf\def\PY@tc##1{\textcolor[rgb]{0.60,0.60,0.60}{##1}}}
\expandafter\def\csname PY@tok@na\endcsname{\def\PY@tc##1{\textcolor[rgb]{0.49,0.56,0.16}{##1}}}
\expandafter\def\csname PY@tok@nt\endcsname{\let\PY@bf=\textbf\def\PY@tc##1{\textcolor[rgb]{0.00,0.50,0.00}{##1}}}
\expandafter\def\csname PY@tok@nd\endcsname{\def\PY@tc##1{\textcolor[rgb]{0.67,0.13,1.00}{##1}}}
\expandafter\def\csname PY@tok@s\endcsname{\def\PY@tc##1{\textcolor[rgb]{0.73,0.13,0.13}{##1}}}
\expandafter\def\csname PY@tok@sd\endcsname{\let\PY@it=\textit\def\PY@tc##1{\textcolor[rgb]{0.73,0.13,0.13}{##1}}}
\expandafter\def\csname PY@tok@si\endcsname{\let\PY@bf=\textbf\def\PY@tc##1{\textcolor[rgb]{0.73,0.40,0.53}{##1}}}
\expandafter\def\csname PY@tok@se\endcsname{\let\PY@bf=\textbf\def\PY@tc##1{\textcolor[rgb]{0.73,0.40,0.13}{##1}}}
\expandafter\def\csname PY@tok@sr\endcsname{\def\PY@tc##1{\textcolor[rgb]{0.73,0.40,0.53}{##1}}}
\expandafter\def\csname PY@tok@ss\endcsname{\def\PY@tc##1{\textcolor[rgb]{0.10,0.09,0.49}{##1}}}
\expandafter\def\csname PY@tok@sx\endcsname{\def\PY@tc##1{\textcolor[rgb]{0.00,0.50,0.00}{##1}}}
\expandafter\def\csname PY@tok@m\endcsname{\def\PY@tc##1{\textcolor[rgb]{0.40,0.40,0.40}{##1}}}
\expandafter\def\csname PY@tok@gh\endcsname{\let\PY@bf=\textbf\def\PY@tc##1{\textcolor[rgb]{0.00,0.00,0.50}{##1}}}
\expandafter\def\csname PY@tok@gu\endcsname{\let\PY@bf=\textbf\def\PY@tc##1{\textcolor[rgb]{0.50,0.00,0.50}{##1}}}
\expandafter\def\csname PY@tok@gd\endcsname{\def\PY@tc##1{\textcolor[rgb]{0.63,0.00,0.00}{##1}}}
\expandafter\def\csname PY@tok@gi\endcsname{\def\PY@tc##1{\textcolor[rgb]{0.00,0.63,0.00}{##1}}}
\expandafter\def\csname PY@tok@gr\endcsname{\def\PY@tc##1{\textcolor[rgb]{1.00,0.00,0.00}{##1}}}
\expandafter\def\csname PY@tok@ge\endcsname{\let\PY@it=\textit}
\expandafter\def\csname PY@tok@gs\endcsname{\let\PY@bf=\textbf}
\expandafter\def\csname PY@tok@gp\endcsname{\let\PY@bf=\textbf\def\PY@tc##1{\textcolor[rgb]{0.00,0.00,0.50}{##1}}}
\expandafter\def\csname PY@tok@go\endcsname{\def\PY@tc##1{\textcolor[rgb]{0.53,0.53,0.53}{##1}}}
\expandafter\def\csname PY@tok@gt\endcsname{\def\PY@tc##1{\textcolor[rgb]{0.00,0.27,0.87}{##1}}}
\expandafter\def\csname PY@tok@err\endcsname{\def\PY@bc##1{\setlength{\fboxsep}{0pt}\fcolorbox[rgb]{1.00,0.00,0.00}{1,1,1}{\strut ##1}}}
\expandafter\def\csname PY@tok@kc\endcsname{\let\PY@bf=\textbf\def\PY@tc##1{\textcolor[rgb]{0.00,0.50,0.00}{##1}}}
\expandafter\def\csname PY@tok@kd\endcsname{\let\PY@bf=\textbf\def\PY@tc##1{\textcolor[rgb]{0.00,0.50,0.00}{##1}}}
\expandafter\def\csname PY@tok@kn\endcsname{\let\PY@bf=\textbf\def\PY@tc##1{\textcolor[rgb]{0.00,0.50,0.00}{##1}}}
\expandafter\def\csname PY@tok@kr\endcsname{\let\PY@bf=\textbf\def\PY@tc##1{\textcolor[rgb]{0.00,0.50,0.00}{##1}}}
\expandafter\def\csname PY@tok@bp\endcsname{\def\PY@tc##1{\textcolor[rgb]{0.00,0.50,0.00}{##1}}}
\expandafter\def\csname PY@tok@fm\endcsname{\def\PY@tc##1{\textcolor[rgb]{0.00,0.00,1.00}{##1}}}
\expandafter\def\csname PY@tok@vc\endcsname{\def\PY@tc##1{\textcolor[rgb]{0.10,0.09,0.49}{##1}}}
\expandafter\def\csname PY@tok@vg\endcsname{\def\PY@tc##1{\textcolor[rgb]{0.10,0.09,0.49}{##1}}}
\expandafter\def\csname PY@tok@vi\endcsname{\def\PY@tc##1{\textcolor[rgb]{0.10,0.09,0.49}{##1}}}
\expandafter\def\csname PY@tok@vm\endcsname{\def\PY@tc##1{\textcolor[rgb]{0.10,0.09,0.49}{##1}}}
\expandafter\def\csname PY@tok@sa\endcsname{\def\PY@tc##1{\textcolor[rgb]{0.73,0.13,0.13}{##1}}}
\expandafter\def\csname PY@tok@sb\endcsname{\def\PY@tc##1{\textcolor[rgb]{0.73,0.13,0.13}{##1}}}
\expandafter\def\csname PY@tok@sc\endcsname{\def\PY@tc##1{\textcolor[rgb]{0.73,0.13,0.13}{##1}}}
\expandafter\def\csname PY@tok@dl\endcsname{\def\PY@tc##1{\textcolor[rgb]{0.73,0.13,0.13}{##1}}}
\expandafter\def\csname PY@tok@s2\endcsname{\def\PY@tc##1{\textcolor[rgb]{0.73,0.13,0.13}{##1}}}
\expandafter\def\csname PY@tok@sh\endcsname{\def\PY@tc##1{\textcolor[rgb]{0.73,0.13,0.13}{##1}}}
\expandafter\def\csname PY@tok@s1\endcsname{\def\PY@tc##1{\textcolor[rgb]{0.73,0.13,0.13}{##1}}}
\expandafter\def\csname PY@tok@mb\endcsname{\def\PY@tc##1{\textcolor[rgb]{0.40,0.40,0.40}{##1}}}
\expandafter\def\csname PY@tok@mf\endcsname{\def\PY@tc##1{\textcolor[rgb]{0.40,0.40,0.40}{##1}}}
\expandafter\def\csname PY@tok@mh\endcsname{\def\PY@tc##1{\textcolor[rgb]{0.40,0.40,0.40}{##1}}}
\expandafter\def\csname PY@tok@mi\endcsname{\def\PY@tc##1{\textcolor[rgb]{0.40,0.40,0.40}{##1}}}
\expandafter\def\csname PY@tok@il\endcsname{\def\PY@tc##1{\textcolor[rgb]{0.40,0.40,0.40}{##1}}}
\expandafter\def\csname PY@tok@mo\endcsname{\def\PY@tc##1{\textcolor[rgb]{0.40,0.40,0.40}{##1}}}
\expandafter\def\csname PY@tok@ch\endcsname{\let\PY@it=\textit\def\PY@tc##1{\textcolor[rgb]{0.25,0.50,0.50}{##1}}}
\expandafter\def\csname PY@tok@cm\endcsname{\let\PY@it=\textit\def\PY@tc##1{\textcolor[rgb]{0.25,0.50,0.50}{##1}}}
\expandafter\def\csname PY@tok@cpf\endcsname{\let\PY@it=\textit\def\PY@tc##1{\textcolor[rgb]{0.25,0.50,0.50}{##1}}}
\expandafter\def\csname PY@tok@c1\endcsname{\let\PY@it=\textit\def\PY@tc##1{\textcolor[rgb]{0.25,0.50,0.50}{##1}}}
\expandafter\def\csname PY@tok@cs\endcsname{\let\PY@it=\textit\def\PY@tc##1{\textcolor[rgb]{0.25,0.50,0.50}{##1}}}

\def\PYZbs{\char`\\}
\def\PYZus{\char`\_}
\def\PYZob{\char`\{}
\def\PYZcb{\char`\}}
\def\PYZca{\char`\^}
\def\PYZam{\char`\&}
\def\PYZlt{\char`\<}
\def\PYZgt{\char`\>}
\def\PYZsh{\char`\#}
\def\PYZpc{\char`\%}
\def\PYZdl{\char`\$}
\def\PYZhy{\char`\-}
\def\PYZsq{\char`\'}
\def\PYZdq{\char`\"}
\def\PYZti{\char`\~}
% for compatibility with earlier versions
\def\PYZat{@}
\def\PYZlb{[}
\def\PYZrb{]}
\makeatother


    % Exact colors from NB
    \definecolor{incolor}{rgb}{0.0, 0.0, 0.5}
    \definecolor{outcolor}{rgb}{0.545, 0.0, 0.0}



    
    % Prevent overflowing lines due to hard-to-break entities
    \sloppy 
    % Setup hyperref package
    \hypersetup{
      breaklinks=true,  % so long urls are correctly broken across lines
      colorlinks=true,
      urlcolor=urlcolor,
      linkcolor=linkcolor,
      citecolor=citecolor,
      }
    % Slightly bigger margins than the latex defaults
    
    \geometry{verbose,tmargin=1in,bmargin=1in,lmargin=1in,rmargin=1in}
    
    

    \begin{document}
    
    
    \maketitle
    
    

    
    \begin{Verbatim}[commandchars=\\\{\}]
{\color{incolor}In [{\color{incolor}1}]:} \PY{k+kn}{import} \PY{n+nn}{pandas}
        \PY{n}{data} \PY{o}{=} \PY{n}{pandas}\PY{o}{.}\PY{n}{read\PYZus{}csv}\PY{p}{(}\PY{l+s+s2}{\PYZdq{}}\PY{l+s+s2}{finanziamenti.csv}\PY{l+s+s2}{\PYZdq{}}\PY{p}{,} \PY{n}{sep}\PY{o}{=}\PY{l+s+s2}{\PYZdq{}}\PY{l+s+s2}{;}\PY{l+s+s2}{\PYZdq{}}\PY{p}{,} \PY{n}{decimal}\PY{o}{=}\PY{l+s+s2}{\PYZdq{}}\PY{l+s+s2}{,}\PY{l+s+s2}{\PYZdq{}}\PY{p}{)}
        \PY{n}{data}\PY{o}{.}\PY{n}{loc}\PY{p}{[}\PY{p}{:}\PY{l+m+mi}{5}\PY{p}{,}\PY{p}{:}\PY{p}{]}
\end{Verbatim}


\begin{Verbatim}[commandchars=\\\{\}]
{\color{outcolor}Out[{\color{outcolor}1}]:}     id  TemaPrioritario                   FONTE  CodiceCategoria  \textbackslash{}
        0    4                5  Compet. per le imprese               18   
        1   38                5  Compet. per le imprese               40   
        2   39               11         Agenda digitale               40   
        3   43                4   Ricerca e innovazione               18   
        4   73                6   Ricerca e innovazione               39   
        5  106                6   Ricerca e innovazione               39   
        
                                                   CATEGORIA   UNITA  FinProvincia  \textbackslash{}
        0                            TECNOLOGIE INFORMATICHE  ASSISI      22356.25   
        1              OPERE E INFRASTRUTTURE PER LA RICERCA  ASSISI      99750.00   
        2              OPERE E INFRASTRUTTURE PER LA RICERCA  ASSISI       3802.90   
        3                            TECNOLOGIE INFORMATICHE  ASSISI     193020.00   
        4  OPERE, IMPIANTI ED ATTREZZATURE PER ATTIVITA' {\ldots}  ASSISI      10688.70   
        5  OPERE, IMPIANTI ED ATTREZZATURE PER ATTIVITA' {\ldots}  ASSISI      13060.00   
        
           FinRegione  TotSpese  
        0    22356.25   83037.5  
        1    99750.00  370500.0  
        2     3802.90   18423.2  
        3   193020.00  474360.0  
        4    10688.70       NaN  
        5    13060.00       NaN  
\end{Verbatim}
            
    \begin{Verbatim}[commandchars=\\\{\}]
{\color{incolor}In [{\color{incolor} }]:} \PY{n}{array}\PY{p}{(}\PY{n}{dim}\PY{o}{=}\PY{n}{c}\PY{p}{(}\PY{l+m+mi}{3}\PY{p}{,}\PY{l+m+mi}{3}\PY{p}{,}\PY{l+m+mi}{3}\PY{p}{)}\PY{p}{)}
\end{Verbatim}


    \begin{enumerate}
\def\labelenumi{\arabic{enumi}.}
\item
  Il carattere CodiceCategoria è nominale, ordinale o scalare?
  Giustificate la risposta. è nominale pur essendo codificato come un
  intero in quando rappresenta una categoria.
\item
  Calcolate la tabella delle frequenze assolute del carattere UNITA.
\end{enumerate}

    \begin{Verbatim}[commandchars=\\\{\}]
{\color{incolor}In [{\color{incolor}2}]:} \PY{n}{freqs} \PY{o}{=} \PY{n}{data}\PY{o}{.}\PY{n}{UNITA}\PY{o}{.}\PY{n}{value\PYZus{}counts}\PY{p}{(}\PY{p}{)}
        \PY{n+nb}{print}\PY{p}{(}\PY{n}{freqs}\PY{p}{)}
\end{Verbatim}


    \begin{Verbatim}[commandchars=\\\{\}]
PERUGIA                 1005
TERNI                    638
FOLIGNO                  449
CITTA DI CASTELLO        288
SLL NON ATTRIBUIBILE     255
ASSISI                   243
SPOLETO                  186
UMBERTIDE                124
TODI                     117
GUBBIO                    76
CASTIGLIONE DEL L         75
GUALDO TADINO             75
ORVIETO                   60
CHIUSI                    30
NORCIA                    28
CASCIA                    13
SLL MULTIPLO               5
CORTONA                    1
Name: UNITA, dtype: int64

    \end{Verbatim}

    \begin{enumerate}
\def\labelenumi{\arabic{enumi}.}
\setcounter{enumi}{2}
\tightlist
\item
  Tracciate un grafico opportuno per descrivere il carattere UNITA.
\end{enumerate}

    \begin{Verbatim}[commandchars=\\\{\}]
{\color{incolor}In [{\color{incolor}4}]:} \PY{k+kn}{import} \PY{n+nn}{matplotlib}\PY{n+nn}{.}\PY{n+nn}{pyplot} \PY{k}{as} \PY{n+nn}{plt}
        \PY{n}{fig}\PY{p}{,} \PY{n}{ax} \PY{o}{=} \PY{n}{plt}\PY{o}{.}\PY{n}{subplots}\PY{p}{(}\PY{p}{)}
        
        \PY{n}{ax}\PY{o}{.}\PY{n}{bar}\PY{p}{(}\PY{n+nb}{range}\PY{p}{(}\PY{n+nb}{len}\PY{p}{(}\PY{n}{freqs}\PY{o}{.}\PY{n}{values}\PY{p}{)}\PY{p}{)}\PY{p}{,}\PY{n}{freqs}\PY{o}{.}\PY{n}{values}\PY{p}{)}    
        \PY{c+c1}{\PYZsh{} set the bars}
        \PY{n}{ax}\PY{o}{.}\PY{n}{set\PYZus{}xticks}\PY{p}{(}\PY{n+nb}{range}\PY{p}{(}\PY{n+nb}{len}\PY{p}{(}\PY{n}{freqs}\PY{o}{.}\PY{n}{values}\PY{p}{)}\PY{p}{)}\PY{p}{)}          
        \PY{c+c1}{\PYZsh{} set where to put labels}
        \PY{n}{ax}\PY{o}{.}\PY{n}{set\PYZus{}xticklabels}\PY{p}{(}\PY{n}{freqs}\PY{o}{.}\PY{n}{keys}\PY{p}{(}\PY{p}{)}\PY{p}{,}\PY{n}{rotation}\PY{o}{=}\PY{l+m+mi}{90}\PY{p}{)}    
        \PY{c+c1}{\PYZsh{} set the labels and how to write them}
        
        \PY{n}{ax2} \PY{o}{=} \PY{n}{ax}\PY{o}{.}\PY{n}{twinx}\PY{p}{(}\PY{p}{)}
        \PY{n}{ax2}\PY{o}{.}\PY{n}{plot}\PY{p}{(}\PY{n+nb}{range}\PY{p}{(}\PY{n+nb}{len}\PY{p}{(}\PY{n}{freqs}\PY{o}{.}\PY{n}{values}\PY{p}{)}\PY{p}{)}\PY{p}{,} \PY{n}{freqs}\PY{o}{.}\PY{n}{cumsum}\PY{p}{(}\PY{p}{)}
                 \PY{o}{/}\PY{n+nb}{sum}\PY{p}{(}\PY{n}{freqs}\PY{o}{.}\PY{n}{values}\PY{p}{)}\PY{p}{,} \PY{n}{color}\PY{o}{=}\PY{l+s+s2}{\PYZdq{}}\PY{l+s+s2}{orange}\PY{l+s+s2}{\PYZdq{}}\PY{p}{,} \PY{n}{marker}\PY{o}{=}\PY{l+s+s2}{\PYZdq{}}\PY{l+s+s2}{D}\PY{l+s+s2}{\PYZdq{}}\PY{p}{,} \PY{n}{ms}\PY{o}{=}\PY{l+m+mi}{4}\PY{p}{)}
        
        \PY{n}{plt}\PY{o}{.}\PY{n}{show}\PY{p}{(}\PY{p}{)}
        
        \PY{c+c1}{\PYZsh{} una valida alternativa è il grafico a torta}
        \PY{c+c1}{\PYZsh{} una (meno) valida alternativa è un barplot semplice}
\end{Verbatim}


    \begin{center}
    \adjustimage{max size={0.9\linewidth}{0.9\paperheight}}{output_5_0.png}
    \end{center}
    { \hspace*{\fill} \\}
    
    \begin{enumerate}
\def\labelenumi{\arabic{enumi}.}
\setcounter{enumi}{3}
\tightlist
\item
  La Figura 1 mostra la funzione di ripartizione empirica per un
  sotto-insieme delle osservazioni relativi al carattere TotSpese, in
  cui gli importi sono indicati in centinaia dimigliaia di Euro.
  Leggendo esclusivamente il grafico:\\
  4.1. indicate quale sottoinsieme di osservazioni è stato utilizzato;
\end{enumerate}

    \begin{Verbatim}[commandchars=\\\{\}]
{\color{incolor}In [{\color{incolor}5}]:} \PY{k+kn}{import} \PY{n+nn}{numpy}
        
        \PY{c+c1}{\PYZsh{} calcoliamo la selezione dei dati che ci interessa \PYZsh{}\PYZsh{}\PYZsh{}}
        \PY{n}{subselection}  \PY{o}{=} \PY{n}{data}\PY{o}{.}\PY{n}{loc}\PY{p}{[}\PY{p}{(}\PY{n}{data}\PY{o}{.}\PY{n}{TotSpese}\PY{o}{.}\PY{n}{notna}\PY{p}{(}\PY{p}{)}\PY{p}{)} \PY{o}{\PYZam{}}
                        \PY{p}{(}\PY{n}{data}\PY{o}{.}\PY{n}{TotSpese}\PY{o}{\PYZgt{}}\PY{o}{=}\PY{l+m+mi}{5}\PY{o}{*}\PY{l+m+mi}{10}\PY{o}{*}\PY{o}{*}\PY{l+m+mi}{5}\PY{p}{)} \PY{o}{\PYZam{}} 
                       \PY{p}{(}\PY{n}{data}\PY{o}{.}\PY{n}{TotSpese}\PY{o}{\PYZlt{}}\PY{o}{=}\PY{l+m+mi}{2}\PY{o}{*}\PY{l+m+mi}{10}\PY{o}{*}\PY{o}{*}\PY{l+m+mi}{6}\PY{p}{)}\PY{p}{,}\PY{p}{]}\PY{o}{.}\PY{n}{TotSpese}
        
        \PY{c+c1}{\PYZsh{} normaliziamo la selezione, i valori sono espressi in }
        \PY{c+c1}{\PYZsh{}centinaia di migliaia}
        \PY{n}{normselection} \PY{o}{=} \PY{n}{numpy}\PY{o}{.}\PY{n}{floor}\PY{p}{(}\PY{n}{subselection}\PY{o}{/}\PY{l+m+mi}{10}\PY{o}{*}\PY{o}{*}\PY{l+m+mi}{5}\PY{p}{)}\PY{o}{.}\PY{n}{astype}\PY{p}{(}\PY{n+nb}{int}\PY{p}{)}
        
        \PY{c+c1}{\PYZsh{} contiamo le occorrenze}
        \PY{n}{freqselection} \PY{o}{=} \PY{n}{normselection}\PY{o}{.}\PY{n}{value\PYZus{}counts}\PY{p}{(}\PY{p}{)}
        
        \PY{c+c1}{\PYZsh{} ordiniamo}
        \PY{n}{freqs} \PY{o}{=} \PY{n}{freqselection}\PY{o}{.}\PY{n}{iloc}\PY{p}{[}\PY{n}{numpy}\PY{o}{.}\PY{n}{argsort}\PY{p}{(}\PY{n}{freqselection}\PY{o}{.}\PY{n}{keys}\PY{p}{(}\PY{p}{)}\PY{p}{)}\PY{p}{]}
        
        \PY{c+c1}{\PYZsh{} calcoliamo la cumulativa }
        \PY{n}{ecdf} \PY{o}{=} \PY{n}{freqs}\PY{o}{.}\PY{n}{cumsum}\PY{p}{(}\PY{p}{)}\PY{o}{/}\PY{n}{freqs}\PY{o}{.}\PY{n}{sum}\PY{p}{(}\PY{p}{)}
        
        
        \PY{c+c1}{\PYZsh{} aggiungiamo gli estremi al fine di avere un plot migliore}
        \PY{n}{x} \PY{o}{=} \PY{p}{[}\PY{l+m+mi}{0}\PY{p}{,}\PY{l+m+mi}{4}\PY{p}{]} \PY{o}{+} \PY{n+nb}{list}\PY{p}{(}\PY{n}{ecdf}\PY{o}{.}\PY{n}{keys}\PY{p}{(}\PY{p}{)}\PY{p}{)} \PY{o}{+} \PY{p}{[}\PY{l+m+mi}{40}\PY{p}{]}
        \PY{n}{y} \PY{o}{=} \PY{p}{[}\PY{l+m+mf}{0.0}\PY{p}{,}\PY{l+m+mf}{0.0}\PY{p}{]} \PY{o}{+} \PY{n+nb}{list}\PY{p}{(}\PY{n}{ecdf}\PY{o}{.}\PY{n}{values}\PY{p}{)} \PY{o}{+} \PY{p}{[}\PY{l+m+mf}{1.0}\PY{p}{]}
        
        \PY{n}{fig}\PY{p}{,} \PY{n}{ax} \PY{o}{=} \PY{n}{plt}\PY{o}{.}\PY{n}{subplots}\PY{p}{(}\PY{p}{)}
        \PY{n}{ax}\PY{o}{.}\PY{n}{step}\PY{p}{(}\PY{n}{x}\PY{p}{,}\PY{n}{y}\PY{p}{)}
        \PY{n}{ax}\PY{o}{.}\PY{n}{set\PYZus{}xlim}\PY{p}{(}\PY{p}{(}\PY{l+m+mi}{0}\PY{p}{,}\PY{l+m+mi}{40}\PY{p}{)}\PY{p}{)}
\end{Verbatim}


\begin{Verbatim}[commandchars=\\\{\}]
{\color{outcolor}Out[{\color{outcolor}5}]:} (0, 40)
\end{Verbatim}
            
    \begin{center}
    \adjustimage{max size={0.9\linewidth}{0.9\paperheight}}{output_7_1.png}
    \end{center}
    { \hspace*{\fill} \\}
    
    4.2. specificate quale percentuale delle osservazioni visualizzate
assume un valore compreso tra uno (escluso) e due milioni di Euro.

    \begin{Verbatim}[commandchars=\\\{\}]
{\color{incolor}In [{\color{incolor}6}]:} \PY{n+nb}{print}\PY{p}{(}\PY{p}{(}\PY{n}{ecdf}\PY{p}{[}\PY{n+nb}{min}\PY{p}{(}\PY{l+m+mi}{20}\PY{p}{,} \PY{n+nb}{max}\PY{p}{(}\PY{n}{ecdf}\PY{o}{.}\PY{n}{keys}\PY{p}{(}\PY{p}{)}\PY{p}{)}\PY{p}{)}\PY{p}{]}\PY{o}{\PYZhy{}}\PY{n}{ecdf}\PY{p}{[}\PY{l+m+mi}{10}\PY{p}{]}\PY{p}{)}\PY{p}{)}
        \PY{n+nb}{print}\PY{p}{(}\PY{n+nb}{len}\PY{p}{(}\PY{n}{data}\PY{o}{.}\PY{n}{loc}\PY{p}{[}\PY{p}{(}\PY{n}{data}\PY{o}{.}\PY{n}{TotSpese}\PY{o}{.}\PY{n}{notna}\PY{p}{(}\PY{p}{)}\PY{p}{)} \PY{o}{\PYZam{}}
                           \PY{p}{(}\PY{n}{data}\PY{o}{.}\PY{n}{TotSpese}\PY{o}{\PYZgt{}}\PY{o}{=}\PY{l+m+mi}{11}\PY{o}{*}\PY{l+m+mi}{10}\PY{o}{*}\PY{o}{*}\PY{l+m+mi}{5}\PY{p}{)} \PY{o}{\PYZam{}} 
                           \PY{p}{(}\PY{n}{data}\PY{o}{.}\PY{n}{TotSpese}\PY{o}{\PYZlt{}}\PY{o}{=}\PY{l+m+mi}{2}\PY{o}{*}\PY{l+m+mi}{10}\PY{o}{*}\PY{o}{*}\PY{l+m+mi}{6}\PY{p}{)}\PY{p}{,}\PY{p}{]}\PY{p}{)}\PY{o}{/}\PY{l+m+mi}{110}\PY{p}{)}
\end{Verbatim}


    \begin{Verbatim}[commandchars=\\\{\}]
0.21818181818181814
0.21818181818181817

    \end{Verbatim}

    \begin{enumerate}
\def\labelenumi{\arabic{enumi}.}
\setcounter{enumi}{4}
\tightlist
\item
  Prendiamo in considerazione la quota di finanziamento erogata dalla
  Provincia.\\
  5.1. Create una variabile (chiamatela progetti\_a, per indicare i
  progetti di tipo A) che contenga la parte di dataset relativa ai
  progetti per i quali la quota provinciale di finanziamento è minore di
  quella regionale, e un'altra (chiamata progetti\_b, per indicare i
  progetti di tipo B) che contenga la parte di dataset relativa ai
  progetti per i quali la quota provinciale di finanziamento è maggiore
  di quella regionale.
\end{enumerate}

    \begin{Verbatim}[commandchars=\\\{\}]
{\color{incolor}In [{\color{incolor}7}]:} \PY{n}{progetti\PYZus{}a} \PY{o}{=} \PY{n}{data}\PY{o}{.}\PY{n}{loc}\PY{p}{[}\PY{p}{(}\PY{n}{data}\PY{o}{.}\PY{n}{FinProvincia}\PY{o}{.}\PY{n}{notna}\PY{p}{(}\PY{p}{)}\PY{p}{)} \PY{o}{\PYZam{}} 
                              \PY{p}{(}\PY{n}{data}\PY{o}{.}\PY{n}{FinRegione}\PY{o}{.}\PY{n}{notna}\PY{p}{(}\PY{p}{)}\PY{p}{)} \PY{o}{\PYZam{}} 
                              \PY{p}{(}\PY{n}{data}\PY{o}{.}\PY{n}{FinProvincia} \PY{o}{\PYZlt{}} \PY{n}{data}\PY{o}{.}\PY{n}{FinRegione}\PY{p}{)}\PY{p}{,}\PY{p}{]}
        \PY{n}{progetti\PYZus{}b} \PY{o}{=} \PY{n}{data}\PY{o}{.}\PY{n}{loc}\PY{p}{[}\PY{p}{(}\PY{n}{data}\PY{o}{.}\PY{n}{FinProvincia}\PY{o}{.}\PY{n}{notna}\PY{p}{(}\PY{p}{)}\PY{p}{)} \PY{o}{\PYZam{}} 
                              \PY{p}{(}\PY{n}{data}\PY{o}{.}\PY{n}{FinRegione}\PY{o}{.}\PY{n}{notna}\PY{p}{(}\PY{p}{)}\PY{p}{)} \PY{o}{\PYZam{}} 
                              \PY{p}{(}\PY{n}{data}\PY{o}{.}\PY{n}{FinProvincia} \PY{o}{\PYZgt{}} \PY{n}{data}\PY{o}{.}\PY{n}{FinRegione}\PY{p}{)}\PY{p}{,}\PY{p}{]}
\end{Verbatim}


    5.2. Quanti sono progetti di tipo A? Quanti sono progetti di tipo B?

    \begin{Verbatim}[commandchars=\\\{\}]
{\color{incolor}In [{\color{incolor}7}]:} \PY{n+nb}{print}\PY{p}{(}\PY{n}{progetti\PYZus{}a}\PY{o}{.}\PY{n+nf+fm}{\PYZus{}\PYZus{}len\PYZus{}\PYZus{}}\PY{p}{(}\PY{p}{)}\PY{p}{)}
        \PY{n+nb}{print}\PY{p}{(}\PY{n}{progetti\PYZus{}b}\PY{o}{.}\PY{n+nf+fm}{\PYZus{}\PYZus{}len\PYZus{}\PYZus{}}\PY{p}{(}\PY{p}{)}\PY{p}{)}
\end{Verbatim}


    \begin{Verbatim}[commandchars=\\\{\}]
368
175

    \end{Verbatim}

    \begin{enumerate}
\def\labelenumi{\arabic{enumi}.}
\setcounter{enumi}{5}
\tightlist
\item
  Concentriamoci sui progetti di tipo A.\\
  6.1. Selezionate i progetti di tipo A che hanno ricevuto un
  finanziamento provinciale compreso tra i 200 e i 1000 euro, estremo
  sinistro incluso, e salvate questa parte di dataset in una variabile
  chiamata selezione\_progetti\_a.
\end{enumerate}

    \begin{Verbatim}[commandchars=\\\{\}]
{\color{incolor}In [{\color{incolor}10}]:} \PY{n}{selezione\PYZus{}progetti\PYZus{}a} \PY{o}{=} \PY{n}{progetti\PYZus{}a}\PY{o}{.}\PY{n}{loc}\PY{p}{[}\PY{p}{(}\PY{n}{progetti\PYZus{}a}\PY{o}{.}\PY{n}{FinProvincia}\PY{o}{\PYZgt{}}\PY{o}{=}\PY{l+m+mi}{200}\PY{p}{)} \PY{o}{\PYZam{}} 
                                               \PY{p}{(}\PY{n}{progetti\PYZus{}a}\PY{o}{.}\PY{n}{FinProvincia}\PY{o}{\PYZlt{}}\PY{l+m+mi}{1000}\PY{p}{)}\PY{p}{,}\PY{p}{]}
\end{Verbatim}


    \begin{Verbatim}[commandchars=\\\{\}]
{\color{incolor}In [{\color{incolor}9}]:} \PY{n}{progetti\PYZus{}a}\PY{o}{=}\PY{n}{data}\PY{p}{[}\PY{p}{(}\PY{n}{data}\PY{p}{[}\PY{l+s+s1}{\PYZsq{}}\PY{l+s+s1}{FinProvincia}\PY{l+s+s1}{\PYZsq{}}\PY{p}{]}\PY{o}{\PYZlt{}}\PY{n}{data}\PY{p}{[}\PY{l+s+s1}{\PYZsq{}}\PY{l+s+s1}{FinRegione}\PY{l+s+s1}{\PYZsq{}}\PY{p}{]}\PY{p}{)} \PY{o}{\PYZam{}} 
                        \PY{p}{(}\PY{n}{data}\PY{o}{.}\PY{n}{FinRegione}\PY{o}{.}\PY{n}{notna}\PY{p}{(}\PY{p}{)}\PY{p}{)} \PY{o}{\PYZam{}} 
                        \PY{p}{(}\PY{n}{data}\PY{o}{.}\PY{n}{FinProvincia}\PY{o}{.}\PY{n}{notna}\PY{p}{(}\PY{p}{)}\PY{p}{)}\PY{p}{]}
        \PY{n}{selezione\PYZus{}progetti\PYZus{}a}\PY{o}{=}\PY{n}{progetti\PYZus{}a}\PY{p}{[}\PY{p}{(}\PY{n}{progetti\PYZus{}a}\PY{p}{[}\PY{l+s+s1}{\PYZsq{}}\PY{l+s+s1}{FinProvincia}\PY{l+s+s1}{\PYZsq{}}\PY{p}{]}\PY{o}{\PYZgt{}}\PY{o}{=}\PY{l+m+mi}{200}\PY{p}{)} \PY{o}{\PYZam{}}
                                        \PY{p}{(}\PY{n}{progetti\PYZus{}a}\PY{p}{[}\PY{l+s+s1}{\PYZsq{}}\PY{l+s+s1}{FinProvincia}\PY{l+s+s1}{\PYZsq{}}\PY{p}{]}\PY{o}{\PYZlt{}}\PY{l+m+mi}{1000}\PY{p}{)}\PY{p}{]}
        \PY{n}{selezione\PYZus{}progetti\PYZus{}a}\PY{o}{.}\PY{n}{FinProvincia}\PY{o}{.}\PY{n}{hist}\PY{p}{(}\PY{n}{bins}\PY{o}{=}\PY{n}{numpy}\PY{o}{.}\PY{n}{arange}\PY{p}{(}\PY{l+m+mi}{200}\PY{p}{,}\PY{l+m+mi}{1000}\PY{p}{,}\PY{l+m+mi}{100}\PY{p}{)}\PY{p}{)}
        \PY{n}{selezione\PYZus{}progetti\PYZus{}a}\PY{o}{.}\PY{n}{mean}\PY{p}{(}\PY{p}{)}
\end{Verbatim}


\begin{Verbatim}[commandchars=\\\{\}]
{\color{outcolor}Out[{\color{outcolor}9}]:} id                 2907.526316
        TemaPrioritario      69.736842
        CodiceCategoria      41.421053
        FinProvincia        636.905263
        FinRegione          702.587895
        TotSpese           3548.256250
        dtype: float64
\end{Verbatim}
            
    \begin{center}
    \adjustimage{max size={0.9\linewidth}{0.9\paperheight}}{output_16_1.png}
    \end{center}
    { \hspace*{\fill} \\}
    
    6.2. Tracciate un istogramma del finanziamento provinciale di tali
progetti, imponendo che le classi abbiano ampiezza 100 euro

    \begin{Verbatim}[commandchars=\\\{\}]
{\color{incolor}In [{\color{incolor}12}]:} \PY{n}{p} \PY{o}{=} \PY{n}{selezione\PYZus{}progetti\PYZus{}a}\PY{o}{.}\PY{n}{FinProvincia}\PY{o}{.}\PY{n}{hist}\PY{p}{(}\PY{n}{bins}\PY{o}{=}\PY{n}{numpy}\PY{o}{.}\PY{n}{arange}\PY{p}{(}\PY{l+m+mi}{200}\PY{p}{,}\PY{l+m+mi}{1000}\PY{p}{,}\PY{l+m+mi}{100}\PY{p}{)}\PY{p}{)}
\end{Verbatim}


    \begin{center}
    \adjustimage{max size={0.9\linewidth}{0.9\paperheight}}{output_18_0.png}
    \end{center}
    { \hspace*{\fill} \\}
    
    6.3. Tracciate anche il boxplot per la medesima quantità.

    \begin{Verbatim}[commandchars=\\\{\}]
{\color{incolor}In [{\color{incolor}10}]:} \PY{n}{p} \PY{o}{=} \PY{n}{selezione\PYZus{}progetti\PYZus{}a}\PY{o}{.}\PY{n}{FinProvincia}\PY{o}{.}\PY{n}{plot}\PY{o}{.}\PY{n}{box}\PY{p}{(}\PY{p}{)}
\end{Verbatim}


    \begin{center}
    \adjustimage{max size={0.9\linewidth}{0.9\paperheight}}{output_20_0.png}
    \end{center}
    { \hspace*{\fill} \\}
    
    6.4. Tra i due grafici appena prodotti, quale ritenete maggiormente
informativo? Giusti-ficate la risposta.

\begin{verbatim}
    Sicuramente l'istogramma in quanto riesce ad rappresentare meglio la distribuzione evidenziando la bimodalità.
\end{verbatim}

6.5. Relativamente a tali progetti, qual è stato l'importo medio
finanziato dalla provincia? Quale la deviazione standard?

    \begin{Verbatim}[commandchars=\\\{\}]
{\color{incolor}In [{\color{incolor}11}]:} \PY{n+nb}{print}\PY{p}{(}\PY{n}{selezione\PYZus{}progetti\PYZus{}a}\PY{o}{.}\PY{n}{FinProvincia}\PY{o}{.}\PY{n}{mean}\PY{p}{(}\PY{p}{)}\PY{p}{)}
         \PY{n+nb}{print}\PY{p}{(}\PY{n}{selezione\PYZus{}progetti\PYZus{}a}\PY{o}{.}\PY{n}{FinProvincia}\PY{o}{.}\PY{n}{std}\PY{p}{(}\PY{p}{)}\PY{p}{)}
\end{Verbatim}


    \begin{Verbatim}[commandchars=\\\{\}]
636.9052631578948
264.80233322588253

    \end{Verbatim}

    6.6. Quanti hanno ricevuto un finanziamento provinciale compreso tra i
500 e i 700 euro?

    \begin{Verbatim}[commandchars=\\\{\}]
{\color{incolor}In [{\color{incolor}13}]:} \PY{n+nb}{len}\PY{p}{(}\PY{n}{selezione\PYZus{}progetti\PYZus{}a}\PY{o}{.}\PY{n}{loc}\PY{p}{[}\PY{p}{(}\PY{n}{selezione\PYZus{}progetti\PYZus{}a}\PY{o}{.}\PY{n}{FinProvincia} \PY{o}{\PYZgt{}}\PY{o}{=} \PY{l+m+mi}{500}\PY{p}{)} \PY{o}{\PYZam{}} 
                                      \PY{p}{(}\PY{n}{selezione\PYZus{}progetti\PYZus{}a}\PY{o}{.}\PY{n}{FinProvincia} \PY{o}{\PYZlt{}}\PY{o}{=} \PY{l+m+mi}{700}\PY{p}{)}\PY{p}{,}\PY{p}{]}\PY{p}{)}
         \PY{c+c1}{\PYZsh{} oppure lo si può anche vedere dal rispettivo istogramma}
\end{Verbatim}


\begin{Verbatim}[commandchars=\\\{\}]
{\color{outcolor}Out[{\color{outcolor}13}]:} 0
\end{Verbatim}
            
    6.7. Esiste una evidente relazione tra finanziamento provinciale e spese
sostenute. Descrivete tale relazione nel modo più dettagliato possibile,
utilizzando un indice numerico e un metodo grafico.

    \begin{Verbatim}[commandchars=\\\{\}]
{\color{incolor}In [{\color{incolor}13}]:} \PY{n}{p} \PY{o}{=} \PY{n}{selezione\PYZus{}progetti\PYZus{}a}\PY{o}{.}\PY{n}{plot}\PY{o}{.}\PY{n}{scatter}\PY{p}{(}\PY{l+s+s2}{\PYZdq{}}\PY{l+s+s2}{FinProvincia}\PY{l+s+s2}{\PYZdq{}}\PY{p}{,}\PY{l+s+s2}{\PYZdq{}}\PY{l+s+s2}{TotSpese}\PY{l+s+s2}{\PYZdq{}}\PY{p}{)}
         \PY{n}{p}\PY{o}{.}\PY{n}{set\PYZus{}xlabel}\PY{p}{(}\PY{l+s+s2}{\PYZdq{}}\PY{l+s+s2}{Finanziamento Provinciale}\PY{l+s+s2}{\PYZdq{}}\PY{p}{)}
         \PY{n}{p}\PY{o}{.}\PY{n}{set\PYZus{}ylabel}\PY{p}{(}\PY{l+s+s2}{\PYZdq{}}\PY{l+s+s2}{Spese Sostenute}\PY{l+s+s2}{\PYZdq{}}\PY{p}{)}
         
         \PY{n}{selezione\PYZus{}progetti\PYZus{}a}\PY{o}{.}\PY{n}{loc}\PY{p}{[}\PY{p}{:}\PY{p}{,}\PY{p}{[}\PY{l+s+s2}{\PYZdq{}}\PY{l+s+s2}{FinProvincia}\PY{l+s+s2}{\PYZdq{}}\PY{p}{,}\PY{l+s+s2}{\PYZdq{}}\PY{l+s+s2}{TotSpese}\PY{l+s+s2}{\PYZdq{}}\PY{p}{]}\PY{p}{]}\PY{o}{.}\PY{n}{corr}\PY{p}{(}\PY{p}{)}
\end{Verbatim}


\begin{Verbatim}[commandchars=\\\{\}]
{\color{outcolor}Out[{\color{outcolor}13}]:}               FinProvincia  TotSpese
         FinProvincia      1.000000  0.696401
         TotSpese          0.696401  1.000000
\end{Verbatim}
            
    \begin{center}
    \adjustimage{max size={0.9\linewidth}{0.9\paperheight}}{output_26_1.png}
    \end{center}
    { \hspace*{\fill} \\}
    
    C'è una relazione lineare tra il finanziamento provinciale e le spese
totali ad eccezione di un outlier che abbassa il coefficiente di
correlazione.

6.8. Nella relazione avrete notato la presenza di almeno un progetto che
si discosta note-volmente dall'andamento più generale. Eliminate tali
progetti dall'insieme dei dati e rispondete nuovamente alle domande del
punto precedente.

\begin{verbatim}
    vediamo che l'outlier appare avere TotSpese 2000.0 eliminiamo dunque tutti i punti che hanno TotSpese inferiori a questa quantità.
\end{verbatim}

    \begin{Verbatim}[commandchars=\\\{\}]
{\color{incolor}In [{\color{incolor}14}]:} \PY{n}{selezione\PYZus{}progetti\PYZus{}a}\PY{o}{.}\PY{n}{loc}\PY{p}{[}\PY{p}{(}\PY{n}{selezione\PYZus{}progetti\PYZus{}a}\PY{o}{.}\PY{n}{TotSpese}\PY{o}{.}\PY{n}{notna}\PY{p}{(}\PY{p}{)}\PY{p}{)} \PY{o}{\PYZam{}}
                                  \PY{p}{(}\PY{n}{selezione\PYZus{}progetti\PYZus{}a}\PY{o}{.}\PY{n}{TotSpese} \PY{o}{\PYZgt{}} \PY{l+m+mi}{2000}\PY{p}{)} \PY{p}{,}
                                  \PY{p}{[}\PY{l+s+s2}{\PYZdq{}}\PY{l+s+s2}{FinProvincia}\PY{l+s+s2}{\PYZdq{}}\PY{p}{,}\PY{l+s+s2}{\PYZdq{}}\PY{l+s+s2}{TotSpese}\PY{l+s+s2}{\PYZdq{}}\PY{p}{]}\PY{p}{]}\PY{o}{.}\PY{n}{corr}\PY{p}{(}\PY{p}{)}
\end{Verbatim}


\begin{Verbatim}[commandchars=\\\{\}]
{\color{outcolor}Out[{\color{outcolor}14}]:}               FinProvincia  TotSpese
         FinProvincia           1.0       1.0
         TotSpese               1.0       1.0
\end{Verbatim}
            
    ora la relazione appare perfettamente lineare.

.7. Ritorniamo al dataset completo. Quanti sono i progetti che non hanno
ancora sostenuto spese?

    \begin{Verbatim}[commandchars=\\\{\}]
{\color{incolor}In [{\color{incolor}15}]:} \PY{n+nb}{len}\PY{p}{(}\PY{n}{data}\PY{o}{.}\PY{n}{loc}\PY{p}{[}\PY{p}{(}\PY{n}{data}\PY{o}{.}\PY{n}{TotSpese} \PY{o}{==} \PY{l+m+mi}{0}\PY{p}{)} \PY{o}{|} \PY{p}{(}\PY{n}{data}\PY{o}{.}\PY{n}{TotSpese}\PY{o}{.}\PY{n}{isna}\PY{p}{(}\PY{p}{)}\PY{p}{)}\PY{p}{,} \PY{p}{]}\PY{p}{)}
\end{Verbatim}


\begin{Verbatim}[commandchars=\\\{\}]
{\color{outcolor}Out[{\color{outcolor}15}]:} 1134
\end{Verbatim}
            

    % Add a bibliography block to the postdoc
    
    
    
    \end{document}
