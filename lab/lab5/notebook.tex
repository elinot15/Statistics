
% Default to the notebook output style

    


% Inherit from the specified cell style.




    
\documentclass[11pt]{article}

    
    
    \usepackage[T1]{fontenc}
    % Nicer default font (+ math font) than Computer Modern for most use cases
    \usepackage{mathpazo}

    % Basic figure setup, for now with no caption control since it's done
    % automatically by Pandoc (which extracts ![](path) syntax from Markdown).
    \usepackage{graphicx}
    % We will generate all images so they have a width \maxwidth. This means
    % that they will get their normal width if they fit onto the page, but
    % are scaled down if they would overflow the margins.
    \makeatletter
    \def\maxwidth{\ifdim\Gin@nat@width>\linewidth\linewidth
    \else\Gin@nat@width\fi}
    \makeatother
    \let\Oldincludegraphics\includegraphics
    % Set max figure width to be 80% of text width, for now hardcoded.
    \renewcommand{\includegraphics}[1]{\Oldincludegraphics[width=.8\maxwidth]{#1}}
    % Ensure that by default, figures have no caption (until we provide a
    % proper Figure object with a Caption API and a way to capture that
    % in the conversion process - todo).
    \usepackage{caption}
    \DeclareCaptionLabelFormat{nolabel}{}
    \captionsetup{labelformat=nolabel}

    \usepackage{adjustbox} % Used to constrain images to a maximum size 
    \usepackage{xcolor} % Allow colors to be defined
    \usepackage{enumerate} % Needed for markdown enumerations to work
    \usepackage{geometry} % Used to adjust the document margins
    \usepackage{amsmath} % Equations
    \usepackage{amssymb} % Equations
    \usepackage{textcomp} % defines textquotesingle
    % Hack from http://tex.stackexchange.com/a/47451/13684:
    \AtBeginDocument{%
        \def\PYZsq{\textquotesingle}% Upright quotes in Pygmentized code
    }
    \usepackage{upquote} % Upright quotes for verbatim code
    \usepackage{eurosym} % defines \euro
    \usepackage[mathletters]{ucs} % Extended unicode (utf-8) support
    \usepackage[utf8x]{inputenc} % Allow utf-8 characters in the tex document
    \usepackage{fancyvrb} % verbatim replacement that allows latex
    \usepackage{grffile} % extends the file name processing of package graphics 
                         % to support a larger range 
    % The hyperref package gives us a pdf with properly built
    % internal navigation ('pdf bookmarks' for the table of contents,
    % internal cross-reference links, web links for URLs, etc.)
    \usepackage{hyperref}
    \usepackage{longtable} % longtable support required by pandoc >1.10
    \usepackage{booktabs}  % table support for pandoc > 1.12.2
    \usepackage[inline]{enumitem} % IRkernel/repr support (it uses the enumerate* environment)
    \usepackage[normalem]{ulem} % ulem is needed to support strikethroughs (\sout)
                                % normalem makes italics be italics, not underlines
    

    
    
    % Colors for the hyperref package
    \definecolor{urlcolor}{rgb}{0,.145,.698}
    \definecolor{linkcolor}{rgb}{.71,0.21,0.01}
    \definecolor{citecolor}{rgb}{.12,.54,.11}

    % ANSI colors
    \definecolor{ansi-black}{HTML}{3E424D}
    \definecolor{ansi-black-intense}{HTML}{282C36}
    \definecolor{ansi-red}{HTML}{E75C58}
    \definecolor{ansi-red-intense}{HTML}{B22B31}
    \definecolor{ansi-green}{HTML}{00A250}
    \definecolor{ansi-green-intense}{HTML}{007427}
    \definecolor{ansi-yellow}{HTML}{DDB62B}
    \definecolor{ansi-yellow-intense}{HTML}{B27D12}
    \definecolor{ansi-blue}{HTML}{208FFB}
    \definecolor{ansi-blue-intense}{HTML}{0065CA}
    \definecolor{ansi-magenta}{HTML}{D160C4}
    \definecolor{ansi-magenta-intense}{HTML}{A03196}
    \definecolor{ansi-cyan}{HTML}{60C6C8}
    \definecolor{ansi-cyan-intense}{HTML}{258F8F}
    \definecolor{ansi-white}{HTML}{C5C1B4}
    \definecolor{ansi-white-intense}{HTML}{A1A6B2}

    % commands and environments needed by pandoc snippets
    % extracted from the output of `pandoc -s`
    \providecommand{\tightlist}{%
      \setlength{\itemsep}{0pt}\setlength{\parskip}{0pt}}
    \DefineVerbatimEnvironment{Highlighting}{Verbatim}{commandchars=\\\{\}}
    % Add ',fontsize=\small' for more characters per line
    \newenvironment{Shaded}{}{}
    \newcommand{\KeywordTok}[1]{\textcolor[rgb]{0.00,0.44,0.13}{\textbf{{#1}}}}
    \newcommand{\DataTypeTok}[1]{\textcolor[rgb]{0.56,0.13,0.00}{{#1}}}
    \newcommand{\DecValTok}[1]{\textcolor[rgb]{0.25,0.63,0.44}{{#1}}}
    \newcommand{\BaseNTok}[1]{\textcolor[rgb]{0.25,0.63,0.44}{{#1}}}
    \newcommand{\FloatTok}[1]{\textcolor[rgb]{0.25,0.63,0.44}{{#1}}}
    \newcommand{\CharTok}[1]{\textcolor[rgb]{0.25,0.44,0.63}{{#1}}}
    \newcommand{\StringTok}[1]{\textcolor[rgb]{0.25,0.44,0.63}{{#1}}}
    \newcommand{\CommentTok}[1]{\textcolor[rgb]{0.38,0.63,0.69}{\textit{{#1}}}}
    \newcommand{\OtherTok}[1]{\textcolor[rgb]{0.00,0.44,0.13}{{#1}}}
    \newcommand{\AlertTok}[1]{\textcolor[rgb]{1.00,0.00,0.00}{\textbf{{#1}}}}
    \newcommand{\FunctionTok}[1]{\textcolor[rgb]{0.02,0.16,0.49}{{#1}}}
    \newcommand{\RegionMarkerTok}[1]{{#1}}
    \newcommand{\ErrorTok}[1]{\textcolor[rgb]{1.00,0.00,0.00}{\textbf{{#1}}}}
    \newcommand{\NormalTok}[1]{{#1}}
    
    % Additional commands for more recent versions of Pandoc
    \newcommand{\ConstantTok}[1]{\textcolor[rgb]{0.53,0.00,0.00}{{#1}}}
    \newcommand{\SpecialCharTok}[1]{\textcolor[rgb]{0.25,0.44,0.63}{{#1}}}
    \newcommand{\VerbatimStringTok}[1]{\textcolor[rgb]{0.25,0.44,0.63}{{#1}}}
    \newcommand{\SpecialStringTok}[1]{\textcolor[rgb]{0.73,0.40,0.53}{{#1}}}
    \newcommand{\ImportTok}[1]{{#1}}
    \newcommand{\DocumentationTok}[1]{\textcolor[rgb]{0.73,0.13,0.13}{\textit{{#1}}}}
    \newcommand{\AnnotationTok}[1]{\textcolor[rgb]{0.38,0.63,0.69}{\textbf{\textit{{#1}}}}}
    \newcommand{\CommentVarTok}[1]{\textcolor[rgb]{0.38,0.63,0.69}{\textbf{\textit{{#1}}}}}
    \newcommand{\VariableTok}[1]{\textcolor[rgb]{0.10,0.09,0.49}{{#1}}}
    \newcommand{\ControlFlowTok}[1]{\textcolor[rgb]{0.00,0.44,0.13}{\textbf{{#1}}}}
    \newcommand{\OperatorTok}[1]{\textcolor[rgb]{0.40,0.40,0.40}{{#1}}}
    \newcommand{\BuiltInTok}[1]{{#1}}
    \newcommand{\ExtensionTok}[1]{{#1}}
    \newcommand{\PreprocessorTok}[1]{\textcolor[rgb]{0.74,0.48,0.00}{{#1}}}
    \newcommand{\AttributeTok}[1]{\textcolor[rgb]{0.49,0.56,0.16}{{#1}}}
    \newcommand{\InformationTok}[1]{\textcolor[rgb]{0.38,0.63,0.69}{\textbf{\textit{{#1}}}}}
    \newcommand{\WarningTok}[1]{\textcolor[rgb]{0.38,0.63,0.69}{\textbf{\textit{{#1}}}}}
    
    
    % Define a nice break command that doesn't care if a line doesn't already
    % exist.
    \def\br{\hspace*{\fill} \\* }
    % Math Jax compatability definitions
    \def\gt{>}
    \def\lt{<}
    % Document parameters
    \title{binomial\_distribution\_python}
    
    
    

    % Pygments definitions
    
\makeatletter
\def\PY@reset{\let\PY@it=\relax \let\PY@bf=\relax%
    \let\PY@ul=\relax \let\PY@tc=\relax%
    \let\PY@bc=\relax \let\PY@ff=\relax}
\def\PY@tok#1{\csname PY@tok@#1\endcsname}
\def\PY@toks#1+{\ifx\relax#1\empty\else%
    \PY@tok{#1}\expandafter\PY@toks\fi}
\def\PY@do#1{\PY@bc{\PY@tc{\PY@ul{%
    \PY@it{\PY@bf{\PY@ff{#1}}}}}}}
\def\PY#1#2{\PY@reset\PY@toks#1+\relax+\PY@do{#2}}

\expandafter\def\csname PY@tok@w\endcsname{\def\PY@tc##1{\textcolor[rgb]{0.73,0.73,0.73}{##1}}}
\expandafter\def\csname PY@tok@c\endcsname{\let\PY@it=\textit\def\PY@tc##1{\textcolor[rgb]{0.25,0.50,0.50}{##1}}}
\expandafter\def\csname PY@tok@cp\endcsname{\def\PY@tc##1{\textcolor[rgb]{0.74,0.48,0.00}{##1}}}
\expandafter\def\csname PY@tok@k\endcsname{\let\PY@bf=\textbf\def\PY@tc##1{\textcolor[rgb]{0.00,0.50,0.00}{##1}}}
\expandafter\def\csname PY@tok@kp\endcsname{\def\PY@tc##1{\textcolor[rgb]{0.00,0.50,0.00}{##1}}}
\expandafter\def\csname PY@tok@kt\endcsname{\def\PY@tc##1{\textcolor[rgb]{0.69,0.00,0.25}{##1}}}
\expandafter\def\csname PY@tok@o\endcsname{\def\PY@tc##1{\textcolor[rgb]{0.40,0.40,0.40}{##1}}}
\expandafter\def\csname PY@tok@ow\endcsname{\let\PY@bf=\textbf\def\PY@tc##1{\textcolor[rgb]{0.67,0.13,1.00}{##1}}}
\expandafter\def\csname PY@tok@nb\endcsname{\def\PY@tc##1{\textcolor[rgb]{0.00,0.50,0.00}{##1}}}
\expandafter\def\csname PY@tok@nf\endcsname{\def\PY@tc##1{\textcolor[rgb]{0.00,0.00,1.00}{##1}}}
\expandafter\def\csname PY@tok@nc\endcsname{\let\PY@bf=\textbf\def\PY@tc##1{\textcolor[rgb]{0.00,0.00,1.00}{##1}}}
\expandafter\def\csname PY@tok@nn\endcsname{\let\PY@bf=\textbf\def\PY@tc##1{\textcolor[rgb]{0.00,0.00,1.00}{##1}}}
\expandafter\def\csname PY@tok@ne\endcsname{\let\PY@bf=\textbf\def\PY@tc##1{\textcolor[rgb]{0.82,0.25,0.23}{##1}}}
\expandafter\def\csname PY@tok@nv\endcsname{\def\PY@tc##1{\textcolor[rgb]{0.10,0.09,0.49}{##1}}}
\expandafter\def\csname PY@tok@no\endcsname{\def\PY@tc##1{\textcolor[rgb]{0.53,0.00,0.00}{##1}}}
\expandafter\def\csname PY@tok@nl\endcsname{\def\PY@tc##1{\textcolor[rgb]{0.63,0.63,0.00}{##1}}}
\expandafter\def\csname PY@tok@ni\endcsname{\let\PY@bf=\textbf\def\PY@tc##1{\textcolor[rgb]{0.60,0.60,0.60}{##1}}}
\expandafter\def\csname PY@tok@na\endcsname{\def\PY@tc##1{\textcolor[rgb]{0.49,0.56,0.16}{##1}}}
\expandafter\def\csname PY@tok@nt\endcsname{\let\PY@bf=\textbf\def\PY@tc##1{\textcolor[rgb]{0.00,0.50,0.00}{##1}}}
\expandafter\def\csname PY@tok@nd\endcsname{\def\PY@tc##1{\textcolor[rgb]{0.67,0.13,1.00}{##1}}}
\expandafter\def\csname PY@tok@s\endcsname{\def\PY@tc##1{\textcolor[rgb]{0.73,0.13,0.13}{##1}}}
\expandafter\def\csname PY@tok@sd\endcsname{\let\PY@it=\textit\def\PY@tc##1{\textcolor[rgb]{0.73,0.13,0.13}{##1}}}
\expandafter\def\csname PY@tok@si\endcsname{\let\PY@bf=\textbf\def\PY@tc##1{\textcolor[rgb]{0.73,0.40,0.53}{##1}}}
\expandafter\def\csname PY@tok@se\endcsname{\let\PY@bf=\textbf\def\PY@tc##1{\textcolor[rgb]{0.73,0.40,0.13}{##1}}}
\expandafter\def\csname PY@tok@sr\endcsname{\def\PY@tc##1{\textcolor[rgb]{0.73,0.40,0.53}{##1}}}
\expandafter\def\csname PY@tok@ss\endcsname{\def\PY@tc##1{\textcolor[rgb]{0.10,0.09,0.49}{##1}}}
\expandafter\def\csname PY@tok@sx\endcsname{\def\PY@tc##1{\textcolor[rgb]{0.00,0.50,0.00}{##1}}}
\expandafter\def\csname PY@tok@m\endcsname{\def\PY@tc##1{\textcolor[rgb]{0.40,0.40,0.40}{##1}}}
\expandafter\def\csname PY@tok@gh\endcsname{\let\PY@bf=\textbf\def\PY@tc##1{\textcolor[rgb]{0.00,0.00,0.50}{##1}}}
\expandafter\def\csname PY@tok@gu\endcsname{\let\PY@bf=\textbf\def\PY@tc##1{\textcolor[rgb]{0.50,0.00,0.50}{##1}}}
\expandafter\def\csname PY@tok@gd\endcsname{\def\PY@tc##1{\textcolor[rgb]{0.63,0.00,0.00}{##1}}}
\expandafter\def\csname PY@tok@gi\endcsname{\def\PY@tc##1{\textcolor[rgb]{0.00,0.63,0.00}{##1}}}
\expandafter\def\csname PY@tok@gr\endcsname{\def\PY@tc##1{\textcolor[rgb]{1.00,0.00,0.00}{##1}}}
\expandafter\def\csname PY@tok@ge\endcsname{\let\PY@it=\textit}
\expandafter\def\csname PY@tok@gs\endcsname{\let\PY@bf=\textbf}
\expandafter\def\csname PY@tok@gp\endcsname{\let\PY@bf=\textbf\def\PY@tc##1{\textcolor[rgb]{0.00,0.00,0.50}{##1}}}
\expandafter\def\csname PY@tok@go\endcsname{\def\PY@tc##1{\textcolor[rgb]{0.53,0.53,0.53}{##1}}}
\expandafter\def\csname PY@tok@gt\endcsname{\def\PY@tc##1{\textcolor[rgb]{0.00,0.27,0.87}{##1}}}
\expandafter\def\csname PY@tok@err\endcsname{\def\PY@bc##1{\setlength{\fboxsep}{0pt}\fcolorbox[rgb]{1.00,0.00,0.00}{1,1,1}{\strut ##1}}}
\expandafter\def\csname PY@tok@kc\endcsname{\let\PY@bf=\textbf\def\PY@tc##1{\textcolor[rgb]{0.00,0.50,0.00}{##1}}}
\expandafter\def\csname PY@tok@kd\endcsname{\let\PY@bf=\textbf\def\PY@tc##1{\textcolor[rgb]{0.00,0.50,0.00}{##1}}}
\expandafter\def\csname PY@tok@kn\endcsname{\let\PY@bf=\textbf\def\PY@tc##1{\textcolor[rgb]{0.00,0.50,0.00}{##1}}}
\expandafter\def\csname PY@tok@kr\endcsname{\let\PY@bf=\textbf\def\PY@tc##1{\textcolor[rgb]{0.00,0.50,0.00}{##1}}}
\expandafter\def\csname PY@tok@bp\endcsname{\def\PY@tc##1{\textcolor[rgb]{0.00,0.50,0.00}{##1}}}
\expandafter\def\csname PY@tok@fm\endcsname{\def\PY@tc##1{\textcolor[rgb]{0.00,0.00,1.00}{##1}}}
\expandafter\def\csname PY@tok@vc\endcsname{\def\PY@tc##1{\textcolor[rgb]{0.10,0.09,0.49}{##1}}}
\expandafter\def\csname PY@tok@vg\endcsname{\def\PY@tc##1{\textcolor[rgb]{0.10,0.09,0.49}{##1}}}
\expandafter\def\csname PY@tok@vi\endcsname{\def\PY@tc##1{\textcolor[rgb]{0.10,0.09,0.49}{##1}}}
\expandafter\def\csname PY@tok@vm\endcsname{\def\PY@tc##1{\textcolor[rgb]{0.10,0.09,0.49}{##1}}}
\expandafter\def\csname PY@tok@sa\endcsname{\def\PY@tc##1{\textcolor[rgb]{0.73,0.13,0.13}{##1}}}
\expandafter\def\csname PY@tok@sb\endcsname{\def\PY@tc##1{\textcolor[rgb]{0.73,0.13,0.13}{##1}}}
\expandafter\def\csname PY@tok@sc\endcsname{\def\PY@tc##1{\textcolor[rgb]{0.73,0.13,0.13}{##1}}}
\expandafter\def\csname PY@tok@dl\endcsname{\def\PY@tc##1{\textcolor[rgb]{0.73,0.13,0.13}{##1}}}
\expandafter\def\csname PY@tok@s2\endcsname{\def\PY@tc##1{\textcolor[rgb]{0.73,0.13,0.13}{##1}}}
\expandafter\def\csname PY@tok@sh\endcsname{\def\PY@tc##1{\textcolor[rgb]{0.73,0.13,0.13}{##1}}}
\expandafter\def\csname PY@tok@s1\endcsname{\def\PY@tc##1{\textcolor[rgb]{0.73,0.13,0.13}{##1}}}
\expandafter\def\csname PY@tok@mb\endcsname{\def\PY@tc##1{\textcolor[rgb]{0.40,0.40,0.40}{##1}}}
\expandafter\def\csname PY@tok@mf\endcsname{\def\PY@tc##1{\textcolor[rgb]{0.40,0.40,0.40}{##1}}}
\expandafter\def\csname PY@tok@mh\endcsname{\def\PY@tc##1{\textcolor[rgb]{0.40,0.40,0.40}{##1}}}
\expandafter\def\csname PY@tok@mi\endcsname{\def\PY@tc##1{\textcolor[rgb]{0.40,0.40,0.40}{##1}}}
\expandafter\def\csname PY@tok@il\endcsname{\def\PY@tc##1{\textcolor[rgb]{0.40,0.40,0.40}{##1}}}
\expandafter\def\csname PY@tok@mo\endcsname{\def\PY@tc##1{\textcolor[rgb]{0.40,0.40,0.40}{##1}}}
\expandafter\def\csname PY@tok@ch\endcsname{\let\PY@it=\textit\def\PY@tc##1{\textcolor[rgb]{0.25,0.50,0.50}{##1}}}
\expandafter\def\csname PY@tok@cm\endcsname{\let\PY@it=\textit\def\PY@tc##1{\textcolor[rgb]{0.25,0.50,0.50}{##1}}}
\expandafter\def\csname PY@tok@cpf\endcsname{\let\PY@it=\textit\def\PY@tc##1{\textcolor[rgb]{0.25,0.50,0.50}{##1}}}
\expandafter\def\csname PY@tok@c1\endcsname{\let\PY@it=\textit\def\PY@tc##1{\textcolor[rgb]{0.25,0.50,0.50}{##1}}}
\expandafter\def\csname PY@tok@cs\endcsname{\let\PY@it=\textit\def\PY@tc##1{\textcolor[rgb]{0.25,0.50,0.50}{##1}}}

\def\PYZbs{\char`\\}
\def\PYZus{\char`\_}
\def\PYZob{\char`\{}
\def\PYZcb{\char`\}}
\def\PYZca{\char`\^}
\def\PYZam{\char`\&}
\def\PYZlt{\char`\<}
\def\PYZgt{\char`\>}
\def\PYZsh{\char`\#}
\def\PYZpc{\char`\%}
\def\PYZdl{\char`\$}
\def\PYZhy{\char`\-}
\def\PYZsq{\char`\'}
\def\PYZdq{\char`\"}
\def\PYZti{\char`\~}
% for compatibility with earlier versions
\def\PYZat{@}
\def\PYZlb{[}
\def\PYZrb{]}
\makeatother


    % Exact colors from NB
    \definecolor{incolor}{rgb}{0.0, 0.0, 0.5}
    \definecolor{outcolor}{rgb}{0.545, 0.0, 0.0}



    
    % Prevent overflowing lines due to hard-to-break entities
    \sloppy 
    % Setup hyperref package
    \hypersetup{
      breaklinks=true,  % so long urls are correctly broken across lines
      colorlinks=true,
      urlcolor=urlcolor,
      linkcolor=linkcolor,
      citecolor=citecolor,
      }
    % Slightly bigger margins than the latex defaults
    
    \geometry{verbose,tmargin=1in,bmargin=1in,lmargin=1in,rmargin=1in}
    
    

    \begin{document}
    
    
    \maketitle
    
    

    
    \begin{Verbatim}[commandchars=\\\{\}]
{\color{incolor}In [{\color{incolor}183}]:} \PY{k+kn}{import} \PY{n+nn}{matplotlib}\PY{n+nn}{.}\PY{n+nn}{pyplot} \PY{k}{as} \PY{n+nn}{plt}
          \PY{k+kn}{import} \PY{n+nn}{numpy}
\end{Verbatim}


    \hypertarget{variabile-aleatoria-bernulliana}{%
\paragraph{VARIABILE ALEATORIA
BERNULLIANA}\label{variabile-aleatoria-bernulliana}}

E' una distribuzione di probabilità su soli due valori chiamati
\texttt{0} e \texttt{1} .\\
Una distribuzione Bernulliana è parametrizzata rispetto ad un parametro
\texttt{p} .\\
\texttt{p} indica la probabilità di successo, ovvero se
\texttt{X\ \textasciitilde{}\ B(p)} allora \texttt{P(X=1)\ =\ p} .\\
Dove con la notazione \texttt{X\ \textasciitilde{}\ B(p)} diciamo che
\texttt{X} è estratto da una distribuzione Bernulliana di parametro
\texttt{p} .\\
Spesso, la bernulliana è associata al lancio di una moneta
potenzialemente truccata, dove \texttt{P(X=testa)=p}.

    \begin{Verbatim}[commandchars=\\\{\}]
{\color{incolor}In [{\color{incolor}184}]:} \PY{c+c1}{\PYZsh{} questa funzione ci ritornerà delle variabili aleatorie        \PYZsh{}\PYZsh{}\PYZsh{}}
          \PY{c+c1}{\PYZsh{} estratte secondo la distribuzione di bernulli con parametro p \PYZsh{}\PYZsh{}\PYZsh{}}
          \PY{k}{def} \PY{n+nf}{get\PYZus{}bernulli\PYZus{}random\PYZus{}variable}\PY{p}{(}\PY{n}{p}\PY{o}{=}\PY{l+m+mf}{0.5}\PY{p}{)}\PY{p}{:}
              \PY{k}{def} \PY{n+nf}{random\PYZus{}variable}\PY{p}{(}\PY{p}{)}\PY{p}{:}
                  \PY{k}{return} \PY{l+m+mi}{1} \PY{k}{if} \PY{n}{numpy}\PY{o}{.}\PY{n}{random}\PY{o}{.}\PY{n}{random}\PY{p}{(}\PY{p}{)}\PY{o}{\PYZlt{}}\PY{n}{p} \PY{k}{else} \PY{l+m+mi}{0}
              \PY{k}{return} \PY{n}{random\PYZus{}variable}
          
          \PY{c+c1}{\PYZsh{} costruiamo una variabile aleatoria corrispondente a una moneta \PYZdq{}fair\PYZdq{} \PYZsh{}\PYZsh{}\PYZsh{}}
          \PY{n}{f0} \PY{o}{=} \PY{n}{get\PYZus{}bernulli\PYZus{}random\PYZus{}variable}\PY{p}{(}\PY{l+m+mf}{0.5}\PY{p}{)}
          
          \PY{c+c1}{\PYZsh{} facciamo un po\PYZsq{} di test \PYZsh{}\PYZsh{}\PYZsh{}}
          \PY{k}{for} \PY{n}{i} \PY{o+ow}{in} \PY{n+nb}{range}\PY{p}{(}\PY{l+m+mi}{10}\PY{p}{)}\PY{p}{:}
              \PY{n+nb}{print}\PY{p}{(}\PY{n}{f0}\PY{p}{(}\PY{p}{)}\PY{p}{,} \PY{n}{end}\PY{o}{=}\PY{l+s+s2}{\PYZdq{}}\PY{l+s+s2}{ }\PY{l+s+s2}{\PYZdq{}}\PY{p}{)}
          \PY{n+nb}{print}\PY{p}{(}\PY{p}{)}
\end{Verbatim}


    \begin{Verbatim}[commandchars=\\\{\}]
1 1 1 1 0 1 1 1 0 0 

    \end{Verbatim}

    Se facciamo abbastanza esperimenti dovremmo notare che la distribuzione
dei nostri lanci converge a quella teorica .\\
Questo vuol dire che anche varianza e media degli esperimenti
convergeranno al valore atteso e alla varianza della distribuzione
teorica.

    \begin{Verbatim}[commandchars=\\\{\}]
{\color{incolor}In [{\color{incolor}185}]:} \PY{c+c1}{\PYZsh{} verifichiamo che il valore atteso di una bernulliana di parametro p è proprio p \PYZsh{}\PYZsh{}\PYZsh{}}
          \PY{c+c1}{\PYZsh{} e che la varianza è p(1\PYZhy{}p)                                                      \PYZsh{}\PYZsh{}\PYZsh{}}
          
          \PY{n}{p} \PY{o}{=} \PY{l+m+mf}{0.75}
          \PY{n}{theoretical\PYZus{}mean} \PY{o}{=} \PY{n}{p}
          \PY{n}{theoretical\PYZus{}variance} \PY{o}{=} \PY{n}{p}\PY{o}{*}\PY{p}{(}\PY{l+m+mi}{1}\PY{o}{\PYZhy{}}\PY{n}{p}\PY{p}{)}
          
          \PY{n}{f0} \PY{o}{=} \PY{n}{get\PYZus{}bernulli\PYZus{}random\PYZus{}variable}\PY{p}{(}\PY{n}{p}\PY{o}{=}\PY{l+m+mf}{0.75}\PY{p}{)}
          \PY{c+c1}{\PYZsh{} facciamo un milione di test \PYZsh{}\PYZsh{}\PYZsh{}}
          \PY{n}{tests} \PY{o}{=} \PY{p}{[}\PY{n}{f0}\PY{p}{(}\PY{p}{)} \PY{k}{for} \PY{n}{i} \PY{o+ow}{in} \PY{n+nb}{range}\PY{p}{(}\PY{l+m+mi}{10}\PY{o}{*}\PY{o}{*}\PY{l+m+mi}{2}\PY{p}{)}\PY{p}{]}
          
          \PY{n+nb}{print}\PY{p}{(}\PY{l+s+s2}{\PYZdq{}}\PY{l+s+s2}{the computed mean is       :}\PY{l+s+s2}{\PYZdq{}}\PY{p}{,}\PY{n}{numpy}\PY{o}{.}\PY{n}{mean}\PY{p}{(}\PY{n}{tests}\PY{p}{)}\PY{p}{)}
          \PY{n+nb}{print}\PY{p}{(}\PY{l+s+s2}{\PYZdq{}}\PY{l+s+s2}{the theoretical mean is    :}\PY{l+s+s2}{\PYZdq{}}\PY{p}{,}\PY{n}{theoretical\PYZus{}mean}\PY{p}{)}
          \PY{n+nb}{print}\PY{p}{(}\PY{l+s+s2}{\PYZdq{}}\PY{l+s+s2}{the computed variance is   :}\PY{l+s+s2}{\PYZdq{}}\PY{p}{,}\PY{n}{numpy}\PY{o}{.}\PY{n}{var}\PY{p}{(}\PY{n}{tests}\PY{p}{)}\PY{p}{)}
          \PY{n+nb}{print}\PY{p}{(}\PY{l+s+s2}{\PYZdq{}}\PY{l+s+s2}{the theoretical variance is:}\PY{l+s+s2}{\PYZdq{}}\PY{p}{,}\PY{n}{theoretical\PYZus{}variance}\PY{p}{)}
\end{Verbatim}


    \begin{Verbatim}[commandchars=\\\{\}]
the computed mean is       : 0.7
the theoretical mean is    : 0.75
the computed variance is   : 0.20999999999999996
the theoretical variance is: 0.1875

    \end{Verbatim}

    \begin{Verbatim}[commandchars=\\\{\}]
{\color{incolor}In [{\color{incolor}186}]:} \PY{c+c1}{\PYZsh{}\PYZsh{}\PYZsh{} plottiamo anche un istogramma \PYZsh{}\PYZsh{}\PYZsh{}}
          \PY{n}{fig}\PY{p}{,} \PY{n}{ax} \PY{o}{=} \PY{n}{plt}\PY{o}{.}\PY{n}{subplots}\PY{p}{(}\PY{n}{figsize}\PY{o}{=}\PY{p}{(}\PY{l+m+mi}{15}\PY{p}{,}\PY{l+m+mi}{10}\PY{p}{)}\PY{p}{)}
          \PY{n}{ax}\PY{o}{.}\PY{n}{hist}\PY{p}{(}\PY{n}{tests}\PY{p}{,} \PY{n}{bins}\PY{o}{=}\PY{p}{[}\PY{l+m+mi}{0}\PY{p}{,}\PY{l+m+mi}{1}\PY{p}{,}\PY{l+m+mi}{2}\PY{p}{]}\PY{p}{,} \PY{n}{density}\PY{o}{=}\PY{k+kc}{True}\PY{p}{,} \PY{n}{align}\PY{o}{=}\PY{l+s+s2}{\PYZdq{}}\PY{l+s+s2}{left}\PY{l+s+s2}{\PYZdq{}}\PY{p}{,} 
                  \PY{n}{color}\PY{o}{=}\PY{l+s+s2}{\PYZdq{}}\PY{l+s+s2}{darkorchid}\PY{l+s+s2}{\PYZdq{}}\PY{p}{,} \PY{n}{rwidth}\PY{o}{=}\PY{l+m+mf}{0.3}\PY{p}{)}
          \PY{n}{ax}\PY{o}{.}\PY{n}{bar}\PY{p}{(}\PY{p}{[}\PY{l+m+mi}{0}\PY{p}{,}\PY{l+m+mi}{1}\PY{p}{]}\PY{p}{,}\PY{p}{[}\PY{l+m+mi}{1}\PY{o}{\PYZhy{}}\PY{n}{p}\PY{p}{,}\PY{n}{p}\PY{p}{]}\PY{p}{,} \PY{n}{align}\PY{o}{=}\PY{l+s+s2}{\PYZdq{}}\PY{l+s+s2}{center}\PY{l+s+s2}{\PYZdq{}}\PY{p}{,} 
                 \PY{n}{color}\PY{o}{=}\PY{l+s+s2}{\PYZdq{}}\PY{l+s+s2}{orange}\PY{l+s+s2}{\PYZdq{}}\PY{p}{,} \PY{n}{width}\PY{o}{=}\PY{l+m+mf}{0.01}\PY{p}{)}
          \PY{n}{ax}\PY{o}{.}\PY{n}{set\PYZus{}xticklabels}\PY{p}{(}\PY{p}{[}\PY{l+m+mi}{0}\PY{p}{,}\PY{l+m+mi}{1}\PY{p}{]}\PY{p}{)}
          \PY{n}{ax}\PY{o}{.}\PY{n}{set\PYZus{}xticks}\PY{p}{(}\PY{p}{[}\PY{l+m+mi}{0}\PY{p}{,}\PY{l+m+mi}{1}\PY{p}{]}\PY{p}{)}
          \PY{n}{ax}\PY{o}{.}\PY{n}{legend}\PY{p}{(}\PY{p}{(}\PY{l+s+s2}{\PYZdq{}}\PY{l+s+s2}{empirical}\PY{l+s+s2}{\PYZdq{}}\PY{p}{,} \PY{l+s+s2}{\PYZdq{}}\PY{l+s+s2}{theoretical}\PY{l+s+s2}{\PYZdq{}}\PY{p}{)}\PY{p}{)}
          \PY{k}{del} \PY{n}{tests}
\end{Verbatim}


    \begin{center}
    \adjustimage{max size={0.9\linewidth}{0.9\paperheight}}{output_5_0.png}
    \end{center}
    { \hspace*{\fill} \\}
    
    \hypertarget{variabile-aleatoria-binomiale}{%
\subsubsection{VARIABILE ALEATORIA
BINOMIALE}\label{variabile-aleatoria-binomiale}}

E' la variabile aleatoria che descrive il numero di successi di un
processo di Bernulli.

Di conseguenza, possiamo costruire la nostra variabile aleatoria
Binomiale sfruttando le Bernulliane che abbiamo costruito in
precedenza.\\
Diremo:

Per denotare che X è estratta da una variabile aleatoria Binomiale
caratterizzata da i parametri n e p.

    \begin{Verbatim}[commandchars=\\\{\}]
{\color{incolor}In [{\color{incolor}5}]:} \PY{c+c1}{\PYZsh{} la distribuzione binomiale B(n,p) somma n variabili aleatorie indipendenti \PYZsh{}\PYZsh{}\PYZsh{}}
        \PY{k}{def} \PY{n+nf}{get\PYZus{}binomial\PYZus{}random\PYZus{}variable}\PY{p}{(}\PY{n}{n}\PY{o}{=}\PY{l+m+mi}{5}\PY{p}{,}\PY{n}{p}\PY{o}{=}\PY{l+m+mf}{0.5}\PY{p}{)}\PY{p}{:}
            \PY{c+c1}{\PYZsh{} costruiamo n bernulliane di parametro p}
            \PY{n}{rvs} \PY{o}{=} \PY{p}{[}\PY{n}{get\PYZus{}bernulli\PYZus{}random\PYZus{}variable}\PY{p}{(}\PY{n}{p}\PY{p}{)} \PY{k}{for} \PY{n}{i} \PY{o+ow}{in} \PY{n+nb}{range}\PY{p}{(}\PY{n}{n}\PY{p}{)}\PY{p}{]}
            \PY{k}{def} \PY{n+nf}{random\PYZus{}variable}\PY{p}{(}\PY{p}{)}\PY{p}{:}
                \PY{k}{return} \PY{n+nb}{sum}\PY{p}{(}\PY{p}{[}\PY{n}{rv}\PY{p}{(}\PY{p}{)} \PY{k}{for} \PY{n}{rv} \PY{o+ow}{in} \PY{n}{rvs}\PY{p}{]}\PY{p}{)}
            \PY{k}{return} \PY{n}{random\PYZus{}variable}
        
        \PY{n}{f0} \PY{o}{=} \PY{n}{get\PYZus{}binomial\PYZus{}random\PYZus{}variable}\PY{p}{(}\PY{n}{n}\PY{o}{=}\PY{l+m+mi}{5}\PY{p}{,}\PY{n}{p}\PY{o}{=}\PY{l+m+mf}{0.5}\PY{p}{)}
        
        \PY{c+c1}{\PYZsh{} facciamo un po\PYZsq{} di test \PYZsh{}\PYZsh{}\PYZsh{}}
        \PY{k}{for} \PY{n}{i} \PY{o+ow}{in} \PY{n+nb}{range}\PY{p}{(}\PY{l+m+mi}{10}\PY{p}{)}\PY{p}{:}
            \PY{n+nb}{print}\PY{p}{(}\PY{n}{f0}\PY{p}{(}\PY{p}{)}\PY{p}{,} \PY{n}{end}\PY{o}{=}\PY{l+s+s2}{\PYZdq{}}\PY{l+s+s2}{ }\PY{l+s+s2}{\PYZdq{}}\PY{p}{)}
\end{Verbatim}


    \begin{Verbatim}[commandchars=\\\{\}]
2 4 1 4 3 0 4 2 3 3 
    \end{Verbatim}

    Ancora una volta possiamo verificare che con l'aumentare del numero di
tentativi la distribuzione ottenuta converge a quella teorica.

    \begin{Verbatim}[commandchars=\\\{\}]
{\color{incolor}In [{\color{incolor}6}]:} \PY{c+c1}{\PYZsh{} verifichiamo che il valore atteso di una binomiale di parametro n,p è n*p \PYZsh{}\PYZsh{}\PYZsh{}}
        \PY{c+c1}{\PYZsh{} e che la varianza è n*p*(1\PYZhy{}p)                                             \PYZsh{}\PYZsh{}\PYZsh{}}
        
        \PY{n}{p} \PY{o}{=} \PY{l+m+mf}{0.75}
        \PY{n}{n} \PY{o}{=} \PY{l+m+mi}{10}
        \PY{n}{theoretical\PYZus{}mean} \PY{o}{=} \PY{n}{n}\PY{o}{*}\PY{n}{p}
        \PY{n}{theoretical\PYZus{}variance} \PY{o}{=} \PY{n}{n}\PY{o}{*}\PY{n}{p}\PY{o}{*}\PY{p}{(}\PY{l+m+mi}{1}\PY{o}{\PYZhy{}}\PY{n}{p}\PY{p}{)}
        
        \PY{n}{f0} \PY{o}{=} \PY{n}{get\PYZus{}binomial\PYZus{}random\PYZus{}variable}\PY{p}{(}\PY{n}{n}\PY{o}{=}\PY{n}{n}\PY{p}{,}\PY{n}{p}\PY{o}{=}\PY{n}{p}\PY{p}{)}
        
        \PY{c+c1}{\PYZsh{} facciamo un milione di test \PYZsh{}\PYZsh{}\PYZsh{}}
        \PY{n}{tests} \PY{o}{=} \PY{p}{[}\PY{n}{f0}\PY{p}{(}\PY{p}{)} \PY{k}{for} \PY{n}{i} \PY{o+ow}{in} \PY{n+nb}{range}\PY{p}{(}\PY{l+m+mi}{10}\PY{o}{*}\PY{o}{*}\PY{l+m+mi}{2}\PY{p}{)}\PY{p}{]}
        
        \PY{n+nb}{print}\PY{p}{(}\PY{l+s+s2}{\PYZdq{}}\PY{l+s+s2}{the computed mean is       :}\PY{l+s+s2}{\PYZdq{}}\PY{p}{,}\PY{n}{numpy}\PY{o}{.}\PY{n}{mean}\PY{p}{(}\PY{n}{tests}\PY{p}{)}\PY{p}{)}
        \PY{n+nb}{print}\PY{p}{(}\PY{l+s+s2}{\PYZdq{}}\PY{l+s+s2}{the theoretical mean is    :}\PY{l+s+s2}{\PYZdq{}}\PY{p}{,}\PY{n}{theoretical\PYZus{}mean}\PY{p}{)}
        \PY{n+nb}{print}\PY{p}{(}\PY{l+s+s2}{\PYZdq{}}\PY{l+s+s2}{the computed variance is   :}\PY{l+s+s2}{\PYZdq{}}\PY{p}{,}\PY{n}{numpy}\PY{o}{.}\PY{n}{var}\PY{p}{(}\PY{n}{tests}\PY{p}{)}\PY{p}{)}
        \PY{n+nb}{print}\PY{p}{(}\PY{l+s+s2}{\PYZdq{}}\PY{l+s+s2}{the theoretical variance is:}\PY{l+s+s2}{\PYZdq{}}\PY{p}{,}\PY{n}{theoretical\PYZus{}variance}\PY{p}{)}
\end{Verbatim}


    \begin{Verbatim}[commandchars=\\\{\}]
the computed mean is       : 7.53
the theoretical mean is    : 7.5
the computed variance is   : 2.3691000000000004
the theoretical variance is: 1.875

    \end{Verbatim}

    \begin{Verbatim}[commandchars=\\\{\}]
{\color{incolor}In [{\color{incolor}7}]:} \PY{c+c1}{\PYZsh{} costruiamo la distribuzione binomiale teorica \PYZsh{}\PYZsh{}\PYZsh{}}
        
        \PY{c+c1}{\PYZsh{} ci serve la funzione fattoriale \PYZsh{}\PYZsh{}\PYZsh{}}
        \PY{k}{def} \PY{n+nf}{factorial}\PY{p}{(}\PY{n}{n}\PY{p}{)}\PY{p}{:}
            \PY{k}{assert} \PY{n}{n} \PY{o}{\PYZgt{}}\PY{o}{=} \PY{l+m+mi}{0}
            \PY{k}{return} \PY{l+m+mi}{1} \PY{k}{if} \PY{n}{n} \PY{o}{\PYZlt{}} \PY{l+m+mi}{2} \PY{k}{else} \PY{n}{n}\PY{o}{*}\PY{n}{factorial}\PY{p}{(}\PY{n}{n}\PY{o}{\PYZhy{}}\PY{l+m+mi}{1}\PY{p}{)}
        
        \PY{c+c1}{\PYZsh{} usando il fattoriale possiamo calcolare il coefficiente binomiale \PYZsh{}\PYZsh{}\PYZsh{}}
        \PY{k}{def} \PY{n+nf}{binomial\PYZus{}coeff}\PY{p}{(}\PY{n}{n}\PY{p}{,}\PY{n}{k}\PY{p}{)}\PY{p}{:}
            \PY{k}{assert} \PY{n}{n} \PY{o}{\PYZgt{}}\PY{o}{=} \PY{n}{k} \PY{o+ow}{and} \PY{n}{n} \PY{o}{\PYZgt{}}\PY{o}{=} \PY{l+m+mi}{0} \PY{o+ow}{and} \PY{n}{k} \PY{o}{\PYZgt{}}\PY{o}{=} \PY{l+m+mi}{0}
            \PY{k}{return} \PY{n}{factorial}\PY{p}{(}\PY{n}{n}\PY{p}{)}\PY{o}{/}\PY{o}{/}\PY{p}{(}\PY{n}{factorial}\PY{p}{(}\PY{n}{k}\PY{p}{)}\PY{o}{*}\PY{n}{factorial}\PY{p}{(}\PY{n}{n}\PY{o}{\PYZhy{}}\PY{n}{k}\PY{p}{)}\PY{p}{)}
             
        \PY{c+c1}{\PYZsh{} usando il coefficiente binomiale possiamo calcolare la distribuzione binomiale \PYZsh{}\PYZsh{}\PYZsh{}}
        \PY{k}{def} \PY{n+nf}{binomial\PYZus{}dist}\PY{p}{(}\PY{n}{n}\PY{p}{,}\PY{n}{p}\PY{p}{)}\PY{p}{:}
            \PY{k}{assert} \PY{n}{p} \PY{o}{\PYZgt{}}\PY{o}{=} \PY{l+m+mi}{0} \PY{o+ow}{and} \PY{n}{p} \PY{o}{\PYZlt{}}\PY{o}{=} \PY{l+m+mi}{1} \PY{o+ow}{and} \PY{n}{n} \PY{o}{\PYZgt{}}\PY{o}{=} \PY{l+m+mi}{0}
            \PY{k}{return} \PY{p}{[}\PY{n}{binomial\PYZus{}coeff}\PY{p}{(}\PY{n}{n}\PY{p}{,}\PY{n}{k}\PY{p}{)}\PY{o}{*}\PY{p}{(}\PY{n}{p}\PY{o}{*}\PY{o}{*}\PY{n}{k}\PY{p}{)}\PY{o}{*}\PY{p}{(}\PY{l+m+mi}{1}\PY{o}{\PYZhy{}}\PY{n}{p}\PY{p}{)}\PY{o}{*}\PY{o}{*}\PY{p}{(}\PY{n}{n}\PY{o}{\PYZhy{}}\PY{n}{k}\PY{p}{)} \PY{k}{for} \PY{n}{k} \PY{o+ow}{in} \PY{n+nb}{range}\PY{p}{(}\PY{n}{n}\PY{o}{+}\PY{l+m+mi}{1}\PY{p}{)}\PY{p}{]}
\end{Verbatim}


    \begin{Verbatim}[commandchars=\\\{\}]
{\color{incolor}In [{\color{incolor}8}]:} \PY{c+c1}{\PYZsh{} ancora una volta, facciamo il plot dell\PYZsq{}istogramma \PYZsh{}\PYZsh{}\PYZsh{}}
        \PY{n}{fig}\PY{p}{,} \PY{n}{ax} \PY{o}{=} \PY{n}{plt}\PY{o}{.}\PY{n}{subplots}\PY{p}{(}\PY{n}{figsize}\PY{o}{=}\PY{p}{(}\PY{l+m+mi}{15}\PY{p}{,}\PY{l+m+mi}{10}\PY{p}{)}\PY{p}{)}
        \PY{n}{ax}\PY{o}{.}\PY{n}{hist}\PY{p}{(}\PY{n}{tests}\PY{p}{,} \PY{n}{bins}\PY{o}{=}\PY{n}{numpy}\PY{o}{.}\PY{n}{arange}\PY{p}{(}\PY{n}{n}\PY{o}{+}\PY{l+m+mi}{2}\PY{p}{)}\PY{o}{\PYZhy{}}\PY{l+m+mf}{0.5}\PY{p}{,} \PY{n}{density}\PY{o}{=}\PY{k+kc}{True}\PY{p}{,} \PY{n}{align}\PY{o}{=}\PY{l+s+s2}{\PYZdq{}}\PY{l+s+s2}{mid}\PY{l+s+s2}{\PYZdq{}}\PY{p}{,} 
                \PY{n}{color}\PY{o}{=}\PY{l+s+s2}{\PYZdq{}}\PY{l+s+s2}{darkorchid}\PY{l+s+s2}{\PYZdq{}}\PY{p}{,} \PY{n}{rwidth}\PY{o}{=}\PY{l+m+mf}{0.7}\PY{p}{)}
        \PY{n}{ax}\PY{o}{.}\PY{n}{bar}\PY{p}{(}\PY{n+nb}{range}\PY{p}{(}\PY{n}{n}\PY{o}{+}\PY{l+m+mi}{1}\PY{p}{)}\PY{p}{,} \PY{n}{binomial\PYZus{}dist}\PY{p}{(}\PY{n}{n}\PY{p}{,}\PY{n}{p}\PY{p}{)}\PY{p}{,} \PY{n}{align}\PY{o}{=}\PY{l+s+s2}{\PYZdq{}}\PY{l+s+s2}{center}\PY{l+s+s2}{\PYZdq{}}\PY{p}{,} 
               \PY{n}{color}\PY{o}{=}\PY{l+s+s2}{\PYZdq{}}\PY{l+s+s2}{orange}\PY{l+s+s2}{\PYZdq{}}\PY{p}{,} \PY{n}{width}\PY{o}{=}\PY{l+m+mf}{0.1}\PY{p}{)}
        \PY{n}{ax}\PY{o}{.}\PY{n}{set\PYZus{}xticklabels}\PY{p}{(}\PY{n+nb}{range}\PY{p}{(}\PY{n}{n}\PY{o}{+}\PY{l+m+mi}{1}\PY{p}{)}\PY{p}{)}
        \PY{n}{ax}\PY{o}{.}\PY{n}{set\PYZus{}xticks}\PY{p}{(}\PY{n+nb}{range}\PY{p}{(}\PY{n}{n}\PY{o}{+}\PY{l+m+mi}{1}\PY{p}{)}\PY{p}{)}
        \PY{n}{ax}\PY{o}{.}\PY{n}{legend}\PY{p}{(}\PY{p}{(}\PY{l+s+s2}{\PYZdq{}}\PY{l+s+s2}{empirical}\PY{l+s+s2}{\PYZdq{}}\PY{p}{,} \PY{l+s+s2}{\PYZdq{}}\PY{l+s+s2}{theoretical}\PY{l+s+s2}{\PYZdq{}}\PY{p}{)}\PY{p}{)}
        \PY{k}{del} \PY{n}{tests}
\end{Verbatim}


    \begin{center}
    \adjustimage{max size={0.9\linewidth}{0.9\paperheight}}{output_11_0.png}
    \end{center}
    { \hspace*{\fill} \\}
    
    \begin{Verbatim}[commandchars=\\\{\}]
{\color{incolor}In [{\color{incolor}9}]:} \PY{c+c1}{\PYZsh{} possiamo giocare un po\PYZsq{} con la binomiale                    \PYZsh{}\PYZsh{}\PYZsh{}}
        \PY{c+c1}{\PYZsh{} creaiamo una funzione che plotti una simulazione dati n e p \PYZsh{}\PYZsh{}\PYZsh{}}
        \PY{c+c1}{\PYZsh{} e la compari con la controparte teorica                     \PYZsh{}\PYZsh{}\PYZsh{}}
        
        \PY{k}{def} \PY{n+nf}{plot\PYZus{}binomial}\PY{p}{(}\PY{n}{n}\PY{o}{=}\PY{l+m+mi}{5}\PY{p}{,}\PY{n}{p}\PY{o}{=}\PY{l+m+mf}{0.5}\PY{p}{,}\PY{n}{steps}\PY{o}{=}\PY{l+m+mi}{10}\PY{o}{*}\PY{o}{*}\PY{l+m+mi}{5}\PY{p}{)}\PY{p}{:}
            \PY{n}{rv} \PY{o}{=} \PY{n}{get\PYZus{}binomial\PYZus{}random\PYZus{}variable}\PY{p}{(}\PY{n}{n}\PY{o}{=}\PY{n}{n}\PY{p}{,}\PY{n}{p}\PY{o}{=}\PY{n}{p}\PY{p}{)}
            \PY{n}{tests} \PY{o}{=} \PY{p}{[}\PY{n}{rv}\PY{p}{(}\PY{p}{)} \PY{k}{for} \PY{n}{i} \PY{o+ow}{in} \PY{n+nb}{range}\PY{p}{(}\PY{n}{steps}\PY{p}{)}\PY{p}{]}
            \PY{n}{fig}\PY{p}{,} \PY{n}{ax} \PY{o}{=} \PY{n}{plt}\PY{o}{.}\PY{n}{subplots}\PY{p}{(}\PY{n}{figsize}\PY{o}{=}\PY{p}{(}\PY{l+m+mi}{15}\PY{p}{,}\PY{l+m+mi}{10}\PY{p}{)}\PY{p}{)}
            \PY{n}{ax}\PY{o}{.}\PY{n}{hist}\PY{p}{(}\PY{n}{tests}\PY{p}{,} \PY{n}{bins}\PY{o}{=}\PY{n}{numpy}\PY{o}{.}\PY{n}{arange}\PY{p}{(}\PY{n}{n}\PY{o}{+}\PY{l+m+mi}{2}\PY{p}{)}\PY{o}{\PYZhy{}}\PY{l+m+mf}{0.5}\PY{p}{,} \PY{n}{density}\PY{o}{=}\PY{k+kc}{True}\PY{p}{,} 
                    \PY{n}{align}\PY{o}{=}\PY{l+s+s2}{\PYZdq{}}\PY{l+s+s2}{mid}\PY{l+s+s2}{\PYZdq{}}\PY{p}{,} \PY{n}{color}\PY{o}{=}\PY{l+s+s2}{\PYZdq{}}\PY{l+s+s2}{darkorchid}\PY{l+s+s2}{\PYZdq{}}\PY{p}{,} \PY{n}{rwidth}\PY{o}{=}\PY{l+m+mi}{1}\PY{p}{,} \PY{n}{zorder}\PY{o}{=}\PY{l+m+mi}{0}\PY{p}{)}
            \PY{n}{ax}\PY{o}{.}\PY{n}{bar}\PY{p}{(}\PY{n+nb}{range}\PY{p}{(}\PY{n}{n}\PY{o}{+}\PY{l+m+mi}{1}\PY{p}{)}\PY{p}{,} \PY{n}{binomial\PYZus{}dist}\PY{p}{(}\PY{n}{n}\PY{p}{,}\PY{n}{p}\PY{p}{)}\PY{p}{,} 
                   \PY{n}{align}\PY{o}{=}\PY{l+s+s2}{\PYZdq{}}\PY{l+s+s2}{center}\PY{l+s+s2}{\PYZdq{}}\PY{p}{,} \PY{n}{color}\PY{o}{=}\PY{l+s+s2}{\PYZdq{}}\PY{l+s+s2}{orange}\PY{l+s+s2}{\PYZdq{}}\PY{p}{,} \PY{n}{zorder}\PY{o}{=}\PY{l+m+mi}{1}\PY{p}{,} \PY{n}{width}\PY{o}{=}\PY{l+m+mf}{0.2}\PY{p}{)}
            \PY{n}{ax}\PY{o}{.}\PY{n}{set\PYZus{}xticklabels}\PY{p}{(}\PY{n+nb}{range}\PY{p}{(}\PY{n}{n}\PY{o}{+}\PY{l+m+mi}{1}\PY{p}{)}\PY{p}{)}
            \PY{n}{ax}\PY{o}{.}\PY{n}{set\PYZus{}xticks}\PY{p}{(}\PY{n+nb}{range}\PY{p}{(}\PY{n}{n}\PY{o}{+}\PY{l+m+mi}{1}\PY{p}{)}\PY{p}{)}
            \PY{n}{ax}\PY{o}{.}\PY{n}{set\PYZus{}title}\PY{p}{(}\PY{l+s+s2}{\PYZdq{}}\PY{l+s+s2}{n=}\PY{l+s+si}{\PYZob{}\PYZcb{}}\PY{l+s+s2}{ and p=}\PY{l+s+si}{\PYZob{}\PYZcb{}}\PY{l+s+s2}{\PYZdq{}}\PY{o}{.}\PY{n}{format}\PY{p}{(}\PY{n}{n}\PY{p}{,}\PY{n}{p}\PY{p}{)}\PY{p}{)}
            \PY{k}{del} \PY{n}{tests}\PY{p}{,} \PY{n}{fig}\PY{p}{,} \PY{n}{ax}
        
        \PY{n}{plot\PYZus{}binomial}\PY{p}{(}\PY{n}{n}\PY{o}{=}\PY{l+m+mi}{25}\PY{p}{,}\PY{n}{p}\PY{o}{=}\PY{l+m+mf}{0.75}\PY{p}{,}\PY{n}{steps}\PY{o}{=}\PY{l+m+mi}{1000}\PY{p}{)}
\end{Verbatim}


    \begin{center}
    \adjustimage{max size={0.9\linewidth}{0.9\paperheight}}{output_12_0.png}
    \end{center}
    { \hspace*{\fill} \\}
    
    \begin{Verbatim}[commandchars=\\\{\}]
{\color{incolor}In [{\color{incolor}10}]:} \PY{c+c1}{\PYZsh{} possiamo giocare un po\PYZsq{} con la binomiale                    \PYZsh{}\PYZsh{}\PYZsh{}}
         \PY{c+c1}{\PYZsh{} creaiamo una funzione che plotti una simulazione dati n e p \PYZsh{}\PYZsh{}\PYZsh{}}
         \PY{c+c1}{\PYZsh{} e la compari con la controparte teorica                     \PYZsh{}\PYZsh{}\PYZsh{}}
         \PY{c+c1}{\PYZsh{} questa volta con la cumulata                                \PYZsh{}\PYZsh{}\PYZsh{}}
         
         \PY{k}{def} \PY{n+nf}{plot\PYZus{}binomial}\PY{p}{(}\PY{n}{n}\PY{o}{=}\PY{l+m+mi}{5}\PY{p}{,}\PY{n}{p}\PY{o}{=}\PY{l+m+mf}{0.5}\PY{p}{,}\PY{n}{steps}\PY{o}{=}\PY{l+m+mi}{10}\PY{o}{*}\PY{o}{*}\PY{l+m+mi}{5}\PY{p}{)}\PY{p}{:}
             \PY{n}{rv} \PY{o}{=} \PY{n}{get\PYZus{}binomial\PYZus{}random\PYZus{}variable}\PY{p}{(}\PY{n}{n}\PY{o}{=}\PY{n}{n}\PY{p}{,}\PY{n}{p}\PY{o}{=}\PY{n}{p}\PY{p}{)}
             \PY{n}{tests} \PY{o}{=} \PY{p}{[}\PY{n}{rv}\PY{p}{(}\PY{p}{)} \PY{k}{for} \PY{n}{i} \PY{o+ow}{in} \PY{n+nb}{range}\PY{p}{(}\PY{n}{steps}\PY{p}{)}\PY{p}{]}
             \PY{n}{fig}\PY{p}{,} \PY{n}{ax} \PY{o}{=} \PY{n}{plt}\PY{o}{.}\PY{n}{subplots}\PY{p}{(}\PY{n}{figsize}\PY{o}{=}\PY{p}{(}\PY{l+m+mi}{15}\PY{p}{,}\PY{l+m+mi}{10}\PY{p}{)}\PY{p}{)}
             
             \PY{n}{quantiles} \PY{o}{=} \PY{n}{numpy}\PY{o}{.}\PY{n}{cumsum}\PY{p}{(}\PY{p}{[}\PY{n}{tests}\PY{o}{.}\PY{n}{count}\PY{p}{(}\PY{n}{i}\PY{p}{)} \PY{k}{for} \PY{n}{i} \PY{o+ow}{in} \PY{n+nb}{range}\PY{p}{(}\PY{n}{n}\PY{o}{+}\PY{l+m+mi}{1}\PY{p}{)}\PY{p}{]}\PY{p}{)}\PY{o}{/}\PY{n+nb}{len}\PY{p}{(}\PY{n}{tests}\PY{p}{)}
             \PY{n}{theoretical\PYZus{}quantiles} \PY{o}{=} \PY{n}{numpy}\PY{o}{.}\PY{n}{cumsum}\PY{p}{(}\PY{n}{binomial\PYZus{}dist}\PY{p}{(}\PY{n}{n}\PY{p}{,}\PY{n}{p}\PY{p}{)}\PY{p}{)}
             
             \PY{n}{ax}\PY{o}{.}\PY{n}{step}\PY{p}{(}\PY{n+nb}{range}\PY{p}{(}\PY{n}{n}\PY{o}{+}\PY{l+m+mi}{1}\PY{p}{)}\PY{p}{,} \PY{n}{quantiles}\PY{p}{,} \PY{n}{color}\PY{o}{=}\PY{l+s+s2}{\PYZdq{}}\PY{l+s+s2}{darkorchid}\PY{l+s+s2}{\PYZdq{}}\PY{p}{,} \PY{n}{zorder}\PY{o}{=}\PY{l+m+mi}{0}\PY{p}{)}
             \PY{n}{ax}\PY{o}{.}\PY{n}{step}\PY{p}{(}\PY{n+nb}{range}\PY{p}{(}\PY{n}{n}\PY{o}{+}\PY{l+m+mi}{1}\PY{p}{)}\PY{p}{,} \PY{n}{theoretical\PYZus{}quantiles}\PY{p}{,} \PY{n}{color}\PY{o}{=}\PY{l+s+s2}{\PYZdq{}}\PY{l+s+s2}{orange}\PY{l+s+s2}{\PYZdq{}}\PY{p}{,} \PY{n}{zorder}\PY{o}{=}\PY{l+m+mi}{1}\PY{p}{)}
             \PY{n}{ax}\PY{o}{.}\PY{n}{set\PYZus{}xticklabels}\PY{p}{(}\PY{n+nb}{range}\PY{p}{(}\PY{n}{n}\PY{o}{+}\PY{l+m+mi}{1}\PY{p}{)}\PY{p}{)}
             \PY{n}{ax}\PY{o}{.}\PY{n}{set\PYZus{}xticks}\PY{p}{(}\PY{n+nb}{range}\PY{p}{(}\PY{n}{n}\PY{o}{+}\PY{l+m+mi}{1}\PY{p}{)}\PY{p}{)}
             \PY{n}{ax}\PY{o}{.}\PY{n}{set\PYZus{}title}\PY{p}{(}\PY{l+s+s2}{\PYZdq{}}\PY{l+s+s2}{n=}\PY{l+s+si}{\PYZob{}\PYZcb{}}\PY{l+s+s2}{ and p=}\PY{l+s+si}{\PYZob{}\PYZcb{}}\PY{l+s+s2}{\PYZdq{}}\PY{o}{.}\PY{n}{format}\PY{p}{(}\PY{n}{n}\PY{p}{,}\PY{n}{p}\PY{p}{)}\PY{p}{)}
             \PY{k}{del} \PY{n}{tests}\PY{p}{,} \PY{n}{fig}\PY{p}{,} \PY{n}{ax}
         
         \PY{n}{plot\PYZus{}binomial}\PY{p}{(}\PY{n}{n}\PY{o}{=}\PY{l+m+mi}{25}\PY{p}{,}\PY{n}{p}\PY{o}{=}\PY{l+m+mf}{0.5}\PY{p}{,}\PY{n}{steps}\PY{o}{=}\PY{l+m+mi}{1000}\PY{p}{)}
\end{Verbatim}


    \begin{center}
    \adjustimage{max size={0.9\linewidth}{0.9\paperheight}}{output_13_0.png}
    \end{center}
    { \hspace*{\fill} \\}
    
    \begin{Verbatim}[commandchars=\\\{\}]
{\color{incolor}In [{\color{incolor}191}]:} \PY{c+c1}{\PYZsh{} numpy ha il suo interno gìà la binomiale \PYZsh{}\PYZsh{}\PYZsh{}}
          
          \PY{n}{n}\PY{o}{=}\PY{l+m+mi}{15}
          \PY{n}{p}\PY{o}{=}\PY{l+m+mf}{0.5}
          
          \PY{n}{tests} \PY{o}{=} \PY{p}{[}\PY{n}{numpy}\PY{o}{.}\PY{n}{random}\PY{o}{.}\PY{n}{binomial}\PY{p}{(}\PY{n}{n}\PY{p}{,}\PY{n}{p}\PY{p}{)} \PY{k}{for} \PY{n}{i} \PY{o+ow}{in} \PY{n+nb}{range}\PY{p}{(}\PY{l+m+mi}{10}\PY{o}{*}\PY{o}{*}\PY{l+m+mi}{4}\PY{p}{)}\PY{p}{]}
          
          \PY{n}{fig}\PY{p}{,} \PY{n}{ax} \PY{o}{=} \PY{n}{plt}\PY{o}{.}\PY{n}{subplots}\PY{p}{(}\PY{n}{figsize}\PY{o}{=}\PY{p}{(}\PY{l+m+mi}{15}\PY{p}{,}\PY{l+m+mi}{10}\PY{p}{)}\PY{p}{)}
          \PY{n}{ax}\PY{o}{.}\PY{n}{hist}\PY{p}{(}\PY{n}{tests}\PY{p}{,} \PY{n}{bins}\PY{o}{=}\PY{n}{numpy}\PY{o}{.}\PY{n}{arange}\PY{p}{(}\PY{n}{n}\PY{o}{+}\PY{l+m+mi}{2}\PY{p}{)}\PY{o}{\PYZhy{}}\PY{l+m+mf}{0.5}\PY{p}{,} \PY{n}{density}\PY{o}{=}\PY{k+kc}{True}\PY{p}{,} 
                  \PY{n}{align}\PY{o}{=}\PY{l+s+s2}{\PYZdq{}}\PY{l+s+s2}{mid}\PY{l+s+s2}{\PYZdq{}}\PY{p}{,} \PY{n}{color}\PY{o}{=}\PY{l+s+s2}{\PYZdq{}}\PY{l+s+s2}{darkorchid}\PY{l+s+s2}{\PYZdq{}}\PY{p}{,} \PY{n}{rwidth}\PY{o}{=}\PY{l+m+mf}{0.7}\PY{p}{)}
          \PY{n}{ax}\PY{o}{.}\PY{n}{bar}\PY{p}{(}\PY{n+nb}{range}\PY{p}{(}\PY{n}{n}\PY{o}{+}\PY{l+m+mi}{1}\PY{p}{)}\PY{p}{,} \PY{n}{binomial\PYZus{}dist}\PY{p}{(}\PY{n}{n}\PY{p}{,}\PY{n}{p}\PY{p}{)}\PY{p}{,} 
                 \PY{n}{align}\PY{o}{=}\PY{l+s+s2}{\PYZdq{}}\PY{l+s+s2}{center}\PY{l+s+s2}{\PYZdq{}}\PY{p}{,} \PY{n}{color}\PY{o}{=}\PY{l+s+s2}{\PYZdq{}}\PY{l+s+s2}{orange}\PY{l+s+s2}{\PYZdq{}}\PY{p}{,} \PY{n}{width}\PY{o}{=}\PY{l+m+mf}{0.1}\PY{p}{)}
          \PY{n}{ax}\PY{o}{.}\PY{n}{set\PYZus{}xticklabels}\PY{p}{(}\PY{n+nb}{range}\PY{p}{(}\PY{n}{n}\PY{o}{+}\PY{l+m+mi}{1}\PY{p}{)}\PY{p}{)}
          \PY{n}{ax}\PY{o}{.}\PY{n}{set\PYZus{}xticks}\PY{p}{(}\PY{n+nb}{range}\PY{p}{(}\PY{n}{n}\PY{o}{+}\PY{l+m+mi}{1}\PY{p}{)}\PY{p}{)}
          \PY{n}{ax}\PY{o}{.}\PY{n}{legend}\PY{p}{(}\PY{p}{(}\PY{l+s+s2}{\PYZdq{}}\PY{l+s+s2}{empirical}\PY{l+s+s2}{\PYZdq{}}\PY{p}{,} \PY{l+s+s2}{\PYZdq{}}\PY{l+s+s2}{theoretical}\PY{l+s+s2}{\PYZdq{}}\PY{p}{)}\PY{p}{)}
          \PY{k}{del} \PY{n}{tests}
\end{Verbatim}


    \begin{center}
    \adjustimage{max size={0.9\linewidth}{0.9\paperheight}}{output_14_0.png}
    \end{center}
    { \hspace*{\fill} \\}
    
    In ultimo vediamo cosa succede al qqplot di una binomiale rispetto ad
una normale.\\
Nella seguente simulazione vengono fatte 10000 campionamenti dalla
nostra binomiale prima con \texttt{n=10} poi \texttt{n=25} a seguire
\texttt{n=50} e infine \texttt{n=100} .\\
cosa accadra per n che tende a infinito?

    \begin{Verbatim}[commandchars=\\\{\}]
{\color{incolor}In [{\color{incolor}179}]:} \PY{k+kn}{import} \PY{n+nn}{statsmodels}\PY{n+nn}{.}\PY{n+nn}{api} \PY{k}{as} \PY{n+nn}{sm}
          \PY{k+kn}{import} \PY{n+nn}{scipy}\PY{n+nn}{.}\PY{n+nn}{stats} \PY{k}{as} \PY{n+nn}{stats}
          
          \PY{n}{fig}\PY{p}{,} \PY{n}{ax} \PY{o}{=} \PY{n}{plt}\PY{o}{.}\PY{n}{subplots}\PY{p}{(}\PY{n}{nrows}\PY{o}{=}\PY{l+m+mi}{2}\PY{p}{,} \PY{n}{ncols}\PY{o}{=}\PY{l+m+mi}{2}\PY{p}{)}
          \PY{n}{fig}\PY{o}{.}\PY{n}{set\PYZus{}size\PYZus{}inches}\PY{p}{(}\PY{l+m+mi}{15}\PY{p}{,}\PY{l+m+mi}{15}\PY{p}{)}
          
          \PY{n}{n}\PY{p}{,} \PY{n}{p}\PY{p}{,} \PY{n}{t} \PY{o}{=} \PY{l+m+mi}{10}\PY{p}{,} \PY{l+m+mf}{0.5}\PY{p}{,} \PY{l+m+mi}{10}\PY{o}{*}\PY{o}{*}\PY{l+m+mi}{4}
          \PY{n}{tests} \PY{o}{=} \PY{n}{numpy}\PY{o}{.}\PY{n}{random}\PY{o}{.}\PY{n}{binomial}\PY{p}{(}\PY{n}{n}\PY{p}{,} \PY{n}{p}\PY{p}{,} \PY{n}{t}\PY{p}{)}
          \PY{n}{plot} \PY{o}{=} \PY{n}{sm}\PY{o}{.}\PY{n}{qqplot}\PY{p}{(}\PY{n}{tests}\PY{p}{,} \PY{n}{dist}\PY{o}{=}\PY{n}{stats}\PY{o}{.}\PY{n}{distributions}\PY{o}{.}\PY{n}{norm}\PY{p}{(}\PY{n}{n}\PY{o}{*}\PY{n}{p}\PY{p}{,}\PY{p}{(}\PY{n}{n}\PY{o}{*}\PY{n}{p}\PY{o}{*}\PY{p}{(}\PY{l+m+mi}{1}\PY{o}{\PYZhy{}}\PY{n}{p}\PY{p}{)}\PY{p}{)}\PY{o}{*}\PY{o}{*}\PY{l+m+mf}{0.5}\PY{p}{)}\PY{p}{,} 
                           \PY{n}{line}\PY{o}{=}\PY{l+s+s1}{\PYZsq{}}\PY{l+s+s1}{45}\PY{l+s+s1}{\PYZsq{}}\PY{p}{,} \PY{n}{ax}\PY{o}{=}\PY{n}{ax}\PY{p}{[}\PY{l+m+mi}{0}\PY{p}{]}\PY{p}{[}\PY{l+m+mi}{0}\PY{p}{]}\PY{p}{)}
          \PY{n}{ax}\PY{p}{[}\PY{l+m+mi}{0}\PY{p}{]}\PY{p}{[}\PY{l+m+mi}{0}\PY{p}{]}\PY{o}{.}\PY{n}{set\PYZus{}title}\PY{p}{(}\PY{l+s+s2}{\PYZdq{}}\PY{l+s+s2}{n=}\PY{l+s+si}{\PYZob{}\PYZcb{}}\PY{l+s+s2}{ and p=}\PY{l+s+si}{\PYZob{}\PYZcb{}}\PY{l+s+s2}{\PYZdq{}}\PY{o}{.}\PY{n}{format}\PY{p}{(}\PY{l+m+mi}{10}\PY{p}{,}\PY{l+m+mf}{0.5}\PY{p}{)}\PY{p}{)}
          
          \PY{n}{n}\PY{p}{,} \PY{n}{p}\PY{p}{,} \PY{n}{t} \PY{o}{=} \PY{l+m+mi}{25}\PY{p}{,} \PY{l+m+mf}{0.5}\PY{p}{,} \PY{l+m+mi}{10}\PY{o}{*}\PY{o}{*}\PY{l+m+mi}{4}
          \PY{n}{tests} \PY{o}{=} \PY{n}{numpy}\PY{o}{.}\PY{n}{random}\PY{o}{.}\PY{n}{binomial}\PY{p}{(}\PY{n}{n}\PY{p}{,} \PY{n}{p}\PY{p}{,} \PY{n}{t}\PY{p}{)}
          \PY{n}{plot} \PY{o}{=} \PY{n}{sm}\PY{o}{.}\PY{n}{qqplot}\PY{p}{(}\PY{n}{tests}\PY{p}{,} \PY{n}{dist}\PY{o}{=}\PY{n}{stats}\PY{o}{.}\PY{n}{distributions}\PY{o}{.}\PY{n}{norm}\PY{p}{(}\PY{n}{n}\PY{o}{*}\PY{n}{p}\PY{p}{,}\PY{p}{(}\PY{n}{n}\PY{o}{*}\PY{n}{p}\PY{o}{*}\PY{p}{(}\PY{l+m+mi}{1}\PY{o}{\PYZhy{}}\PY{n}{p}\PY{p}{)}\PY{p}{)}\PY{o}{*}\PY{o}{*}\PY{l+m+mf}{0.5}\PY{p}{)}\PY{p}{,} 
                           \PY{n}{line}\PY{o}{=}\PY{l+s+s1}{\PYZsq{}}\PY{l+s+s1}{45}\PY{l+s+s1}{\PYZsq{}}\PY{p}{,} \PY{n}{ax}\PY{o}{=}\PY{n}{ax}\PY{p}{[}\PY{l+m+mi}{0}\PY{p}{]}\PY{p}{[}\PY{l+m+mi}{1}\PY{p}{]}\PY{p}{)}
          \PY{n}{ax}\PY{p}{[}\PY{l+m+mi}{0}\PY{p}{]}\PY{p}{[}\PY{l+m+mi}{1}\PY{p}{]}\PY{o}{.}\PY{n}{set\PYZus{}title}\PY{p}{(}\PY{l+s+s2}{\PYZdq{}}\PY{l+s+s2}{n=}\PY{l+s+si}{\PYZob{}\PYZcb{}}\PY{l+s+s2}{ and p=}\PY{l+s+si}{\PYZob{}\PYZcb{}}\PY{l+s+s2}{\PYZdq{}}\PY{o}{.}\PY{n}{format}\PY{p}{(}\PY{l+m+mi}{25}\PY{p}{,}\PY{l+m+mf}{0.5}\PY{p}{)}\PY{p}{)}
          
          \PY{n}{n}\PY{p}{,} \PY{n}{p}\PY{p}{,} \PY{n}{t} \PY{o}{=} \PY{l+m+mi}{50}\PY{p}{,} \PY{l+m+mf}{0.5}\PY{p}{,} \PY{l+m+mi}{10}\PY{o}{*}\PY{o}{*}\PY{l+m+mi}{4}
          \PY{n}{tests} \PY{o}{=} \PY{n}{numpy}\PY{o}{.}\PY{n}{random}\PY{o}{.}\PY{n}{binomial}\PY{p}{(}\PY{n}{n}\PY{p}{,} \PY{n}{p}\PY{p}{,} \PY{n}{t}\PY{p}{)}
          \PY{n}{plot} \PY{o}{=} \PY{n}{sm}\PY{o}{.}\PY{n}{qqplot}\PY{p}{(}\PY{n}{tests}\PY{p}{,} \PY{n}{dist}\PY{o}{=}\PY{n}{stats}\PY{o}{.}\PY{n}{distributions}\PY{o}{.}\PY{n}{norm}\PY{p}{(}\PY{n}{n}\PY{o}{*}\PY{n}{p}\PY{p}{,}\PY{p}{(}\PY{n}{n}\PY{o}{*}\PY{n}{p}\PY{o}{*}\PY{p}{(}\PY{l+m+mi}{1}\PY{o}{\PYZhy{}}\PY{n}{p}\PY{p}{)}\PY{p}{)}\PY{o}{*}\PY{o}{*}\PY{l+m+mf}{0.5}\PY{p}{)}\PY{p}{,} 
                           \PY{n}{line}\PY{o}{=}\PY{l+s+s1}{\PYZsq{}}\PY{l+s+s1}{45}\PY{l+s+s1}{\PYZsq{}}\PY{p}{,} \PY{n}{ax}\PY{o}{=}\PY{n}{ax}\PY{p}{[}\PY{l+m+mi}{1}\PY{p}{]}\PY{p}{[}\PY{l+m+mi}{0}\PY{p}{]}\PY{p}{)}
          \PY{n}{ax}\PY{p}{[}\PY{l+m+mi}{1}\PY{p}{]}\PY{p}{[}\PY{l+m+mi}{0}\PY{p}{]}\PY{o}{.}\PY{n}{set\PYZus{}title}\PY{p}{(}\PY{l+s+s2}{\PYZdq{}}\PY{l+s+s2}{n=}\PY{l+s+si}{\PYZob{}\PYZcb{}}\PY{l+s+s2}{ and p=}\PY{l+s+si}{\PYZob{}\PYZcb{}}\PY{l+s+s2}{\PYZdq{}}\PY{o}{.}\PY{n}{format}\PY{p}{(}\PY{l+m+mi}{50}\PY{p}{,}\PY{l+m+mf}{0.5}\PY{p}{)}\PY{p}{)}
          
          \PY{n}{n}\PY{p}{,} \PY{n}{p}\PY{p}{,} \PY{n}{t} \PY{o}{=} \PY{l+m+mi}{100}\PY{p}{,} \PY{l+m+mf}{0.5}\PY{p}{,} \PY{l+m+mi}{10}\PY{o}{*}\PY{o}{*}\PY{l+m+mi}{4}
          \PY{n}{tests} \PY{o}{=} \PY{n}{numpy}\PY{o}{.}\PY{n}{random}\PY{o}{.}\PY{n}{binomial}\PY{p}{(}\PY{n}{n}\PY{p}{,} \PY{n}{p}\PY{p}{,} \PY{n}{t}\PY{p}{)}
          \PY{n}{plot} \PY{o}{=} \PY{n}{sm}\PY{o}{.}\PY{n}{qqplot}\PY{p}{(}\PY{n}{tests}\PY{p}{,} \PY{n}{dist}\PY{o}{=}\PY{n}{stats}\PY{o}{.}\PY{n}{distributions}\PY{o}{.}\PY{n}{norm}\PY{p}{(}\PY{n}{n}\PY{o}{*}\PY{n}{p}\PY{p}{,}\PY{p}{(}\PY{n}{n}\PY{o}{*}\PY{n}{p}\PY{o}{*}\PY{p}{(}\PY{l+m+mi}{1}\PY{o}{\PYZhy{}}\PY{n}{p}\PY{p}{)}\PY{p}{)}\PY{o}{*}\PY{o}{*}\PY{l+m+mf}{0.5}\PY{p}{)}\PY{p}{,} 
                           \PY{n}{line}\PY{o}{=}\PY{l+s+s1}{\PYZsq{}}\PY{l+s+s1}{45}\PY{l+s+s1}{\PYZsq{}}\PY{p}{,} \PY{n}{ax}\PY{o}{=}\PY{n}{ax}\PY{p}{[}\PY{l+m+mi}{1}\PY{p}{]}\PY{p}{[}\PY{l+m+mi}{1}\PY{p}{]}\PY{p}{)}
          \PY{n}{ax}\PY{p}{[}\PY{l+m+mi}{1}\PY{p}{]}\PY{p}{[}\PY{l+m+mi}{1}\PY{p}{]}\PY{o}{.}\PY{n}{set\PYZus{}title}\PY{p}{(}\PY{l+s+s2}{\PYZdq{}}\PY{l+s+s2}{n=}\PY{l+s+si}{\PYZob{}\PYZcb{}}\PY{l+s+s2}{ and p=}\PY{l+s+si}{\PYZob{}\PYZcb{}}\PY{l+s+s2}{\PYZdq{}}\PY{o}{.}\PY{n}{format}\PY{p}{(}\PY{l+m+mi}{100}\PY{p}{,}\PY{l+m+mf}{0.5}\PY{p}{)}\PY{p}{)}
\end{Verbatim}


\begin{Verbatim}[commandchars=\\\{\}]
{\color{outcolor}Out[{\color{outcolor}179}]:} Text(0.5, 1.0, 'n=100 and p=0.5')
\end{Verbatim}
            
    \begin{center}
    \adjustimage{max size={0.9\linewidth}{0.9\paperheight}}{output_16_1.png}
    \end{center}
    { \hspace*{\fill} \\}
    
    \begin{Verbatim}[commandchars=\\\{\}]
{\color{incolor}In [{\color{incolor}182}]:} \PY{c+c1}{\PYZsh{} lets use the sliders \PYZsh{}\PYZsh{}\PYZsh{}}
          
          \PY{k+kn}{from} \PY{n+nn}{\PYZus{}\PYZus{}future\PYZus{}\PYZus{}} \PY{k}{import} \PY{n}{print\PYZus{}function}
          \PY{k+kn}{from} \PY{n+nn}{ipywidgets} \PY{k}{import} \PY{n}{interact}\PY{p}{,} \PY{n}{interactive}\PY{p}{,} \PY{n}{fixed}\PY{p}{,} \PY{n}{interact\PYZus{}manual}
          \PY{k+kn}{import} \PY{n+nn}{ipywidgets} \PY{k}{as} \PY{n+nn}{widgets}
          
          
          \PY{k}{def} \PY{n+nf}{plot\PYZus{}binomial}\PY{p}{(}\PY{n}{n}\PY{o}{=}\PY{l+m+mi}{5}\PY{p}{,}\PY{n}{p}\PY{o}{=}\PY{l+m+mf}{0.5}\PY{p}{,}\PY{n}{steps}\PY{o}{=}\PY{l+m+mi}{10}\PY{o}{*}\PY{o}{*}\PY{l+m+mi}{5}\PY{p}{)}\PY{p}{:}
              \PY{n}{rv} \PY{o}{=} \PY{n}{get\PYZus{}binomial\PYZus{}random\PYZus{}variable}\PY{p}{(}\PY{n}{n}\PY{o}{=}\PY{n}{n}\PY{p}{,}\PY{n}{p}\PY{o}{=}\PY{n}{p}\PY{p}{)}
              \PY{n}{tests} \PY{o}{=} \PY{p}{[}\PY{n}{rv}\PY{p}{(}\PY{p}{)} \PY{k}{for} \PY{n}{i} \PY{o+ow}{in} \PY{n+nb}{range}\PY{p}{(}\PY{n}{steps}\PY{p}{)}\PY{p}{]}
              \PY{n}{fig}\PY{p}{,} \PY{n}{ax} \PY{o}{=} \PY{n}{plt}\PY{o}{.}\PY{n}{subplots}\PY{p}{(}\PY{l+m+mi}{1}\PY{p}{,}\PY{l+m+mi}{2}\PY{p}{,}\PY{n}{figsize}\PY{o}{=}\PY{p}{(}\PY{l+m+mi}{20}\PY{p}{,}\PY{l+m+mi}{10}\PY{p}{)}\PY{p}{)}
              
              \PY{n}{quantiles} \PY{o}{=} \PY{n}{numpy}\PY{o}{.}\PY{n}{cumsum}\PY{p}{(}\PY{p}{[}\PY{n}{tests}\PY{o}{.}\PY{n}{count}\PY{p}{(}\PY{n}{i}\PY{p}{)} \PY{k}{for} \PY{n}{i} \PY{o+ow}{in} \PY{n+nb}{range}\PY{p}{(}\PY{n}{n}\PY{o}{+}\PY{l+m+mi}{1}\PY{p}{)}\PY{p}{]}\PY{p}{)}\PY{o}{/}\PY{n+nb}{len}\PY{p}{(}\PY{n}{tests}\PY{p}{)}
              \PY{n}{theoretical\PYZus{}quantiles} \PY{o}{=} \PY{n}{numpy}\PY{o}{.}\PY{n}{cumsum}\PY{p}{(}\PY{n}{binomial\PYZus{}dist}\PY{p}{(}\PY{n}{n}\PY{p}{,}\PY{n}{p}\PY{p}{)}\PY{p}{)}
              
              \PY{n}{ax}\PY{p}{[}\PY{l+m+mi}{0}\PY{p}{]}\PY{o}{.}\PY{n}{step}\PY{p}{(}\PY{n+nb}{range}\PY{p}{(}\PY{n}{n}\PY{o}{+}\PY{l+m+mi}{1}\PY{p}{)}\PY{p}{,} \PY{n}{quantiles}\PY{p}{,} \PY{n}{color}\PY{o}{=}\PY{l+s+s2}{\PYZdq{}}\PY{l+s+s2}{darkorchid}\PY{l+s+s2}{\PYZdq{}}\PY{p}{,} \PY{n}{zorder}\PY{o}{=}\PY{l+m+mi}{0}\PY{p}{)}
              \PY{n}{ax}\PY{p}{[}\PY{l+m+mi}{0}\PY{p}{]}\PY{o}{.}\PY{n}{step}\PY{p}{(}\PY{n+nb}{range}\PY{p}{(}\PY{n}{n}\PY{o}{+}\PY{l+m+mi}{1}\PY{p}{)}\PY{p}{,} \PY{n}{theoretical\PYZus{}quantiles}\PY{p}{,} \PY{n}{color}\PY{o}{=}\PY{l+s+s2}{\PYZdq{}}\PY{l+s+s2}{orange}\PY{l+s+s2}{\PYZdq{}}\PY{p}{,} \PY{n}{zorder}\PY{o}{=}\PY{l+m+mi}{1}\PY{p}{)}
              \PY{n}{ax}\PY{p}{[}\PY{l+m+mi}{0}\PY{p}{]}\PY{o}{.}\PY{n}{set\PYZus{}title}\PY{p}{(}\PY{l+s+s2}{\PYZdq{}}\PY{l+s+s2}{n=}\PY{l+s+si}{\PYZob{}\PYZcb{}}\PY{l+s+s2}{ and p=}\PY{l+s+si}{\PYZob{}\PYZcb{}}\PY{l+s+s2}{\PYZdq{}}\PY{o}{.}\PY{n}{format}\PY{p}{(}\PY{n}{n}\PY{p}{,}\PY{n}{p}\PY{p}{)}\PY{p}{)}
          
              \PY{n}{ax}\PY{p}{[}\PY{l+m+mi}{1}\PY{p}{]}\PY{o}{.}\PY{n}{hist}\PY{p}{(}\PY{n}{tests}\PY{p}{,} \PY{n}{bins}\PY{o}{=}\PY{n}{numpy}\PY{o}{.}\PY{n}{arange}\PY{p}{(}\PY{n}{n}\PY{o}{+}\PY{l+m+mi}{2}\PY{p}{)}\PY{o}{\PYZhy{}}\PY{l+m+mf}{0.5}\PY{p}{,} \PY{n}{density}\PY{o}{=}\PY{k+kc}{True}\PY{p}{,} \PY{n}{align}\PY{o}{=}\PY{l+s+s2}{\PYZdq{}}\PY{l+s+s2}{mid}\PY{l+s+s2}{\PYZdq{}}\PY{p}{,} 
                         \PY{n}{color}\PY{o}{=}\PY{l+s+s2}{\PYZdq{}}\PY{l+s+s2}{darkorchid}\PY{l+s+s2}{\PYZdq{}}\PY{p}{,} \PY{n}{rwidth}\PY{o}{=}\PY{l+m+mi}{1}\PY{p}{,} \PY{n}{zorder}\PY{o}{=}\PY{l+m+mi}{0}\PY{p}{)}
              \PY{n}{ax}\PY{p}{[}\PY{l+m+mi}{1}\PY{p}{]}\PY{o}{.}\PY{n}{bar}\PY{p}{(}\PY{n+nb}{range}\PY{p}{(}\PY{n}{n}\PY{o}{+}\PY{l+m+mi}{1}\PY{p}{)}\PY{p}{,} \PY{n}{binomial\PYZus{}dist}\PY{p}{(}\PY{n}{n}\PY{p}{,}\PY{n}{p}\PY{p}{)}\PY{p}{,} \PY{n}{align}\PY{o}{=}\PY{l+s+s2}{\PYZdq{}}\PY{l+s+s2}{center}\PY{l+s+s2}{\PYZdq{}}\PY{p}{,} 
                        \PY{n}{color}\PY{o}{=}\PY{l+s+s2}{\PYZdq{}}\PY{l+s+s2}{orange}\PY{l+s+s2}{\PYZdq{}}\PY{p}{,} \PY{n}{zorder}\PY{o}{=}\PY{l+m+mi}{1}\PY{p}{,} \PY{n}{width}\PY{o}{=}\PY{l+m+mf}{0.2}\PY{p}{)}
              \PY{n}{ax}\PY{p}{[}\PY{l+m+mi}{1}\PY{p}{]}\PY{o}{.}\PY{n}{set\PYZus{}title}\PY{p}{(}\PY{l+s+s2}{\PYZdq{}}\PY{l+s+s2}{n=}\PY{l+s+si}{\PYZob{}\PYZcb{}}\PY{l+s+s2}{ and p=}\PY{l+s+si}{\PYZob{}\PYZcb{}}\PY{l+s+s2}{\PYZdq{}}\PY{o}{.}\PY{n}{format}\PY{p}{(}\PY{n}{n}\PY{p}{,}\PY{n}{p}\PY{p}{)}\PY{p}{)}
          
              \PY{n}{fig}\PY{p}{,} \PY{n}{ax} \PY{o}{=} \PY{n}{plt}\PY{o}{.}\PY{n}{subplots}\PY{p}{(}\PY{l+m+mi}{1}\PY{p}{,}\PY{l+m+mi}{1}\PY{p}{,}\PY{n}{figsize}\PY{o}{=}\PY{p}{(}\PY{l+m+mi}{10}\PY{p}{,}\PY{l+m+mi}{10}\PY{p}{)}\PY{p}{)}
              \PY{n}{plot} \PY{o}{=} \PY{n}{sm}\PY{o}{.}\PY{n}{qqplot}\PY{p}{(}\PY{n}{numpy}\PY{o}{.}\PY{n}{array}\PY{p}{(}\PY{n}{tests}\PY{p}{)}\PY{p}{,} 
                               \PY{n}{dist}\PY{o}{=}\PY{n}{stats}\PY{o}{.}\PY{n}{distributions}\PY{o}{.}\PY{n}{norm}\PY{p}{(}\PY{n}{n}\PY{o}{*}\PY{n}{p}\PY{p}{,}\PY{p}{(}\PY{n}{n}\PY{o}{*}\PY{n}{p}\PY{o}{*}\PY{p}{(}\PY{l+m+mi}{1}\PY{o}{\PYZhy{}}\PY{n}{p}\PY{p}{)}\PY{p}{)}\PY{o}{*}\PY{o}{*}\PY{l+m+mf}{0.5}\PY{p}{)}\PY{p}{,}
                               \PY{n}{line}\PY{o}{=}\PY{l+s+s1}{\PYZsq{}}\PY{l+s+s1}{45}\PY{l+s+s1}{\PYZsq{}}\PY{p}{,} \PY{n}{ax}\PY{o}{=}\PY{n}{ax}\PY{p}{)}
              \PY{n}{ax}\PY{o}{.}\PY{n}{set\PYZus{}title}\PY{p}{(}\PY{l+s+s2}{\PYZdq{}}\PY{l+s+s2}{n=}\PY{l+s+si}{\PYZob{}\PYZcb{}}\PY{l+s+s2}{ and p=}\PY{l+s+si}{\PYZob{}\PYZcb{}}\PY{l+s+s2}{\PYZdq{}}\PY{o}{.}\PY{n}{format}\PY{p}{(}\PY{l+m+mi}{100}\PY{p}{,}\PY{l+m+mf}{0.5}\PY{p}{)}\PY{p}{)}
              
          \PY{n}{interact}\PY{p}{(}\PY{n}{plot\PYZus{}binomial}\PY{p}{,} \PY{n}{n}\PY{o}{=}\PY{n}{widgets}\PY{o}{.}\PY{n}{IntSlider}  \PY{p}{(}\PY{n+nb}{min}\PY{o}{=}\PY{l+m+mi}{0}\PY{p}{,} \PY{n+nb}{max}\PY{o}{=}\PY{l+m+mi}{1000} \PY{p}{,} \PY{n}{step}\PY{o}{=}\PY{l+m+mi}{1}   \PY{p}{,} 
                                                        \PY{n}{value}\PY{o}{=}\PY{l+m+mi}{10}\PY{p}{,} 
                                                        \PY{n}{layout}\PY{o}{=}\PY{p}{\PYZob{}}\PY{l+s+s1}{\PYZsq{}}\PY{l+s+s1}{width}\PY{l+s+s1}{\PYZsq{}}\PY{p}{:} \PY{l+s+s1}{\PYZsq{}}\PY{l+s+s1}{1000px}\PY{l+s+s1}{\PYZsq{}}\PY{p}{\PYZcb{}}\PY{p}{)}\PY{p}{,}
                                  \PY{n}{p}\PY{o}{=}\PY{n}{widgets}\PY{o}{.}\PY{n}{FloatSlider}\PY{p}{(}\PY{n+nb}{min}\PY{o}{=}\PY{l+m+mi}{0}\PY{p}{,} \PY{n+nb}{max}\PY{o}{=}\PY{l+m+mi}{1}    \PY{p}{,} \PY{n}{step}\PY{o}{=}\PY{l+m+mf}{0.01}\PY{p}{,} 
                                                        \PY{n}{value}\PY{o}{=}\PY{l+m+mf}{0.5}\PY{p}{,} 
                                                        \PY{n}{layout}\PY{o}{=}\PY{p}{\PYZob{}}\PY{l+s+s1}{\PYZsq{}}\PY{l+s+s1}{width}\PY{l+s+s1}{\PYZsq{}}\PY{p}{:} \PY{l+s+s1}{\PYZsq{}}\PY{l+s+s1}{1000px}\PY{l+s+s1}{\PYZsq{}}\PY{p}{\PYZcb{}}\PY{p}{)}\PY{p}{,}
                              \PY{n}{steps}\PY{o}{=}\PY{n}{widgets}\PY{o}{.}\PY{n}{IntSlider}  \PY{p}{(}\PY{n+nb}{min}\PY{o}{=}\PY{l+m+mi}{1}\PY{p}{,} \PY{n+nb}{max}\PY{o}{=}\PY{l+m+mi}{10000}\PY{p}{,} \PY{n}{step}\PY{o}{=}\PY{l+m+mi}{1}   \PY{p}{,} 
                                                        \PY{n}{value}\PY{o}{=}\PY{l+m+mi}{100}\PY{p}{,} 
                                                        \PY{n}{layout}\PY{o}{=}\PY{p}{\PYZob{}}\PY{l+s+s1}{\PYZsq{}}\PY{l+s+s1}{width}\PY{l+s+s1}{\PYZsq{}}\PY{p}{:} \PY{l+s+s1}{\PYZsq{}}\PY{l+s+s1}{1000px}\PY{l+s+s1}{\PYZsq{}}\PY{p}{\PYZcb{}}\PY{p}{)}\PY{p}{)}\PY{p}{;}
\end{Verbatim}


    
    \begin{verbatim}
interactive(children=(IntSlider(value=10, description='n', layout=Layout(width='1000px'), max=1000), FloatSlid…
    \end{verbatim}

    

    % Add a bibliography block to the postdoc
    
    
    
    \end{document}
